\documentclass[11pt, a4paper]{article}

% Setup
\usepackage[margin=2.4cm, top=3.5cm]{geometry}
\usepackage[utf8]{inputenc}
\usepackage[ngerman]{babel}

% Package imports
\usepackage{amsfonts}
\usepackage{amsmath}
\usepackage{amssymb}
\usepackage{amsthm}
\usepackage{mathtools}
\usepackage{setspace}
\usepackage{float}
\usepackage{enumitem}
\usepackage{hyperref}
\usepackage[pagestyles]{titlesec}
\usepackage{fancyhdr}
\usepackage{colonequals}
\usepackage{caption}
\usepackage{tikz}
\usepackage{marginnote}
\usepackage{etoolbox}
\usepackage{mdframed}
\usepackage{aligned-overset}
\usepackage{esint}
\usepackage{scalerel}

% Font-Encoding
\usepackage[T1]{fontenc}
\usepackage{lmodern}

% TikZ packages
\usetikzlibrary{patterns}

% Theorems
\newtheoremstyle{plain}{}{}{}{}{\bfseries}{.}{ }{}
\theoremstyle{plain}
\newtheorem{blockelement}{Blockelement}[subsection]
\newtheorem{bemerkung}[blockelement]{Bemerkung}
\newtheorem{definition}[blockelement]{Definition}
\newtheorem{lemma}[blockelement]{Lemma}
\newtheorem{satz}[blockelement]{Satz}
\newtheorem{notation}[blockelement]{Notation}
\newtheorem{korollar}[blockelement]{Korollar}
\newtheorem{uebung}[blockelement]{Übung}
\newtheorem{beispiel}[blockelement]{Beispiel}
\newtheorem{folgerung}[blockelement]{Folgerung}
\newtheorem{axiom}[blockelement]{Axiom}
\newtheorem{beobachtung}[blockelement]{Beobachtung}
\newtheorem{konzept}[blockelement]{Konzept}
\newtheorem{konstruktion}[blockelement]{Konstruktion}
\newtheorem{visualisierung}[blockelement]{Visualisierung}
\newtheorem{anwendung}[blockelement]{Anwendung}
\newtheorem{skizze}[blockelement]{Skizze}
\newtheorem{konvention}[blockelement]{Konvention}
\newtheorem{genv}[blockelement]{}

% Numbering (equations and conditions)
\numberwithin{equation}{subsection}
\newcommand{\numbereq}[1]{\addtocounter{equation}{1}\tag{\theequation}\label{#1}}
\newcounter{condition}
\renewcommand{\thecondition}{V\arabic{condition}}
\newcommand{\condition}[1]{\hypertarget{#1}{\refstepcounter{condition}(\label{#1}\thecondition})}

\DeclareMathAlphabet{\altmathbb}{U}{BOONDOX-ds}{m}{n}

% Long equations
\allowdisplaybreaks

% \left \right
\newcommand{\set}[1]{\left\{#1\right\}}
\newcommand{\pair}[1]{\left(#1\right)}
\newcommand{\of}[1]{\mathopen{}\mathclose{}\bgroup\left(#1\aftergroup\egroup\right)}
\newcommand{\abs}[1]{\left\lvert#1\right\rvert}
\newcommand{\norm}[1]{\left\lVert#1\right\rVert}
\newcommand{\linterv}[1]{\left[#1\right)}
\newcommand{\rinterv}[1]{\left(#1\right]}
\newcommand{\interv}[1]{\left[#1\right]}
\newcommand{\scalprod}[1]{\left<#1\right>}

% Shorten commands
\newcommand{\equivalent}[0]{\Leftrightarrow{}}
\newcommand{\impl}[0]{\Rightarrow{}}
\newcommand{\definedasequiv}[0]{\ratio\Leftrightarrow{}}
\renewcommand{\emptyset}{\varnothing}
\newcommand{\dif}{\mathop{}\!\mathrm{d}}
\newcommand{\Dif}{\mathop{}\!\mathrm{D}}
\newcommand{\dsty}{\displaystyle}
\newcommand{\charfunc}{\altmathbb{1}}

\newcommand{\toinf}{\to\infty}
\newcommand{\fa}{\;\forall}
\newcommand{\ex}{\;\exists}
\newcommand{\conj}[1]{\overline{#1}}
\newcommand{\comp}[1]{{#1}^{\mathrm{C}}}

\newcommand{\annot}[3][]{\overset{\text{#3}}#1{#2}}
\newcommand{\anf}[1]{\glqq{}#1\grqq}
\newcommand{\OBDA}{o.B.d.A. }
\newcommand{\theoremescape}{\leavevmode}
\newcommand{\aligntoright}[2]{\hfill#1\hspace{#2\textwidth}~}
\newcommand{\horizontalline}[0]{\par\noindent\rule{0.05\textwidth}{0.1pt}\\}
\newcommand{\rgbcolor}[3]{rgb,255:red,#1;green,#2;blue,#3}
\newcommand{\fixedspace}[2]{\makebox[#1][l]{#2}}

\let\Re\relax
\let\Im\relax

% MathOperators
\DeclareMathOperator{\grad}{Grad}
\DeclareMathOperator{\bild}{Bild}
\DeclareMathOperator{\Re}{Re}
\DeclareMathOperator{\Im}{Im}
\DeclareMathOperator{\arcsinh}{arcsinh}
\DeclareMathOperator{\arccosh}{arccosh}
\DeclareMathOperator{\diam}{diam}
\DeclareMathOperator{\fehler}{Fehler}
\DeclareMathOperator{\D}{D\!}
\DeclareMathOperator{\Id}{Id}
\DeclareMathOperator{\op}{op}
\DeclareMathOperator{\rank}{rk}
\DeclareMathOperator{\spann}{Spann}
\DeclareMathOperator{\flaeche}{Fläche}
\DeclareMathOperator{\bew}{Bew}
\DeclareMathOperator{\diag}{diag}
\DeclareMathOperator{\fdiv}{div}

% Mengenbezeichner
\newcommand{\R}{\mathbb{R}}
\newcommand{\N}{\mathbb{N}}
\newcommand{\C}{\mathbb{C}}
\newcommand{\Z}{\mathbb{Z}}
\newcommand{\Q}{\mathbb{Q}}
\newcommand{\K}{\mathbb{K}}

\newcommand{\mA}{\mathcal{A}}
\newcommand{\mB}{\mathcal{B}}
\newcommand{\mC}{\mathcal{C}}
\newcommand{\mD}{\mathcal{D}}
\newcommand{\mE}{\mathcal{E}}
\newcommand{\mF}{\mathcal{F}}
\newcommand{\mG}{\mathcal{G}}
\newcommand{\mH}{\mathcal{H}}
\newcommand{\mJ}{\mathcal{J}}
\newcommand{\mK}{\mathcal{K}}
\newcommand{\mL}{\mathcal{L}}
\newcommand{\mM}{\mathcal{M}}
\newcommand{\mO}{\mathcal{O}}
\newcommand{\mP}{\mathcal{P}}
\newcommand{\mQ}{\mathcal{Q}}
\newcommand{\mR}{\mathcal{R}}
\newcommand{\mS}{\mathcal{S}}
\newcommand{\mPC}{\mathcal{PC}}

\reversemarginpar

% Spezielle Symbole
\NewDocumentCommand{\Tau}{e{^_}}{
    \scalerel*{\tau}{X}
    \IfValueT{#1}{^{#1}}
    \IfValueT{#2}{_{\!\!#2}}
}

% Spezielle Commands
\newcommand\subseccount[0]{
    \refstepcounter{subsection}
}

% Envs
\newenvironment{induktionsanfang}{
    \rule{0pt}{3ex}\noindent
    \begin{minipage}[t]{0.11\textwidth}
    {I-Anfang}
    \end{minipage}
    \hfill
    \begin{minipage}[t]{0.89\textwidth}
    }
    {
    \end{minipage}
}
\newenvironment{induktionsvoraussetzung}{
    \rule{0pt}{3ex}\noindent
    \begin{minipage}[t]{0.11\textwidth}
    {I-Vor.}
    \end{minipage}
    \hfill
    \begin{minipage}[t]{0.89\textwidth}
    }
    {
    \end{minipage}
}
\newenvironment{induktionsschritt}{
    \rule{0pt}{3ex}\noindent
    \begin{minipage}[t]{0.11\textwidth}
    {I-Schritt}
    \end{minipage}
    \hfill
    \begin{minipage}[t]{0.89\textwidth}
    }
    {
    \end{minipage}
}

% Section style
\titleformat*{\section}{\LARGE\bfseries}
\titleformat*{\subsection}{\large\bfseries}

% Page styles
\newpagestyle{sectionpage}{
    \sethead{}{}{}
    \setfoot{}{\thepage}{Version: \today}
}
\newpagestyle{headfootdefault}{
    \sethead{\thesection~\textit{\sectiontitle}}{}{\thesubsection~\textit{\subsectiontitle}}
    \setfoot{}{\thepage}{Version: \today}
}
\pagestyle{headfootdefault}

\begin{document}
    \title{\vspace{3cm} Skript zur Vorlesung\\Analysis IV\\bei Prof. Dr. Dirk Hundertmark}
    \author{Karlsruher Institut für Technologie}
    \date{Sommersemester 2025}
    \maketitle
    \begin{center}
        Dieses Skript ist inoffiziell. Es besteht kein\\Anspruch auf Vollständigkeit oder Korrektheit.
    \end{center}
    \newpage
    \thispagestyle{sectionpage}

    \tableofcontents
    ~\\
    Alle mit [*] markierten Kapitel sind noch nicht Korrektur gelesen und bedürfen eventuell noch Änderungen.

    \newpage


    \section{[*] Erinnerungen/Rückblick}
    \thispagestyle{sectionpage}

    \subsection{Komplexe Zahlen $\C$}

    \begin{bemerkung}[$\C$ ist ein Körper]
        \marginnote{[22. Apr]}
        Wir kennen bereits die komplexen Zahlen. Wir betrachten eine komplexe Zahl als Tupel $\pair{a,b} \in\R\times\R$ mit Addition
        \begin{align*}
            \pair{a,b} + \pair{c,d} &\coloneqq \pair{a+c, b+d}
            \intertext{sowie Multiplikation}
            \pair{a, b}\cdot\pair{c,d} &\coloneqq \pair{ac - bd, ad+bc}
        \end{align*}
        Durch Nachrechnen zeigt sich, dass $\C$ so die Körperaxiome erfüllt, wobei $\pair{0, 0}$ bzw. $\pair{1, 0}$ die neutralen Elemente bezüglich Addition bzw. Multiplikation sind. Für die herkömmliche Darstellung der komplexen Zahlen definieren wir außerdem $i \coloneqq \pair{0, 1}$. Visualisieren lässt sich das dann in der \textit{Gaußschen Zahlenebene}.
    \end{bemerkung}

    \begin{bemerkung}
        Sei $z = \pair{a, b} \in\R\times\R$. Dann gilt $z = \pair{a, 0} + \pair{0, b} = a + bi$. Wir können also alle komplexen Zahlen in der Form $a + bi$ schreiben.
    \end{bemerkung}

    \begin{definition}[Komplexe Konjugation]
        Wir definieren außerdem die komplexe Konjugation: Sei wieder $z = a + bi$. Dann ist die komplexe Konjugation von $z$ definiert durch $\conj{z} \coloneqq a - bi$. Damit ergibt sich die multiplikative Inverse $z^{-1} = \frac{\conj{z}}{\abs{z}^2}$, die sich leicht durch Nachrechnen bestätigen lässt. Die additive Inverse $\pair{-a, -b}$ ergibt sich direkt aus der Definition der Addition.
    \end{definition}

    \begin{definition}[Real- und Komplexteil]
        Sei $z = a + bi$, $a, b\in\R$. Dann schreiben wir $\Re\of{z} = a$ sowie $\Im\of{z} = b$. Außerdem gilt dann
        \begin{align*}
            \Re\of{z} &= \frac{1}{2}\pair{z+\conj{z}}\\
            \Im\of{z} &= \frac{1}{i}\pair{z-\conj{z}}
        \end{align*}
    \end{definition}

    \begin{satz}[Cauchy-Schwarz für $\C$]
        Seien $z, w\in\C$. Dann gilt $\Re\of{\conj{z}w} \leq \abs{z}\abs{w}$.

        \begin{proof}
            \textit{(fehlt)}
        \end{proof}
    \end{satz}

    \subsection{Konvergenz}
    \begin{definition}[Konvergenz]
        Sei $(z_n)_n \subseteq\C$ eine Folge. Dann konvergiert diese gegen $z$, falls
        \begin{align*}
            \lim_{n\toinf} \abs{z - z_n} &= 0
            \intertext{Wir schreiben dann $\lim_{n\toinf} z_n = z$ oder $z_n \to z$ für $n\toinf$. Äquivalent dazu ist die Bedingung}
            \forall \varepsilon > 0 \ex N_{\varepsilon}\in\N\colon \abs{z - z_n} &< \varepsilon\quad\forall n\geq N_{\varepsilon}
        \end{align*}
    \end{definition}

    \begin{definition}[Cauchy-Folgen]
        Wir nennen $(z_n)_n$ eine Cauchy-Folge, falls
        \begin{align*}
            \forall\varepsilon > 0\ex N_{\varepsilon}\in\N\colon \abs{z_n - z_m} < \varepsilon\quad\forall n,m > N_{\varepsilon}
        \end{align*}
    \end{definition}

    \begin{satz}[Vollständigkeit von $\C$]
        Die Folge $(z_n)_n$ konvergiert genau dann, wenn $(z_n)_n$ eine Cauchy-Folge ist.
        \begin{proof}
        (Nicht hier, siehe Ana 3)
            .
        \end{proof}
    \end{satz}

    \begin{bemerkung}[Konvergenz von Reihen]
        Eine Reihe $ \sum_{n=1}^{\infty} z_n$ konvergiert per Definition, wenn die Folge der Partialsummen $(s_n)_n$, $s_n \coloneqq \sum_{j=1}^{n} z_j$ konvergiert. Notwendig für die Konvergenz von $ \sum_{j=1}^{\infty} z_j$ ist dabei, dass $z_n \to 0$. Hinreichend ist z.B., dass $ \sum_{n=1}^{\infty} \abs{z_n}$ konvergiert. In diesem Fall sprechen wir von absoluter Konvergenz.
    \end{bemerkung}

    \subsection{Ein paar Definitionen}

    \begin{definition}[Topologische Grundlagen: Offene und abgeschlossene Mengen, Rand und Abschluss]
        Wir definieren die (offene) $\varepsilon$-Scheibe um $z$
        \begin{align*}
            D_{\varepsilon}\of{z} &\coloneqq \set{w\in\C: \abs{z-w} < \varepsilon}
            \intertext{sowie den $\varepsilon$-Kreis um $z$}
            C_{\varepsilon}\of{z} &\coloneqq \set{w: \abs{z-w} = \varepsilon}
            \intertext{Eine Menge $S\subseteq\C$ heißt damit offen, falls}
            \forall z\in S\ex r > 0\colon &D_{r}\of{z} \subseteq S
            \intertext{Es sei $\comp{S} \coloneqq \C\setminus S$. Dann nennen wir $S$ abgeschlossen, falls $\comp{S}$ offen ist. Wir definieren außerdem noch den Rand von $S$}
            \partial S &\coloneqq \set{z\in\C: \forall \varepsilon > 0\colon S \cap D_{\varepsilon}\of{z} \neq \emptyset \land \comp{S}\cap D_{\varepsilon}\of{z} \neq \emptyset}
            \intertext{Damit definieren wir außerdem den Abschluss von $S$}
            \overline{S} &\coloneqq S \cup \partial S
        \end{align*}
        Wir sagen $S$ ist beschränkt, falls $S \subseteq D_{R}\of{0}$ für ein $R > 0$. Außerdem ist $S$ kompakt, falls $S$ sowohl abgeschlossen als auch beschränkt ist.\endgraf\noindent $S$ ist nicht-zusammenhängend, falls es offene disjunkte Mengen $A, B$ gibt mit $S \subseteq A \cup B$ mit $S \cap A \neq \emptyset$, $S\cap B \neq \emptyset$. $S$ ist zusammenhängend, falls es nicht nicht-zusammenhängend ist.
    \end{definition}

    \begin{definition}
        Sei $z, w \in\C$. Dann definieren wir $\interv{z, w} \coloneqq \set{\pair{1- \Theta}z + \Theta w: 0 \leq \Theta \leq 1}$ als die Strecke zwischen $z$ und $w$. Wir sagen eine Menge $S\subseteq\C$ ist polygonal zusammenhängend, falls es einen polygonalen Weg zwischen jeder Kombination von zwei Punkten $a, b \in S$ gibt. Das heißt es gibt $z_{1, \ldots, n}$ sodass
        \begin{align*}
            \interv{a, z_1} \cup \interv{z_1, z_2} \cup \dots \cup \interv{z_{n-1}, z_{n}}\cup \interv{z_n, b} \subseteq S
        \end{align*}
    \end{definition}

    \begin{definition}
        Eine offene, zusammenhängende Menge heißt Gebiet.
    \end{definition}

    \begin{satz} % Satz 5
        Sei $U$ offen. Dann ist $U$ genau dann zusammenhängend, wenn es polygonal zusammenhängend ist.
    \end{satz}


    \newpage


    \section{[*] Analytische Polynome}
    \thispagestyle{sectionpage}
    \subseccount

    \textbf{Motivation.} Sei $P\of{x, y} = \dsty\sum_{r=0}^{N_1} \sum_{s=0}^{N_2} \alpha_{r, s} x^r y ^s$ mit $x, y\in\R$, $a_n\in\C$. Wir sagen $P$ ist analytisch, wenn es ein Polynom in $x + yi$ ist. Das heißt $P\of{x, y} = \dsty\sum_{n=0}^{L} \alpha_n \pair{x+ iy}^n = f\of{x+yi}$ für passende $a_n$. Frage: Wann ist ein Polynom analytisch?

    \begin{beispiel}
        \marginnote{[28. Apr]}
        \label{beispiel:polynom-1}
        \theoremescape
        \begin{enumerate}[label=(\roman*)]
            \item Das Polynom $P(x,y) = x^2 - y^2 + 2ixy$ ist analytisch, da $P(x,y) = \pair{x+iy}^2$.
            \item $P(x,y) = x^2-y^2- 2ixy$ ist nicht analytisch.
        \end{enumerate}
        \begin{proof}[Beweis für (ii)]
            Angenommen
            \begin{align*}
                x^2 - y^2 -2ixy &= \sum_{k=0}^{N} a_k\pair{x+iy}^k
                \intertext{Dann gilt für $y = 0$}
                x^2 &= \sum_{k=0}^{N} a_k x^k\\
                \intertext{Damit gilt nach Koeffizientenvergleich}
                \impl \alpha_0 = \alpha_1 &= 0 = \alpha_3 = \ldots = \alpha_N\\
                \alpha_2 &= 1\qedhere
            \end{align*}
        \end{proof}
    \end{beispiel}

    \begin{definition}[Parteielle Ableitung]
        Es sei $f: \R^2 \to \C$ mit
        \begin{align*}
            f\of{x,y} &= \begin{pmatrix}
                             u\of{x,y} \\
                             v\of{x,y}
            \end{pmatrix} = u\of{x,y} + v\of{x,y}i
            \intertext{Dann definieren wir die partiellen Ableitungen von $f$ wie folgt}
            f_{x} &\coloneqq \partial_x f = \partial_x u + i \partial_x v = u_x + i v_x\\
            f_y &\coloneqq \partial_y f = \partial_y u + i \partial_y v = u_y + i v_y
        \end{align*}
    \end{definition}

    \begin{satz} % Satz 2
        \label{satz:analytisch-diff}
        Ein Polynom $P\of{x,y}$ ist genau dann analytisch, wenn $\partial_y P = i\partial_x P$.

        \begin{proof}
            \anf{$\impl$}
            \begin{align*}
                P\of{x,y} &= \sum_{n=0}^{N} a_n \pair{x+iy}^n\\
                \partial_x P &= \sum_{n=0}^{N} \alpha_n \partial_x\pair{x+iy}^n = \sum_{n=0}^{N} \alpha_n n\pair{x+iy}^{n-1}\\
                \partial_y P &= \sum_{n=0}^{N} \alpha_n \partial_y\pair{x+iy}^n = \sum_{n=0}^{N} \alpha_n ni \pair{x+iy}^{n-1}
            \end{align*}
            \anf{$\Leftarrow$} Sei $\partial_x P = i\partial_y P$. Dann ist
            \begin{align*}
                P\of{x,y} &= \sum_{r=0}^{N_1} \sum_{s=0}^{N_2} \alpha_{r,s} x^r y^s\\
                &= \sum_{n=0}^{N_1 + N_2} \sum_{0\leq r \leq N_1,~0\leq s \leq N_2,~s+r = n}^{} \alpha_{r,s} x^r y^s\\
                &= \sum_{n=0}^{N_1 + N_2} \underbrace{\sum_{0\leq t \leq L}^{} \alpha_{n,t} x^{n-t} y^t}_{\eqqcolon G_n\of{x,y}}\\
                \impl G_n\of{\lambda x, \lambda y} &= \lambda^n G\of{x,y}
                \intertext{Nach unserer Voraussetzung gilt für die $G_n$ auch die Gleichung über die partiellen Ableitungen. Das heißt für festes $n$}
                \partial_y G_n\of{x, y} &= \sum_{t=1}^{n} t \alpha_t x^{n-t} y^{t-1}\\
                i \partial_x G_n\of{x,y} &= i \pair{ \sum_{t=0}^{n-1} \pair{n-t} \alpha_t x^{n-t-1} y^t}\\
                \intertext{Da die linken Seiten der Gleichungen übereinstimmen, gilt auch}
                c_1 x^{n-1} + 2c_2 x^{n-2}y + \ldots + nc_n y^{n-1} &= i\pair{nc_0 x^{n-1} + \pair{n-1}c_1 x^{n-2}y + \ldots + c_{n-1}y^{n-1}}
                \intertext{Nach Koeffizientenvergleich gilt damit}
                c_1 &= i n c_0 = i\binom{n}{1}c_0\\
                c_2 &= i^2 \frac{n\pair{n-1}}{2}c_0 = i^2 \binom{i^2}{2}c_0
                \intertext{Induktiv setzt sich das fort zu}
                c_k &= i^k \binom{n}{k}c_0
                \intertext{Damit gilt}
                G_n\of{x,y} &= \sum_{k=0}^{n} c_k x^{n-k} y^k = \sum_{k=0}^{n} i^k \binom{n}{k}c_0 x^{n-k}y^k\\
                &= c_0 \sum_{k=0}^{n} \binom{n}{k}x^{n-k}\pair{iy}^k = c_k \pair{x+iy}^n\\
                \impl P\of{x,y} &= \sum_{n=0}^{N} G_n\of{x,y} = \sum_{n=0}^{N} \alpha_{n,0} \pair{x+iy}^n
            \end{align*}
            Das heißt $P$ ist analytisch.\qedhere
        \end{proof}
    \end{satz}

    \begin{bemerkung}
        Beispiel~\ref{beispiel:polynom-1} lässt sich jetzt mit Satz~\ref{satz:analytisch-diff} auch direkter ohne Koeffizientenvergleich nachrechnen.
    \end{bemerkung}
    \newpage


    \section{[*] Stenographische Projektion}
    \thispagestyle{sectionpage}
    \subseccount

    \textbf{Motivation.} Es sei $\sum \coloneqq \set{ \pair{\xi, \eta, \zeta}\in\R^3: \xi^2 + \eta^2 + \pair{\zeta - \frac{1}{2}}^2 = \frac{1}{4}}$ die Sphäre im $\R^3$ mit Radius $\frac{1}{2}$ um den Punkt $\pair{0, 0, \frac{1}{2}}$. Dann lässt sich eine Abbildung $\pair{\xi, \eta, \zeta} \in \sum \setminus\set{\pair{0, 0, 1}} \mapsto z\in\C = \R^2$ definieren. Wobei $z$ der Schnittpunkt der Geraden durch Nordpol und $\pair{\xi, \eta, \zeta}$ ist.\\
    Sei
    \begin{align*}
        \lambda \pair{\pair{\xi, \eta, \zeta} - \pair{0, 0, 1}} &= \pair{x, y, 0} - \pair{0, 0, 1} = \pair{x,y,-1}
        \intertext{Dann gilt}
        \lambda\xi = x,~\lambda\eta = y, \lambda\pair{\zeta - 1} &= -1\\
        \impl \lambda &= \frac{1}{1-\zeta}\\
        \impl \frac{x}{\xi} = \lambda &= \frac{y}{\eta}\\
        \impl x &= \frac{\xi}{1-\zeta},~y= \frac{\eta}{1-\zeta}
        \intertext{Sind $x,y$ gegeben. Dann gilt}
        \xi &= \frac{x}{x^2+y^2+1}\tag{1}\\
        \eta &= \frac{y}{x^2+y^2+1}\tag{2}\\
        \zeta &= \frac{x^2+y^2}{x^2+y^2+1}\tag{3}
    \end{align*}
    (Selber machen).

    \begin{definition}
        Sei $(z_n)_n \subseteq \C$. Wir schreiben $z_n \toinf$, falls $\abs{z_n}\toinf$ für $n\toinf$ sowie $f\of{z_n}\toinf$, falls $\abs{f\of{z_n}}\toinf$ für $n\toinf$.
    \end{definition}

    \begin{bemerkung}[Zusammenhang von Kreisen in $\sum$ und $\C$]
        Ein Kreis in $\sum$ ist ein Schnitt von $\sum$ mit einer Ebene im $\R^3$ der Form $A\xi + B\eta + C\xi = D$. Dann folgt nach (1)-(3)
        \begin{align*}
            D &= A \frac{x}{x^2+y^2+1} + B \frac{y}{x^2+y^2+1} + C \frac{x^2+y^2}{x^2+y^2+1}\\
            \equivalent D &= \pair{C - D}\pair{x^2+y^2} + Ax + By
        \end{align*}
        \textsc{Fall 1}: $C = D$. Dann ist $Ax + By = D$ eine Linie in $\C$.\\
        \textsc{Fall 2}: $C \neq D$ $\impl$ Kreis in $\R^2 = \C$.
    \end{bemerkung}
    Damit wäre der folgende Satz bewiesen:
    \begin{satz}
        (\textit{Sinngemäß: Kreise in $\sum$ werden stenographisch auf Geraden projeziert.})
    \end{satz}

    \newpage


    \section{[*] Komplexe Differenzierbarkeit}
    \thispagestyle{sectionpage}
    \subseccount

    \begin{definition}
        Wir sagen $f: \C \to \C$ ist (komplex) differenzierbar in $z_0$, falls
        \begin{align*}
            \lim_{h\to 0} \frac{f\of{z_0 + h} - f\of{z_0}}{h} &\eqqcolon f'\of{z_0}
        \end{align*}
        existiert. (Dabei ist zu beachten, dass $h$ in $\C$ gegen $0$ konvergiert)
    \end{definition}

    \begin{folgerung}[Cauchy-Riemannsche Differentialgleichung]
        \theoremescape
        \begin{enumerate}
            \item Setze $h = t \in\R\setminus\set{0}$ (das heißt wir wählen eine Folge von $h$, die in den reellen Zahlen gegen $0$ konvergiert) und $z_0 = x_0 + iy_0$
            \begin{align*}
                \frac{f\of{z_0 + h } - f\of{z_0}}{h} &= \frac{f\of{z_0 + t} - f\of{z_0}}{t} = \frac{f\of{x_0 + t + iy_0} - f\of{x_0 + iy_0}}{t}\\
                &= \frac{f\of{x_0 + t, y_0} - f\of{x_0, y_0}}{t} \to \partial_x f\of{x_0, y_0} = f_x\of{x_0, y_0} = f_x\of{z_0}
            \end{align*}
            \item Setze $h= it$, $t\in\R\setminus\set{0}$
            \begin{align*}
                \frac{f\of{z_0+h} - f\of{z_0}}{h} &= \frac{f\of{x_0 + iy + it} - f\of{x_0 + y_0}}{it}\\
                &= \frac{1}{i} \frac{f\of{x_0, y_ + t} - f\of{x_0, y_0}}{t} \to \frac{1}{i} \partial_y f\of{z_0}
            \end{align*}
        \end{enumerate}
        Ist die Funktion $f$ (komplex) diffbar, dann müssen die Grenzwerte übereinstimmen und es muss gelten
        \begin{align*}
            \partial_y f\of{z_0} &= i \partial_x f\of{z_0}
            \intertext{Wenn $f= u + iv$ für Funktionen $u$ und $v$, dann lässt sich äquivalent auch fordern}
            \partial_y u = -\partial_x v \quad\text{und}&\quad \partial_y v = \partial_x u
        \end{align*}
    \end{folgerung}


\end{document}
