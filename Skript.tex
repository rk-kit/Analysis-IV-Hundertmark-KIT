\documentclass[11pt, a4paper]{article}

% Setup
\usepackage[margin=2.4cm, top=3.5cm]{geometry}
\usepackage[utf8]{inputenc}
\usepackage[ngerman]{babel}

% Package imports
\usepackage{amsfonts}
\usepackage{amsmath}
\usepackage{amssymb}
\usepackage{amsthm}
\usepackage{mathtools}
\usepackage{setspace}
\usepackage{float}
\usepackage{enumitem}
\usepackage{hyperref}
\usepackage[pagestyles]{titlesec}
\usepackage{fancyhdr}
\usepackage{colonequals}
\usepackage{caption}
\usepackage{tikz}
\usepackage{marginnote}
\usepackage{etoolbox}
\usepackage{mdframed}
\usepackage{aligned-overset}
\usepackage{esint}
\usepackage{scalerel}

% Font-Encoding
\usepackage[T1]{fontenc}
\usepackage{lmodern}

% TikZ packages
\usetikzlibrary{patterns}

% Theorems
\newtheoremstyle{plain}{}{}{}{}{\bfseries}{.}{ }{}
\theoremstyle{plain}
\newtheorem{blockelement}{Blockelement}[subsection]
\newtheorem{bemerkung}[blockelement]{Bemerkung}
\newtheorem{definition}[blockelement]{Definition}
\newtheorem{lemma}[blockelement]{Lemma}
\newtheorem{satz}[blockelement]{Satz}
\newtheorem{notation}[blockelement]{Notation}
\newtheorem{korollar}[blockelement]{Korollar}
\newtheorem{uebung}[blockelement]{Übung}
\newtheorem{beispiel}[blockelement]{Beispiel}
\newtheorem{folgerung}[blockelement]{Folgerung}
\newtheorem{axiom}[blockelement]{Axiom}
\newtheorem{beobachtung}[blockelement]{Beobachtung}
\newtheorem{konzept}[blockelement]{Konzept}
\newtheorem{konstruktion}[blockelement]{Konstruktion}
\newtheorem{visualisierung}[blockelement]{Visualisierung}
\newtheorem{anwendung}[blockelement]{Anwendung}
\newtheorem{skizze}[blockelement]{Skizze}
\newtheorem{konvention}[blockelement]{Konvention}
\newtheorem{genv}[blockelement]{}

% Numbering (equations and conditions)
\numberwithin{equation}{subsection}
\newcommand{\numbereq}[1]{\addtocounter{equation}{1}\tag{\theequation}\label{#1}}
\newcounter{condition}
\renewcommand{\thecondition}{V\arabic{condition}}
\newcommand{\condition}[1]{\hypertarget{#1}{\refstepcounter{condition}(\label{#1}\thecondition})}

\DeclareMathAlphabet{\altmathbb}{U}{BOONDOX-ds}{m}{n}

% Long equations
\allowdisplaybreaks

% \left \right
\newcommand{\set}[1]{\left\{#1\right\}}
\newcommand{\pair}[1]{\left(#1\right)}
\newcommand{\of}[1]{\mathopen{}\mathclose{}\bgroup\left(#1\aftergroup\egroup\right)}
\newcommand{\abs}[1]{\left\lvert#1\right\rvert}
\newcommand{\norm}[1]{\left\lVert#1\right\rVert}
\newcommand{\linterv}[1]{\left[#1\right)}
\newcommand{\rinterv}[1]{\left(#1\right]}
\newcommand{\interv}[1]{\left[#1\right]}
\newcommand{\scalprod}[1]{\left<#1\right>}

% Shorten commands
\newcommand{\equivalent}[0]{\Leftrightarrow{}}
\newcommand{\impl}[0]{\Rightarrow{}}
\newcommand{\definedasequiv}[0]{\ratio\Leftrightarrow{}}
\renewcommand{\emptyset}{\varnothing}
\newcommand{\dif}{\mathop{}\!\mathrm{d}}
\newcommand{\Dif}{\mathop{}\!\mathrm{D}}
\newcommand{\dsty}{\displaystyle}
\newcommand{\charfunc}{\altmathbb{1}}

\newcommand{\toinf}{\to\infty}
\newcommand{\fa}{\;\forall}
\newcommand{\ex}{\;\exists}
\newcommand{\conj}[1]{\overline{#1}}
\newcommand{\comp}[1]{{#1}^{\mathrm{C}}}

\newcommand{\annot}[3][]{\overset{\text{#3}}#1{#2}}
\newcommand{\anf}[1]{\glqq{}#1\grqq}
\newcommand{\OBDA}{o.B.d.A. }
\newcommand{\theoremescape}{\leavevmode}
\newcommand{\aligntoright}[2]{\hfill#1\hspace{#2\textwidth}~}
\newcommand{\horizontalline}[0]{\par\noindent\rule{0.05\textwidth}{0.1pt}\\}
\newcommand{\rgbcolor}[3]{rgb,255:red,#1;green,#2;blue,#3}
\newcommand{\fixedspace}[2]{\makebox[#1][l]{#2}}
\newcommand{\verteq}{\rotatebox{-90}{$~=$}}
\newcommand{\equalto}[2]{\underset{\scriptstyle\overset{\mkern4mu\verteq}}{#1}}

\let\Re\relax
\let\Im\relax

% MathOperators
\DeclareMathOperator{\grad}{Grad}
\DeclareMathOperator{\bild}{Bild}
\DeclareMathOperator{\Re}{Re}
\DeclareMathOperator{\Im}{Im}
\DeclareMathOperator{\arcsinh}{arcsinh}
\DeclareMathOperator{\arccosh}{arccosh}
\DeclareMathOperator{\diam}{diam}
\DeclareMathOperator{\fehler}{Fehler}
\DeclareMathOperator{\D}{D\!}
\DeclareMathOperator{\Id}{Id}
\DeclareMathOperator{\op}{op}
\DeclareMathOperator{\rank}{rk}
\DeclareMathOperator{\spann}{Spann}
\DeclareMathOperator{\flaeche}{Fläche}
\DeclareMathOperator{\bew}{Bew}
\DeclareMathOperator{\diag}{diag}
\DeclareMathOperator{\fdiv}{div}
\DeclareMathOperator{\Arg}{Arg}
\DeclareMathOperator{\Res}{Res}
\DeclareMathOperator{\laenge}{Länge}
\DeclareMathOperator{\sing}{Sing}

% Mengenbezeichner
\newcommand{\R}{\mathbb{R}}
\newcommand{\N}{\mathbb{N}}
\newcommand{\C}{\mathbb{C}}
\newcommand{\Z}{\mathbb{Z}}
\newcommand{\Q}{\mathbb{Q}}
\newcommand{\K}{\mathbb{K}}

\newcommand{\mA}{\mathcal{A}}
\newcommand{\mB}{\mathcal{B}}
\newcommand{\mC}{\mathcal{C}}
\newcommand{\mD}{\mathcal{D}}
\newcommand{\mE}{\mathcal{E}}
\newcommand{\mF}{\mathcal{F}}
\newcommand{\mG}{\mathcal{G}}
\newcommand{\mH}{\mathcal{H}}
\newcommand{\mJ}{\mathcal{J}}
\newcommand{\mK}{\mathcal{K}}
\newcommand{\mL}{\mathcal{L}}
\newcommand{\mM}{\mathcal{M}}
\newcommand{\mO}{\mathcal{O}}
\newcommand{\mP}{\mathcal{P}}
\newcommand{\mQ}{\mathcal{Q}}
\newcommand{\mR}{\mathcal{R}}
\newcommand{\mS}{\mathcal{S}}
\newcommand{\mPC}{\mathcal{PC}}

\reversemarginpar

% Spezielle Symbole
\NewDocumentCommand{\Tau}{e{^_}}{
    \scalerel*{\tau}{X}
    \IfValueT{#1}{^{#1}}
    \IfValueT{#2}{_{\!\!#2}}
}

% Spezielle Commands
\newcommand\subseccount[0]{
    \refstepcounter{subsection}
}

% Envs
\newenvironment{induktionsanfang}{
    \rule{0pt}{3ex}\noindent
    \begin{minipage}[t]{0.11\textwidth}
    {I-Anfang}
    \end{minipage}
    \hfill
    \begin{minipage}[t]{0.89\textwidth}
    }
    {
    \end{minipage}
}
\newenvironment{induktionsvoraussetzung}{
    \rule{0pt}{3ex}\noindent
    \begin{minipage}[t]{0.11\textwidth}
    {I-Vor.}
    \end{minipage}
    \hfill
    \begin{minipage}[t]{0.89\textwidth}
    }
    {
    \end{minipage}
}
\newenvironment{induktionsschritt}{
    \rule{0pt}{3ex}\noindent
    \begin{minipage}[t]{0.11\textwidth}
    {I-Schritt}
    \end{minipage}
    \hfill
    \begin{minipage}[t]{0.89\textwidth}
    }
    {
    \end{minipage}
}

% Section style
\titleformat*{\section}{\LARGE\bfseries}
\titleformat*{\subsection}{\large\bfseries}

% Page styles
\newpagestyle{sectionpage}{
    \sethead{}{}{}
    \setfoot{}{\thepage}{Version: \today}
}
\newpagestyle{headfootdefault}{
    \sethead{\thesection~\textit{\sectiontitle}}{}{\thesubsection~\textit{\subsectiontitle}}
    \setfoot{}{\thepage}{Version: \today}
}
\pagestyle{headfootdefault}

\begin{document}
    \title{\vspace{3cm} Skript zur Vorlesung\\Analysis IV\\bei Prof. Dr. Dirk Hundertmark}
    \author{Karlsruher Institut für Technologie}
    \date{Sommersemester 2025}
    \maketitle
    \begin{center}
        Dieses Skript ist inoffiziell. Es besteht kein\\Anspruch auf Vollständigkeit oder Korrektheit.
    \end{center}
    \newpage
    \thispagestyle{sectionpage}

    \tableofcontents
    ~\\
    Alle mit [*] markierten Kapitel sind noch nicht Korrektur gelesen und bedürfen eventuell noch Änderungen.

    \newpage


    \section{[*] Erinnerungen/Rückblick}
    \thispagestyle{sectionpage}

    \subsection{Komplexe Zahlen $\C$}

    \begin{bemerkung}[$\C$ ist ein Körper]
        \marginnote{[22. Apr]}
        Wir kennen bereits die komplexen Zahlen. Wir betrachten eine komplexe Zahl als Tupel $\pair{a,b} \in\R\times\R$ mit Addition
        \begin{align*}
            \pair{a,b} + \pair{c,d} &\coloneqq \pair{a+c, b+d}
            \intertext{sowie Multiplikation}
            \pair{a, b}\cdot\pair{c,d} &\coloneqq \pair{ac - bd, ad+bc}
        \end{align*}
        Durch Nachrechnen zeigt sich, dass $\C$ so die Körperaxiome erfüllt, wobei $\pair{0, 0}$ bzw. $\pair{1, 0}$ die neutralen Elemente bezüglich Addition bzw. Multiplikation sind. Für die herkömmliche Darstellung der komplexen Zahlen definieren wir außerdem $i \coloneqq \pair{0, 1}$. Visualisieren lässt sich das dann in der \textit{Gaußschen Zahlenebene}.
    \end{bemerkung}

    \begin{bemerkung}
        Sei $z = \pair{a, b} \in\R\times\R$. Dann gilt $z = \pair{a, 0} + \pair{0, b} = a + bi$. Wir können also alle komplexen Zahlen in der Form $a + bi$ schreiben.
    \end{bemerkung}

    \begin{definition}[Komplexe Konjugation]
        Wir definieren außerdem die komplexe Konjugation: Sei wieder $z = a + bi$. Dann ist die komplexe Konjugation von $z$ definiert durch $\conj{z} \coloneqq a - bi$. Damit ergibt sich die multiplikative Inverse $z^{-1} = \frac{\conj{z}}{\abs{z}^2}$, die sich leicht durch Nachrechnen bestätigen lässt. Die additive Inverse $\pair{-a, -b}$ ergibt sich direkt aus der Definition der Addition.
    \end{definition}

    \begin{definition}[Real- und Imaginärteil]
        Sei $z = a + bi$, $a, b\in\R$. Dann schreiben wir $\Re\of{z} = a$ sowie $\Im\of{z} = b$. Außerdem gilt dann
        \begin{align*}
            \Re\of{z} &= \frac{1}{2}\pair{z+\conj{z}}\\
            \Im\of{z} &= \frac{1}{2i}\pair{z-\conj{z}}
        \end{align*}
    \end{definition}

    \begin{definition}[Betrag einer komplexen Zahl]
        Sei $z = a + bi$, $a,b\in\R$. Dann definieren wir (kompatibel mit dem Betrag in $\R$) den Betrag von $z$ als
        \begin{align*}
            \abs{z} &\coloneqq \sqrt{a^2+b^2}
        \end{align*}
    \end{definition}

    \begin{satz}[Cauchy-Schwarz für $\C$]
        Seien $z, w\in\C$. Dann gilt $\Re\of{\conj{z}w} \leq \abs{z}\abs{w}$.

        \begin{proof}
            \textit{(fehlt)}
        \end{proof}
    \end{satz}

    \subsection{Konvergenz}
    \begin{definition}[Konvergenz]
        Sei $(z_n)_n \subseteq\C$ eine Folge. Dann konvergiert diese gegen $z$, falls
        \begin{align*}
            \lim_{n\toinf} \abs{z - z_n} &= 0
            \intertext{Wir schreiben dann $\lim_{n\toinf} z_n = z$ oder $z_n \to z$ für $n\toinf$. Äquivalent dazu ist die Bedingung}
            \forall \varepsilon > 0 \ex N_{\varepsilon}\in\N\colon \abs{z - z_n} &< \varepsilon\quad\forall n\geq N_{\varepsilon}
        \end{align*}
    \end{definition}

    \begin{definition}[Cauchy-Folgen]
        Wir nennen $(z_n)_n$ eine Cauchy-Folge, falls
        \begin{align*}
            \forall\varepsilon > 0\ex N_{\varepsilon}\in\N\colon \abs{z_n - z_m} < \varepsilon\quad\forall n,m > N_{\varepsilon}
        \end{align*}
    \end{definition}

    \begin{satz}[Vollständigkeit von $\C$]
        Die Folge $(z_n)_n$ konvergiert genau dann, wenn $(z_n)_n$ eine Cauchy-Folge ist.
        \begin{proof}
        (Nicht hier, siehe Ana 3)
            .
        \end{proof}
    \end{satz}

    \begin{bemerkung}[Konvergenz von Reihen]
        Eine Reihe $ \sum_{n=1}^{\infty} z_n$ konvergiert per Definition, wenn die Folge der Partialsummen $(s_n)_n$, $s_n \coloneqq \sum_{j=1}^{n} z_j$ konvergiert. Notwendig für die Konvergenz von $ \sum_{j=1}^{\infty} z_j$ ist dabei, dass $z_n \to 0$. Hinreichend ist z.B., dass $ \sum_{n=1}^{\infty} \abs{z_n}$ konvergiert. In diesem Fall sprechen wir von absoluter Konvergenz.
    \end{bemerkung}

    \subsection{Ein paar Definitionen}

    \begin{definition}[Topologische Grundlagen: Offene und abgeschlossene Mengen, Rand und Abschluss]
        Wir definieren die (offene) $\varepsilon$-Scheibe um $z$
        \begin{align*}
            D_{\varepsilon}\of{z} &\coloneqq \set{w\in\C: \abs{z-w} < \varepsilon}
            \intertext{sowie den $\varepsilon$-Kreis um $z$}
            C_{\varepsilon}\of{z} &\coloneqq \set{w: \abs{z-w} = \varepsilon}
            \intertext{Eine Menge $S\subseteq\C$ heißt damit offen, falls}
            \forall z\in S\ex r > 0&\colon D_{r}\of{z} \subseteq S
        \end{align*}
        Es sei $\comp{S} \coloneqq \C\setminus S$. Dann nennen wir $S$ abgeschlossen, falls $\comp{S}$ offen ist. Wir definieren außerdem noch den Rand von $S$
        \begin{align*}
            \partial S &\coloneqq \set{z\in\C: \forall \varepsilon > 0\colon S \cap D_{\varepsilon}\of{z} \neq \emptyset \land \comp{S}\cap D_{\varepsilon}\of{z} \neq \emptyset}
        \end{align*}
        Damit definieren wir außerdem den Abschluss von $S$
        \begin{align*}
            \overline{S} &\coloneqq S \cup \partial S
        \end{align*}
        Wir sagen $S$ ist beschränkt, falls $S \subseteq D_{R}\of{0}$ für ein $R > 0$. Außerdem ist $S$ kompakt, falls $S$ sowohl abgeschlossen als auch beschränkt ist.\endgraf\noindent $S$ ist nicht-zusammenhängend, falls es offene disjunkte Mengen $A, B$ gibt mit $S \subseteq A \cup B$ mit $S \cap A \neq \emptyset$, $S\cap B \neq \emptyset$. $S$ ist zusammenhängend, falls es nicht nicht-zusammenhängend ist.
    \end{definition}

    \begin{definition}
        Sei $z, w \in\C$. Dann definieren wir $\interv{z, w} \coloneqq \set{\pair{1- \Theta}z + \Theta w: 0 \leq \Theta \leq 1}$ als die Strecke zwischen $z$ und $w$. Wir sagen eine Menge $S\subseteq\C$ ist polygonal zusammenhängend, falls es einen polygonalen Weg zwischen jeder Kombination von zwei Punkten $a, b \in S$ gibt. Das heißt es gibt $z_{1}, \ldots z_{n}$ sodass
        \begin{align*}
            \interv{a, z_1} \cup \interv{z_1, z_2} \cup \dots \cup \interv{z_{n-1}, z_{n}}\cup \interv{z_n, b} \subseteq S
        \end{align*}
    \end{definition}

    \begin{definition}
        Eine offene, zusammenhängende Menge heißt Gebiet.
    \end{definition}

    \begin{satz} % Satz 5
        Sei $U$ offen. Dann ist $U$ genau dann zusammenhängend, wenn es polygonal zusammenhängend ist.
    \end{satz}


    \newpage


    \section{[*] Analytische Polynome}
    \thispagestyle{sectionpage}
    \subseccount

    \textbf{Motivation.} Sei $P\of{x, y} = \dsty\sum_{r=0}^{N_1} \sum_{s=0}^{N_2} \alpha_{r, s} x^r y ^s$ mit $x, y\in\R$, $a_n\in\C$. Wir sagen $P$ ist analytisch, wenn es ein Polynom in $x + yi$ ist. Das heißt $P\of{x, y} = \dsty\sum_{n=0}^{L} \alpha_n \pair{x+ iy}^n \eqqcolon f\of{x+yi}$ für passende $a_n$. Frage: Wann ist ein Polynom analytisch?

    \begin{beispiel}
        \marginnote{[28. Apr]}
        \label{beispiel:polynom-1}
        \theoremescape
        \begin{enumerate}[label=(\roman*)]
            \item Das Polynom $P(x,y) = x^2 - y^2 + 2ixy$ ist analytisch, da $P(x,y) = \pair{x+iy}^2$.
            \item $P(x,y) = x^2-y^2- 2ixy$ ist nicht analytisch.
        \end{enumerate}
        \begin{proof}[Beweis für (ii)]
            Angenommen
            \begin{align*}
                x^2 - y^2 -2ixy &= \sum_{k=0}^{N} \alpha_k\pair{x+iy}^k
                \intertext{Dann gilt für $y = 0$}
                x^2 &= \sum_{k=0}^{N} \alpha_k x^k\\
                \intertext{Damit gilt nach Koeffizientenvergleich}
                \impl \alpha_0 = \alpha_1 &= 0 = \alpha_3 = \ldots = \alpha_N\\
                \alpha_2 &= 1\\
                \impl x^2 + y^2 -2ixy &= (x +iy)^2
            \end{align*}
            Ausmultiplizieren zeigt, dass die letzte Gleichung im Allgemeinen falsch ist.
        \end{proof}
    \end{beispiel}

    \begin{definition}[Partielle Ableitung]
        Es sei $f: \R^2 \to \C$ mit
        \begin{align*}
            f\of{x,y} &= \begin{pmatrix}
                             u\of{x,y} \\
                             v\of{x,y}
            \end{pmatrix} = u\of{x,y} + v\of{x,y}i
            \intertext{Dann definieren wir die partiellen Ableitungen von $f$ wie folgt}
            f_{x} &\coloneqq \partial_x f = \partial_x u + i \partial_x v = u_x + i v_x\\
            f_y &\coloneqq \partial_y f = \partial_y u + i \partial_y v = u_y + i v_y
        \end{align*}
    \end{definition}

    \begin{satz} % Satz 2
        \label{satz:analytisch-diff}
        Ein Polynom $P\of{x,y}$ ist genau dann analytisch, wenn
        \begin{align*}
            \partial_y P = i\partial_x P\numbereq{eq:analytisch-diff}
        \end{align*}

        \begin{proof}
            \anf{$\impl$}
            \begin{align*}
                P\of{x,y} &= \sum_{n=0}^{N} a_n \pair{x+iy}^n\\
                \partial_x P &= \sum_{n=0}^{N} \alpha_n \partial_x\pair{x+iy}^n = \sum_{n=0}^{N} \alpha_n n\pair{x+iy}^{n-1}\\
                \partial_y P &= \sum_{n=0}^{N} \alpha_n \partial_y\pair{x+iy}^n = \sum_{n=0}^{N} \alpha_n ni \pair{x+iy}^{n-1}
            \end{align*}
            \anf{$\Leftarrow$} Sei $\partial_x P = i\partial_y P$. Dann ist
            \begin{align*}
                P\of{x,y} &= \sum_{r=0}^{N_1} \sum_{s=0}^{N_2} \alpha_{r,s} x^r y^s\\
                &= \sum_{n=0}^{N_1 + N_2} \sum_{\substack{0\leq r \leq N_1\\ 0\leq s \leq N_2\\ s+r = n}}^{} \alpha_{r,s} x^r y^s\\
                &= \sum_{n=0}^{N_1 + N_2} \underbrace{\sum_{t=0}^{N_1 + N_2} \alpha_{n,t} x^{n-t} y^t}_{\eqqcolon G_n\of{x,y}}\\
                \impl G_n\of{\lambda x, \lambda y} &= \lambda^n G_n\of{x,y}
                \intertext{Nach unserer Voraussetzung gilt (\ref{eq:analytisch-diff}) auch für die $G_n$. Das heißt für festes $n$ gilt mit $c_t \coloneqq \alpha_{n, t}$}
                \partial_y G_n\of{x, y} &= \sum_{t=1}^{n} t c_t x^{n-t} y^{t-1}\\
                \vspace{-2cm}
                \verteq \quad\quad &\\
                i \partial_x G_n\of{x,y} &= i \pair{ \sum_{t=0}^{n-1} \pair{n-t} c_t x^{n-t-1} y^t}\\
                \intertext{Da die linken Seiten der Gleichungen übereinstimmen, gilt auch}
                c_1 x^{n-1} + 2c_2 x^{n-2}y + \ldots + nc_n y^{n-1} &= i\pair{nc_0 x^{n-1} + \pair{n-1}c_1 x^{n-2}y + \ldots + c_{n-1}y^{n-1}}
                \intertext{Nach Koeffizientenvergleich gilt damit}
                c_1 &= i n c_0 = i\binom{n}{1}c_0\\
                c_2 &= i^2 \frac{n\pair{n-1}}{2}c_0 = i^2 \binom{n}{2}c_0
                \intertext{Induktiv setzt sich das fort zu}
                c_k &= i^k \binom{n}{k}c_0
                \intertext{Damit gilt}
                G_n\of{x,y} &= \sum_{k=0}^{n} c_k x^{n-k} y^k = \sum_{k=0}^{n} i^k \binom{n}{k}c_0 x^{n-k}y^k\\
                &= c_0 \sum_{k=0}^{n} \binom{n}{k}x^{n-k}\pair{iy}^k = c_0 \pair{x+iy}^n\\
                \impl P\of{x,y} &= \sum_{n=0}^{N} G_n\of{x,y} = \sum_{n=0}^{N} \alpha_{n,0} \pair{x+iy}^n
            \end{align*}
            Das heißt $P$ ist analytisch.\qedhere
        \end{proof}
    \end{satz}

    \begin{bemerkung}
        Beispiel~\ref{beispiel:polynom-1} lässt sich jetzt mit Satz~\ref{satz:analytisch-diff} auch direkter ohne Koeffizientenvergleich nachrechnen.
    \end{bemerkung}

    \newpage


    \section{[*] Stenographische Projektion}
    \thispagestyle{sectionpage}
    \subseccount

    \textbf{Motivation.} Es sei $\sum \coloneqq \set{ \pair{\xi, \eta, \zeta}\in\R^3: \xi^2 + \eta^2 + \pair{\zeta - \frac{1}{2}}^2 = \frac{1}{4}}$ die Sphäre im $\R^3$ mit Radius $\frac{1}{2}$ um den Punkt $\pair{0, 0, \frac{1}{2}}$. Dann lässt sich eine Abbildung $\pair{\xi, \eta, \zeta} \in \sum \setminus\set{\pair{0, 0, 1}} \mapsto z\in\C = \R^2$ definieren. Wobei $z$ der Schnittpunkt der Geraden durch Nordpol und $\pair{\xi, \eta, \zeta}$ ist.\\
    Sei
    \begin{align*}
        \lambda \pair{\pair{\xi, \eta, \zeta} - \pair{0, 0, 1}} &= \pair{x, y, 0} - \pair{0, 0, 1} = \pair{x,y,-1}
        \intertext{Dann gilt}
        \lambda\xi = x,~\lambda\eta = y, \lambda\pair{\zeta - 1} &= -1\\
        \impl \lambda &= \frac{1}{1-\zeta}\\
        \impl \frac{x}{\xi} = \lambda &= \frac{y}{\eta}\\
        \impl x &= \frac{\xi}{1-\zeta},~y= \frac{\eta}{1-\zeta}
        \intertext{Sind $x,y$ gegeben. Dann gilt}
        \xi &= \frac{x}{x^2+y^2+1}\tag{1}\\
        \eta &= \frac{y}{x^2+y^2+1}\tag{2}\\
        \zeta &= \frac{x^2+y^2}{x^2+y^2+1}\tag{3}
    \end{align*}
    (Selber machen).

    \begin{definition}
        Sei $(z_n)_n \subseteq \C$. Wir schreiben $z_n \toinf$, falls $\abs{z_n}\toinf$ für $n\toinf$ sowie $f\of{z_n}\toinf$, falls $\abs{f\of{z_n}}\toinf$ für $n\toinf$.
    \end{definition}

    \begin{bemerkung}[Zusammenhang von Kreisen in $\sum$ und $\C$]
        Ein Kreis in $\sum$ ist ein Schnitt von $\sum$ mit einer Ebene im $\R^3$ der Form $A\xi + B\eta + C\xi = D$. Dann folgt nach (1)-(3)
        \begin{align*}
            D &= A \frac{x}{x^2+y^2+1} + B \frac{y}{x^2+y^2+1} + C \frac{x^2+y^2}{x^2+y^2+1}\\
            \equivalent D &= \pair{C - D}\pair{x^2+y^2} + Ax + By
        \end{align*}
        \textsc{Fall 1}: $C = D$. Dann ist $Ax + By = D$ eine Linie in $\C$.\\
        \textsc{Fall 2}: $C \neq D$ $\impl$ Kreis in $\R^2 = \C$.
    \end{bemerkung}
    Damit wäre der folgende Satz bewiesen:
    \begin{satz}
    (\textit{Sinngemäß: Kreise in $\sum$ werden stenographisch auf Geraden projeziert.})
    \end{satz}

    \newpage


    \section{[*] Komplexe Differenzierbarkeit}

    \subsection{Die Cauchy-Riemannsche Differentialgleichung}
    \thispagestyle{sectionpage}

    \begin{definition}
        Wir sagen $f: \C \to \C$ ist (komplex) differenzierbar in $z_0$, falls
        \begin{align*}
            \lim_{h\to 0} \frac{f\of{z_0 + h} - f\of{z_0}}{h} &\eqqcolon f'\of{z_0}
        \end{align*}
        existiert. (Dabei ist zu beachten, dass $h$ in $\C$ gegen $0$ konvergiert)
    \end{definition}

    \begin{folgerung}[Cauchy-Riemannsche Differentialgleichung]
        \theoremescape
        \begin{enumerate}
            \item Setze $h = t \in\R\setminus\set{0}$ (das heißt wir wählen eine Folge von $h$, die in den reellen Zahlen gegen $0$ konvergiert) und $z_0 = x_0 + iy_0$
            \begin{align*}
                \frac{f\of{z_0 + h } - f\of{z_0}}{h} &= \frac{f\of{z_0 + t} - f\of{z_0}}{t} = \frac{f\of{x_0 + t + iy_0} - f\of{x_0 + iy_0}}{t}\\
                &= \frac{f\of{x_0 + t, y_0} - f\of{x_0, y_0}}{t} \to \partial_x f\of{x_0, y_0} = f_x\of{x_0, y_0} = f_x\of{z_0}
            \end{align*}
            \item Setze $h= it$, $t\in\R\setminus\set{0}$
            \begin{align*}
                \frac{f\of{z_0+h} - f\of{z_0}}{h} &= \frac{f\of{x_0 + iy + it} - f\of{x_0 + y_0}}{it}\\
                &= \frac{1}{i} \frac{f\of{x_0, y_ + t} - f\of{x_0, y_0}}{t} \to \frac{1}{i} \partial_y f\of{z_0}
            \end{align*}
        \end{enumerate}
        Ist die Funktion $f$ (komplex) diffbar, dann müssen die Grenzwerte übereinstimmen und es muss gelten
        \begin{align*}
            \partial_y f\of{z_0} &= i \partial_x f\of{z_0}
            \intertext{Wenn $f= u + iv$ für Funktionen $u$ und $v$, dann lässt sich äquivalent auch fordern}
            \partial_y u = -\partial_x v \quad\text{und}&\quad \partial_y v = \partial_x u
        \end{align*}
        Wir formulieren das nochmal formal als Satz:
    \end{folgerung}

    \begin{satz} % Satz 4
        \marginnote{[29. Apr]}
        \label{satz:cauchy-riemann}
        Sei $U\subseteq\C$ offen. Ist $f$ in $z\in U$ (komplex) differenzierbar, so existiert die partielle Ableitung $f_x\of{z}$ und $f_y\of{z}$ und es gilt
        \begin{align*}
            f_y\of{z} &= i f_x\of{z}
        \end{align*}
    \end{satz}

    \begin{bemerkung}
        Die Umkehrung von Satz~\ref{satz:cauchy-riemann} gilt nicht. Als Beispiel betrachten wir
        \begin{align*}
            f\of{z} &= \begin{cases}
                           \frac{xy\pair{x+iy}}{x^2+y^2} &z=x+iy\neq 0\\
                           0 & z=0
            \end{cases}
            \intertext{Dann gilt}
            f\of{x+i0} &= 0 = f\of{0+iy}\\
            \partial_x f\of{0} &= 0 = \partial_y f\of{0}
            \intertext{Man rechnet allerdings nach, dass}
            \frac{f\of{x+\alpha ix} - f\of{0}}{x + \alpha ix} &= \frac{\alpha}{1+\alpha^2}
        \end{align*}
        Das heißt für $x\to 0$ kriegen wir einen anderen Grenzwert als 0.
    \end{bemerkung}

    \begin{satz}[Ableitung von Kompositionen komplexer Funktionen]
        Sei $U\subseteq\C$ offen, $z\in U$ und $f,g$ in $z$ differenzierbar. Dann sind auch $f+g$, $f\cdot g$, $\frac{f}{g}$ (falls $g\of{z}\neq 0$) in $z$ differenzierbar und es gilt
        \begin{align*}
            \pair{f+g}'\of{z} &= f'\of{z} + g'\of{z}\\
            \pair{fg}'\of{z} &= f'\of{z}g\of{z} + f\of{z}g'\of{z}\\
            \pair{\frac{f}{g}}'\of{z} &= \frac{g\of{z}f'\of{z} - f\of{z}g'\of{z}}{g\of{z}^2}
        \end{align*}
        \begin{proof}
        (Lässt sich mit Differenzenquotient nachrechnen, siehe Analysis 1)
            .
        \end{proof}
    \end{satz}

    \begin{satz}[Ableitung von komplexen Polynomen]
        \label{satz:polynome-ableitung}
        Sei $P\of{z} = \sum_{j=0}^{n} a_j z^j$ ein Polynom in $\C$. Dann gilt $P'\of{z} = \sum_{j=1}^{n} j a_j z^{j-1}$.
        \begin{proof}
            Wir betrachten nur die Monome. Es gilt
            \begin{align*}
                \pair{z+h}^n &= \sum_{k=0}^{n} \binom{n}{k} z^{n-k}h^k\\
                \impl\pair{z+h}^{n} - z^n &= \sum_{k=1}^{n} \binom{n}{k}z^{n-k}h^k = nz^{n-1}h + \sum_{k=2}^{n} \binom{n}{k}z^{n-k}h^k\\
                \impl\lim_{h\to 0} \frac{\pair{z+h}^n-z^n}{h} &= \lim_{h\to 0} \pair{nz^{n-1} + \sum_{k=2}^{n} \binom{n}{k}z^{n-k}h^{k-1}} = nz^{n-1}\qedhere
            \end{align*}
        \end{proof}
    \end{satz}

    \begin{bemerkung}[Ableitung von komplexen Potenzreihen]
        Satz~\ref{satz:polynome-ableitung} überträgt sich auch auf komplexe Potenzreihen. Das heißt sei $f\of{z} = \sum_{n=0}^{\infty} a_n z^n$ eine Potenzreihe mit Konvergenzradius $R$. Dann ist $f$ differenzierbar innerhalb der Kreisscheibe mit $f'\of{z} = \sum_{n=1}^{\infty} n a_n z^{n-1}$.
    \end{bemerkung}

    \begin{bemerkung}[Alternative Herleitung von Cauchy-Riemann]
        Es sei $f: U \to \C$ eine komplexe Funktion mit $U\subseteq\C$ offen. Dann können wir diese auch intepretieren als Funktion $f: U \to\R^2$ mit $U\subseteq\R^2$ offen. Wir erinnern uns, dass $f$ dann im Punkt $\pair{x,y}$ differenzierbar ist, sofern es eine Abbildung $A: \R^2\to\R^2$ gibt mit
        \begin{align*}
            f\of{x+h_1, g+h_2} &= f\of{x,y} + Ah + \varepsilon\of{h}\cdot h\tag{$h\coloneqq\begin{pmatrix}
                                                                                               h_1 \\
                                                                                               h_2
            \end{pmatrix}$}
            \intertext{und $\varepsilon\of{h}\to 0$ für $h\to 0$. Wir spalten $f$ außerdem in zwei Funktionen auf. Das heißt es sei}
            f &= \begin{pmatrix}
                     u \\
                     v
            \end{pmatrix}
            \intertext{Dann gilt (sofern $f$ differenzierbar ist)}
            A &= \pair{\partial_x f, \partial_y f} = \begin{pmatrix}
                                                         \partial_x u & \partial_y u \\
                                                         \partial_x v & \partial_y v
            \end{pmatrix} = \begin{pmatrix}
                                u_x & u_y \\
                                v_x & v_y
            \end{pmatrix}
            \intertext{Frage: Wann entspricht $A$ der Multiplikation mit einer komplexen Zahl?}
            \begin{pmatrix}
                \alpha & \beta  \\
                \gamma & \delta
            \end{pmatrix}\begin{pmatrix}
                             h_1 \\
                             h_2
            \end{pmatrix} &= Ah = f'\of{z}\cdot h = \pair{a+ib}\pair{h_1 + ih_2} = ah_1 -bh_2 + i\pair{bh_1 + ah_2}\\
            \intertext{Das entspricht gerade}
            \begin{pmatrix}
                ah_1 - bh_2 \\
                bh_1 + ah_2
            \end{pmatrix} &= \begin{pmatrix}
                                 a & -b \\
                                 b & a
            \end{pmatrix}\begin{pmatrix}
                             h_1 \\
                             h_2
            \end{pmatrix}
            \intertext{Insgesamt muss für (komplexe) Differenzierbarkeit also gelten}
            \begin{pmatrix}
                a & -b \\
                b & a
            \end{pmatrix} &= \begin{pmatrix}
                                 u_x & u_y \\
                                 v_x & v_y
            \end{pmatrix}\\
            \impl u_x = v_y \quad&\text{und}\quad u_y = -v_x
        \end{align*}
    \end{bemerkung}

    \begin{satz} % Satz 5
        Angenommen die partiellen Ableitungen $f_x, f_y$ in einer Umgebung von $z$ existieren und sind stetig (und erfüllen damit die Cauchy-Riemann'sche-Differentialgleichung). Dann ist $f$ komplex differenzierbar in $z$ und es gilt
        \begin{align*}
            f'\of{z} &= f_x\of{z}
        \end{align*}

        \begin{proof}
            Sei $f = u+iv$ und $h= \xi + i\eta$. Wir müssen zeigen, dass
            \begin{align*}
                \frac{f\of{z+h} - f\of{z}}{h} &\to f_x\of{z} \text{ für } h\to 0\\
                u\of{z} = u\of{x,y}\quad&\quad v\of{z} = v\of{x,y}\\
                f\of{z+h} - f\of{z} &= u\of{x+\xi, y+\eta} - u\of{x,y} + i\pair{v\of{x+\xi, y+\eta} - v\of{x,y}}\\
                \frac{u\of{z+h} - u\of{z}}{h} &= \frac{u\of{x+\xi, y+\eta} - u\of{x,y}}{\xi + i\eta}\\
                &= \frac{u\of{x+\xi, y+\eta} - u\of{x+\xi, y} + u\of{x+\xi, y} - u\of{x,y}}{\xi + i\eta}\\
                &= \frac{u\of{x+\xi, y+\eta} - u\of{x+\xi, y}}{\xi + i\eta} + \frac{u\of{x+\xi, y} - u\of{x,y}}{\xi + i\eta}\\
                &= \frac{\eta}{\xi + i\eta} u_y\underbrace{\of{x+\xi, y+\Theta_1\eta}}_{\eqqcolon z_1} + \frac{\xi}{\xi + i\eta} u_x\underbrace{\of{x+\Theta_2 \xi, y}}_{\eqqcolon z_3}\tag{$0<\Theta_i<1$}
                \intertext{Analog}
                \frac{v\of{x+\xi, y+\eta} - v\of{x,y}}{\xi + i\eta} &= \frac{\eta}{\xi+i\eta} v_y\underbrace{\of{x+\xi, y+\Theta_3 \eta}}_{\eqqcolon z_2} + \frac{\xi}{\xi+i\eta}v_x\underbrace{\of{x+\Theta_4\eta\xi, y}}_{\eqqcolon z_4}\\
                \impl \frac{f\of{z+h} - f\of{z}}{h} &= \frac{u\of{x+\xi, y+\eta} - u\of{x,y}}{\xi+i\eta} + i \frac{v\of{x+\xi, y+\eta} - v\of{x,y}}{\xi+i\eta}\\
                &= \frac{\eta}{\xi+i\eta}\pair{u_y\of{z_1} + iv_y\of{z_2}} + \frac{\xi}{\xi+i\eta}\pair{u_x\of{z_3} + iv_x\of{z_4}}
                \intertext{Außerdem}
                f_x\of{z} &= \frac{h}{h}f_x\of{z} = \frac{\xi + i\eta}{\xi + i\eta}f_x\of{z} = \frac{\xi}{\xi+i\eta} f_x\of{z} + \frac{\eta}{\xi+i\eta} if_x\of{z}
                \intertext{Nach der Cauchy-Riemann'schen-DG gilt jetzt}
                &= \frac{\xi}{\xi+i\eta} f_x\of{z} + \frac{\eta}{\xi+i\eta} f_y\of{z}\\
                \impl \frac{f\of{z+h} - f\of{z}}{h} - f_x\of{z} &= \frac{\eta}{\xi+i\eta}\underbrace{\pair{u_x\of{z_1} + i v_y\of{z_2} - f_y\of{z}}}_{\to 0} + \frac{\xi}{\xi - i\eta}\underbrace{\pair{u_x\of{z_3} + iv_x\of{z_4} - f_x\of{z}}}_{\to 0}
                \intertext{Wobei die beiden Konvergenzen gegen 0 die Stetigkeit der partiellen Ableitungen benötigt. Zusätzlich gilt}
                \abs{\frac{\xi}{\xi + i\eta}} \leq \abs{\frac{\xi}{\xi}} = 1 \quad&\text{und}\quad \abs{\frac{\eta}{\xi+i\eta}} \leq \abs{\frac{\eta}{i\eta}} = 1
                \intertext{Das heißt die Vorfaktoren sind begrenzt und damit folgt}
                \frac{f\of{z+h} - f\of{z}}{h} - f_x\of{z} &\to 0
            \end{align*}
            Womit wir die Behauptung gezeigt haben.
        \end{proof}
    \end{satz}

    \begin{beispiel}
        Es sei $f\of{z} = x^2 + y^2 = z\conj{z}$. Dann gilt $f_x = 2x$ sowie $f_y = 2y$. Das heißt $f_y = if_x$ gilt nur für $z = 0$. Ist $f$ dann differenzierbar?
    \end{beispiel}

    \begin{definition}
        Wir sagen $f$ ist analytisch in $z$, falls $f$ (komplex) differenzierbar ist in einer Umgebung von $z$. $f$ ist außerdem analytisch auf $S\subseteq\C$, falls $f$ (komplex) differenzierbar ist in einer offenen Umgebung von $S$.
    \end{definition}

    \begin{satz} % Satz 7
        Sei $f = u + iv$ analytisch in einer offenen zusammenhängenden Menge $D$ ($D$ Umgebung). Ist $u$ konstant auf $D$, so ist $f$ konstant.

        \begin{proof}
            $u$ ist konstant auf $D$. Das heißt $u_x = u_y = 0$ auf $D$. Nach Satz~\ref{satz:cauchy-riemann} ist auch $v_x = v_y = 0$ auf $D$. Da $D$ zusammenhängend ist, ist also auch $v$ konstant und damit ist $f = u + iv$ konstant.
        \end{proof}
    \end{satz}

    \begin{satz} % Satz 8
        \label{satz:temp-8}
        Sei $f = u+iv$ analytisch auf einer Umgebung $D$. Ist $\abs{f}$ konstant auf $D$, so ist $f$ konstant.

        \begin{proof}
            Sei \OBDA $\abs{f} > 0$. Dann gilt $\abs{f}^2 = u^2 + v^2 = C > 0$
            \begin{align*}
                0 &= \partial_x\pair{u^2+v^2} = 2uv_x + 2vu_x\\
                0 &= \partial_y\pair{u^2+v^2} = 2uu_x + 2vv_y
                \intertext{Nach Cauchy-Riemann folgt}
                \impl &\begin{cases}
                           0 = u u_x - v u_y\\
                           0 = u u_y + v u_x
                \end{cases}\\
                \impl &\begin{cases}
                           0 = u^2 u_x - uvu_y\\
                           0 = uvu_y + v^2 u_x
                \end{cases}\\
                \impl 0 &= \pair{u^2+v^2}u_x = C\cdot u_x\\
                \impl u_x &= 0\\
                \impl v_y &= u_x = 0
            \end{align*}
            Analog zeigt man $u_y = -v_x = 0$. Damit sind $u,v$ und somit $f$ konstant.
        \end{proof}
    \end{satz}

    \subsection{Die Funktionen $e^z, \cos z, \sin z$}

    Wir hatten bereits
    \begin{align*}
        e^z &= \sum_{n=0}^{\infty} \frac{z^n}{n!}\\
        \frac{\dif}{\dif z}e^z &= e^z\\
        e^{i\varphi} &= \sum_{n \text{ gerade}}^{} \frac{(-1)^n\varphi^n}{n!} + \sum_{n\text{ ungerade}}^{} \frac{\pair{-1}^{n}\varphi^n}{n!}\\
        &= \cos\varphi + \sin\varphi\\
        \conj{e^{z}} &= e^{\conj{z}}\\
        \cos \varphi &= \frac{1}{2}\pair{e^{i\varphi} + e^{-i\varphi}}\\
        \sin\varphi &= \frac{1}{2i}\pair{e^{i\varphi} + e^{-i\varphi}}
        \intertext{Definiere}
        \cos\varphi &\coloneqq \sum_{k=0}^{\infty} \pair{-1}^{k} \frac{z^{2k}}{(2k)!} = \frac{1}{2}\pair{e^{iz} + e^{-iz}}\\
        \sin\varphi &\coloneqq \sum_{k=0}^{\infty} \pair{-1}^{k} \frac{z^{2k+1}}{(2k+1)!} = \frac{1}{2i}\pair{e^{iz} - e^{-iz}}
    \end{align*}

    \newpage


    \section{[*] Linienintegrale}

    \subsection{Definition und Berechnung}
    \thispagestyle{sectionpage}

    \begin{definition}
        \marginnote{[05. Mai]}
        Sei $f: \interv{a,b} \to\C$ eine Funktion auf $I=\interv{a,b}$. Dann gilt
        \begin{align*}
            \int_{I}^{} f \dif t &= \int_{a}^{b} f\of{t} \dif t \coloneqq \int_{a}^{b} Re\of{f\of{t}} \dif t + i \int_{a}^{b} \Im\of{f\of{t}} \dif t
        \end{align*}
    \end{definition}

    \begin{definition}[Glatte Kurven]
        \theoremescape
        \begin{enumerate}[label=(\roman*)]
            \item Sei $z\of{t} \coloneqq x\of{t} + i y\of{t}$ ($a \leq t \leq b$). Die Kurve $C: z\of{t},~a\leq t \leq b$ ist bestimmt durch $z\of{t}$ und heißt stückweise differenzierbar und wir setzen
            \begin{align*}
                \dot z\of{t} &= \frac{\dif}{\dif t}z\of{t} = \dot x\of{t} + i \dot y \of{t} = \frac{\dif x\of{t}}{\dif t} + i \frac{\dif y\of{t}}{\dif t}
                \intertext{falls die Funktionen $x, y, : \interv{a,b}\to \R$ stetig auf $\interv{a,b}$ sind und es eine Partition}
                \interv{a,b} &= \interv{t_0, t_1 } \cup \ldots \cup \interv{t_{n-1}, t_n}\tag{$t_i \leq t_{i+1}, t_0 = a, t_n = b$}
            \end{align*}
            gibt, sodass $x\of{t}, y\of{t}$ stetig differenzierbar auf $\interv{t_{j-1}, t_j}$ sind
            \item Die Kurve heißt glatt, falls $\dot z\of{t} \neq 0$ bis auf endlich viele $t\in\interv{a,b}$
        \end{enumerate}
    \end{definition}

    \begin{definition}[Kurvenintegral]
        Sei $C: z\of{t},~a\leq t \leq b$ eine (glatte) Kurve. Das Linienintegral von $f$ (definiert in einer Umgebung von $C$ oder nur auf $C$) ist definiert durch
        \begin{align*}
            \int_{C}^{} f \dif z &= \int_{C}^{} f\of{z} \dif z \coloneqq \int_{a}^{b} f\of{z\of{t}}\dot z\of{t} \dif t
        \end{align*}
    \end{definition}

    \begin{definition}[Zwei Kurven]
        Zwei Kurven $C_1 \coloneqq z\of{t},~a\leq t \leq b$, $C_2: w\of{t},~c\leq t\leq d$ sind (glatt) äquivalent, falls es eine bijektive $\mC^1$-Abbildung $\lambda: \interv{c,d}\to\interv{a,b}$ gibt mit $\lambda\of{c} = a$, $\lambda\of{d} = b$, $\lambda'\of{t} \geq 0$ und $w\of{t} = z\of{\lambda\of{t}}$ (für $c \leq t \leq d$).
    \end{definition}

    \begin{satz}
        Sind die Kurven $C_1, C_2$ äquivalent, so folgt
        \begin{align*}
            \int_{C_1}^{} f \dif z &= \int_{C_2}^{} f \dif z
        \end{align*}

        \begin{proof}
            Es gilt die Kettenregel
            \begin{align*}
                \frac{\dif}{\dif t} F\of{z\of{t}} &= F'\of{z\of{t}} \dot z\of{t}
                \intertext{Nach den Substitutionsregeln folgt also}
                \int_{c}^{d} h\of{z\of{\lambda\of{t}}} \dot z\of{\lambda\of{t}}\dot\lambda\of{t} \dif t &= \int_{a}^{b} h\of{z\of{s}}\dot z\of{s}\dif s\qedhere
            \end{align*}
        \end{proof}
    \end{satz}

    \begin{satz}
        Es gilt
        \begin{align*}
            \int_{C}^{} f \dif t &= - \int_{-C}^{} f \dif t
        \end{align*}
        wobei $-C: z\of{a+b-t},~a\leq t \leq b$.

        \begin{proof}
            Wir setzen $w\of{t} = z\of{a+b-t}$. Dann gilt $\dot w\of{t} = - \dot z\of{a+b-t}$. Dann ist
            \begin{align*}
                \int_{-C}^{} f\of{w} \dif w &= \int_{a}^{b} f\of{w\of{t}}\dot w\of{t} \dif t = - \int_{a}^{b} f\of{z\of{a+b-t}}\dot z\of{a+b-t} \dif t\\
                &= \int_{b}^{a} f\of{z\of{s}}\dot z\of{s} \dif s = - \int_{a}^{b} f\of{z\of{s}}\dot z\of{s} \dif s = -\int_{C}^{} f\of{z} \dif z\qedhere
            \end{align*}
        \end{proof}
    \end{satz}

    \begin{beispiel}
        Sei $f\of{z} = x^2 + i y^2$ mit $z = x + iy$ sowie $C= z\of{t} = \pair{1+i}t$ für $0\leq t \leq 1$.
        \begin{align*}
            \int_{C}^{} f\of{z} \dif z &= \int_{0}^{1} f\of{\pair{1+i}t}\pair{1+i} \dif t\\
            &= \pair{1+i}^2 \int_{0}^{1} t^2 \dif t = \frac{2i}{3}
        \end{align*}
    \end{beispiel}

    \begin{beispiel}
        Sei $f\of{z} = \frac{1}{z} = \frac{1}{x+iy} = \frac{x}{x + y^2} - i \frac{y}{x^2+y^2}$ mit $C: z\of{t} = R\pair{\cos t + i \sin t}$
        \begin{align*}
            \int_{C}^{} \frac{1}{z} \dif z &= \int_{0}^{2\pi} \pair{\frac{R\cos t}{R^2} - \frac{R \sin t}{R^2}} \dif t\\
            &= i \int_{0}^{2\pi} \pair{\cos^2 t + \sin^2 t} \dif 2\pi t
            \intertext{Alternativ}
            z\of{t} &= R e^{it}\\
            \dot z\of{t} &= i R e^{it}\\
            \int_{C}^{} \frac{\dif z}{z} \dif t &= \int_{0}^{2\pi} \frac{1}{R e^{it}}\cdot R e^{it} \dif t\\
            &= i \int_{0}^{2\pi}  \dif t = 2\pi i
        \end{align*}
    \end{beispiel}

    \begin{beispiel}
        Sei $f = 1$ und $C$ eine beliebige Kurve
        \begin{align*}
            \int_{C}^{}  \dif z &= \int_{a}^{b} \dot z\of{t} \dif t = z\of{b} - z\of{a}
        \end{align*}
    \end{beispiel}

    \begin{satz}[$C$-glatte Kurve]
        Seinen $f, g$ stetige Funktionen auf $\C$. Sei $\alpha\in\C$. Dann gilt
        \begin{align*}
            \int_{C}^{} f+g \dif z &= \int_{C}^{} f \dif z + \int_{C}^{} g \dif z\\
            \int_{C}^{} \alpha f \dif z &= \alpha \int_{C}^{} f \dif z
        \end{align*}
        \begin{proof}
        (Selber machen)
            .
        \end{proof}
    \end{satz}

    \begin{definition}
        Seien $\alpha, \beta\in\C$. Wir schreiben $\alpha \ll \beta$, falls $\abs{\alpha} \leq \abs{\beta}$.
    \end{definition}

    \begin{lemma}
        Sei $G: \interv{a,b}\to\C$ stetig. Dann folgt
        \begin{align*}
            \int_{a}^{b} G\of{t} \dif t &\ll \int_{a}^{b} \abs{G\of{t}} \dif t\\
            \impl \int_{a}^{b} G\of{t} \dif t &\leq \int_{a}^{b} \abs{G\of{t}} \dif t
            \intertext{Beweise $G\of{t} = \Re\of{G\of{t}} - i \sin\of{G\of{t}}$. Trick:}
            \int_{a}^{b} G\of{t} \dif t &= R e^{i\varphi}\tag{$R >0, \varphi\in\R$}\\
            \int_{1}^{b} G\of{t} \dif t &= R = \conj{e} \int_{a}^{b} G\of{t} \dif t = \int_{a}^{b} e^{-i\varphi}G\of{t} \dif t\\
            \Re\of{ \int_{a}^{b} e^{-i\varphi} G\of{t}\dif t} = \int_{a}^{b} \Re\of{e^{-i\varphi} G\of{t}} \dif t\\
            &\leq \int_{a}^{b} \abs{G\of{t}} \dif t\qedhere
        \end{align*}
    \end{lemma}

    \begin{satz}
        Sei $C$ eine Kurve der Länge $L$ auf $f$, stetig auf $\C$ und $\abs{f\of{z}} \leq M~\forall z\in\C$. Dann folgt
        \begin{align*}
            \int_{C}^{} f\of{z} \dif z &\ll M \cdot L
        \end{align*}
        \begin{proof}
        (Fehlt)
        \end{proof}
    \end{satz}

    \begin{korollar}
        Sei $(f_n)_n$ eine Folge stetiger Funktionen und $f_n \to f$ . Dann folgt
        \begin{align*}
            \int_{C}^{} f \dif z &= \lim_{h\toinf} \int_{C}^{} f_n \dif z
        \end{align*}

        \begin{proof}
            \begin{align*}
                \abs{ \int_{C}^{} f\of{z} \dif z - \int_{C}^{} f_n \dif z} &= \abs{ \int_{C}^{} \pair{f\of{z} - f_n\of{z}} \dif z}\\
                &\leq \int_{a}^{b} \dif f\of{z}of{h} - f_n\of{z\of{t}} \dif t = \underbrace{\sup_{z\in\C} \abs{f\of{z} - f_n\of{z}}}_{\to 0} \cdot L_z\qedhere
            \end{align*}
        \end{proof}
    \end{korollar}

    \begin{lemma}
        \label{lemma:stammfunktion-kurvint}
        Angenommen $f = F'$, $F$ analytisch auch eine Kurve $C$ mit Analysepunkt $z\of{a}$, Endpunkt $z\of{b}$. Dann folgt
        \begin{align*}
            \int_{C}^{} f \dif z &= F\of{z\of{b}} - F\of{z\of{a}}
        \end{align*}

        \begin{proof}
            \begin{align*}
                F\of{z\of{t}} &= F'\of{z\of{t}}\dot z\of{t} = f\of{z\of{t}}z\of{t}\\
                \int_{C}^{} f \dif z &= \int_{a}^{b} f\of{z\of{t}}\dot z\of{t} \dif t\\
                &= \int_{a}^{b} \gamma\of{t}  \dif t = \gamma\of{b} - \gamma\of{a} = F\of{z\of{b}} - F\of{z\of{a}}\qedhere
            \end{align*}
        \end{proof}
    \end{lemma}

    \subsection{Integrale über geschlossenen Kurven}

    \begin{definition}
        \marginnote{[06. Mai]}
        Sei $U\subseteq\C$ offen und $C: z\of{t},~a\leq t\leq b$ eine Kurve in $U$. Dann nennen wir $C$ geschlossen, falls $z\of{a} = z\of{b}$. Eine geschlossene Linie heißt einfach, falls $z\of{t_1} = z\of{t_2}$ für $t_1 < t_2 \in\interv{a,b}$ schon impliziert, dass $t_1 = a, t_2 = b$.
    \end{definition}

    \begin{bemerkung}
        Ein (Standard-)Rechteck $R$ in $U$ ist ein abgeschlossenes, achsen-paralleles Rechteck. Rand $\Gamma = \partial R$ wird entgegen dem Uhrzeigersinn durchlaufen.
    \end{bemerkung}

    Das erste Ziel dieses Kapitels soll es sein, den folgenden Satz zu beweisen:

    \begin{satz}[Rechteckssatz I] % Satz 16
        \label{satz:rechtecksatz-1}
        Sei $U\subseteq\C$ offen, $f: U\to\C$ analytisch und $R\subseteq U$ ein Rechteck mit Rand $\Gamma$. Dann gilt
        \begin{align*}
            \int_{\Gamma}^{} f\of{z} \dif z &= 0
        \end{align*}
    \end{satz}

    \begin{lemma} % Lemma 17
        Sei $f\of{z} = \alpha + \beta z$ affin linear. Dann gilt Satz~\ref{satz:rechtecksatz-1}

        \begin{proof}
            Es ist $F\of{z} = \alpha z + \frac{\beta}{2}z^2$ eine Stammfunktion von $f$ mit $z\of{t}$ Parameterlösung von $\Gamma$. Dann gilt
            \begin{align*}
                \int_{\Gamma}^{} f\of{z} \dif z \annot[{&}]{=}{Lemma~\ref{lemma:stammfunktion-kurvint}} F\of{z\of{b}} - F\of{z\of{a}} = 0
            \end{align*}
            wobei wir verwendet haben, dass $z\of{b} = z\of{a}$.
        \end{proof}
    \end{lemma}

    \begin{proof}[Beweis von Satz~\ref{satz:rechtecksatz-1}]
    (Erfolt durch Überdeckung und schlaue Abschätzung, fehlt hier)
    \end{proof}

    \begin{satz} % Satz 18
        \label{satz:int-lokal}
        Sei $U\subseteq\C$ offen, $f: U \to \C$ analytisch, $z_0\in U$ und $R > 0$, sodass $D_R\of{z_0} = \set{\abs{z-z_0} \leq R}\subseteq U$. Dann existiert eine analytische Funktion $F: D_R\of{z_0} \to \C$ mit $F'\of{z} = f\of{z}~\forall z\in D_R\of{z_0}$. D.h. $f$ hat lokal in der Nähe von $z_0$ eine Stammfunktion.

        \begin{proof}
            O.B.d.A. sei $U=D_R\of{z_0}$. Zwischen $w, z\in D_R\of{z_0}$ gibt es einen Weg
            \begin{align*}
                \gamma_{w,z}\of{t} &= \begin{cases}
                                          w + \Re\of{z-w}t &0\leq t \leq 1\\
                                          w + \Re\of{z-w} + i \Im\of{z-w}\pair{t-1} & 1\leq t \leq 2
                \end{cases}
                \intertext{Wir setzen dann}
                F\of{z} &= \int_{\gamma_{z_0, z}}^{} f\of{w} \dif w\\
                \impl F\of{z+h} - F\of{z} &= \int_{\gamma_{z_0, z+h}}^{} f\of{w} \dif w - \int_{\gamma_{z_0, z}}^{} f\of{w} \dif w\\
                \intertext{Durch geometrische Überlegungen sehen wir, dass sich die Pfade auch folgendermaßen darstellen lassen}
                &= \int_{\gamma_{z,z+h}}^{} f\of{w} \dif w + \int_{\Gamma_{z, z+h}}^{} f\of{w} \dif w \annot{=}{Satz~\ref{satz:rechtecksatz-1}} \int_{\gamma_{z,z+h}}^{} f\of{w} \dif w\\
                \impl \frac{F\of{z+h} - F\of{z}}{h} &= \frac{1}{h} \int_{\gamma_{z, z+h}}^{} f\of{w} \dif w
                \intertext{Es gilt außerdem}
                \int_{\gamma_{z, z+h}}^{} 1 \dif w &= z + h - w = h
                \intertext{Das heißt für festes $z$}
                \frac{1}{h} \int_{}^{} f\of{z} \dif w &= f\of{z}\\
                \impl \frac{F\of{z+h} - F\of{z}}{h} - f\of{z} &= \frac{1}{h}\int_{\gamma_{z, z+h}}^{} \pair{f\of{w} - f\of{z}} \dif w\\
                \impl \abs{\frac{F\of{z+h} - F\of{z}}{h} - f\of{z}} &\leq \frac{1}{\abs{h}} \abs{ \int_{\gamma_{z, z+h}}^{} f\of{w} - f\of{z} \dif w}\\
                &\leq \frac{1}{\abs{h}} \sup_{w\in\gamma_{z, z+h}} \abs{f\of{w} - f\of{z}}\cdot \pair{\abs{\Re\of{h}} + \abs{\Im\of{h}}}\\
                &\leq \frac{1}{\abs{h}} \sup_{w\in\gamma_{z, z+h}} \abs{f\of{w} - f\of{z}}\cdot 2\abs{h}\\
                &= 2 \sup_{w\in\gamma_{z, z+h}} \abs{f\of{w} - f\of{z}} \to 0\text{ für }h\to 0\qedhere
            \end{align*}
        \end{proof}
    \end{satz}

    \begin{satz} % Satz 19
        Ist $f: D_{R}\of{z_0}\to\C$ analytisch und $C$ eine glatte, geschlossene Kurve in $D_R\of{z_0}$, so ist
        \begin{align*}
            \int_{C}^{} f\of{z} \dif z &= 0
        \end{align*}

        \begin{proof}
            Nach Satz~\ref{satz:int-lokal} hat $f$ eine Stammfunktion $F$ auf $D_R\of{z_0}$ und $F$ ist analytisch. $C: z\of{t},~a\leq t \leq b$ und wegen geschlossen gilt $z\of{a} = z\of{b}$. Damit folgt
            \begin{align*}
                \int_{C}^{} f\of{z} \dif z &= F\of{z\of{b}} - F\of{z\of{a}} = 0\qedhere
            \end{align*}
        \end{proof}
    \end{satz}

    Vorsicht, dieser Satz gilt nur lokal:

    \begin{beispiel}
        Sei $f\of{z} = \frac{1}{z}$ und $C: Re^{it},~0\leq t \leq 2\pi$. Dann gilt
        \begin{align*}
            \int_{C}^{} \frac{1}{z} \dif z &= \int_{0}^{2\pi} \frac{iRe^{it}}{Re^{it}} \dif t = 2\pi i\\
            \intertext{Andererseits ist}
            \int_{C}^{} z^k \dif z &= 0\tag{$k\in\Z\setminus\set{-1}$}
        \end{align*}
    \end{beispiel}

    \newpage


    \section{[*] Lokale Eigenschaften analytischer Funktionen}
    \subseccount
    \thispagestyle{sectionpage}

    Sei $f: U\to\C$ analytisch und $\alpha\in\C$. Wir definieren
    \begin{align*}
        g\of{z} &\coloneqq \begin{cases}
                               \frac{f\of{z} - f\of{\alpha}}{z-\alpha} &z\neq \alpha\\
                               f'\of{\alpha} &z = \alpha
        \end{cases}
    \end{align*}
    ist stetig auf $U$. Wir wollen jetzt den Rechtecksatz auf $g$ erweitern.

    \begin{satz}[Rechtecksatz II] % Satz 1
        Ist $R\subseteq U$ ein Rechteck. dann gilt
        \begin{align*}
            \int_{\partial R}^{} g\of{z} \dif z &= 0
        \end{align*}

        \begin{proof}
            \textsc{Fall 1}: $\alpha\not\in R$. Dann betrachten wir eine Umgebung von $R$, in der $\alpha$ nicht enthalten ist und verwenden Satz~\ref{satz:rechtecksatz-1}, da $g$ analytisch auf $R$ ist.\\[.5\baselineskip]
            \textsc{Fall 2}: $\alpha\in\partial R$. Dann teilen wir $R$ in 6 Subrechtecke $R_{k\in\set{1,\ldots, 6}}$ und es sei $\alpha\in\partial R_1$. Dann gilt
            \begin{align*}
                \impl \int_{\Gamma}^{} g\of{z} \dif z &= \sum_{k=1}^{6} \int_{\partial R_k}^{} f\of{z} \dif z
                \intertext{Nach \textsc{Fall 1} haben wir dann}
                &= \int_{\partial R_1}^{} g\of{z} \dif z\\
                \abs{ \int_{\partial R_1}^{} g\of{z} \dif z} &\leq \underbrace{\sup_{z\in\partial R_1} \abs{g\of{z}}}_{<\infty}\cdot\diam\of{R_1}
            \end{align*}
            Wir können $R_1$ aber beliebig klein machen. Damit geht der Ausdruck gegen 0.\\[.5\baselineskip]
            \textsc{Fall 3}: $\alpha\in R\setminus\partial R$. Wir gehen wie in \textsc{Fall 2} vor und teilen $R$ in 6 Subrechtecke auf. Damit gilt
            \begin{align*}
                \int_{\partial R}^{} g\of{z} \dif z &= \int_{\partial R_1}^{} g\of{z} \dif z \to 0 \text{ für }\diam\of{R_1}\to 0\qedhere
            \end{align*}
        \end{proof}
    \end{satz}

    \newpage


    \section{[*] Lokale Eigenschaften analytischer Funktionen}

    \subsection{Vorbereitung}

    \thispagestyle{sectionpage}

    \marginnote{[12. Mai]}
    \textbf{Situation:} Sei $U\subseteq\C$ offen und $f: U \to \C$ analytisch. Es sei $\alpha\in\C$, dann ist
    \begin{align*}
        g\of{z} &\coloneqq \begin{cases}
                               \frac{f\of{z} - f\of{\alpha}}{z-\alpha} &z\neq \alpha\\
                               f'\of{\alpha} &z=\alpha
        \end{cases}
    \end{align*}
    stetig auf $U$ und analytisch auf $U\setminus\set{\alpha}$.

    \begin{korollar}
        \label{kor:stammfunktion}
        In der obigen Situation hat $g$ wieder lokal eine Stammfunktion. Genau gilt: Für jeden Punkt $z_0\in U$ gibt es ein $F: D_R\of{z_0} \to \C$ analytisch mit $F' = g$ auf $D_R\of{z_0}$ für ein $R > 0$. Ist insbesondere $C$ eine geschlossene Kurve in $D_R\of{z_0}$, so ist
        \begin{align*}
            \int_{C}^{} g\of{z} \dif z &= 0
        \end{align*}
    \end{korollar}

    \begin{satz}[Cauchy-Integralformel I]
        \label{satz:cauchy-int}
        Sei $U\subseteq\C$ offen, $\alpha\in U$, $f: U\to\C$ analytisch und $R > 0$, sodass $D_R\of{\alpha}\subseteq U$. Ferner sei $0 < \rho < R$ und $C\rho\of{\alpha}$ die Kurve $z\of{t} \coloneqq \rho e^{it}+\alpha$. Dann folgt
        \begin{align*}
            f\of{\alpha} &= \frac{1}{2\pi i} \int_{C\rho\of{\alpha}}^{} \frac{f\of{w}}{w - \alpha} \dif w
        \end{align*}

        \begin{proof}
            Wir betrachten
            \begin{align*}
                g\of{z} &\coloneqq \begin{cases}
                                       \frac{f\of{z} - f\of{\alpha}}{z-\alpha} &z\neq \alpha\\
                                       f'\of{\alpha} &z=\alpha
                \end{cases}
                \intertext{Dann gilt nach Korollar~\ref{kor:stammfunktion}, dass}
                0 &= \int_{C\rho\of{\alpha}}^{} g\of{w} \dif w = \int_{C\rho\of{\alpha}}^{} \frac{f\of{w} - f\of{\alpha}}{w-\alpha} \dif w\\
                &= \int_{C\rho\of{\alpha}}^{} \frac{f\of{w}}{w-\alpha} \dif \alpha - \int_{C\rho\of{\alpha}}^{} \frac{f\of{\alpha}}{w-\alpha} \dif w\\
                \impl \int_{C\rho\of{\alpha}}^{} \frac{f\of{w}}{w-\alpha} \dif w &= f\of{\alpha} \int_{C\rho\of{\alpha}}^{} \frac{1}{w-\alpha} \dif x = f\of{\alpha}\int_{0}^{2\pi} \frac{i\rho e^{it}}{\rho e^{it}} \dif t = f\of{\alpha} 2\pi i\\
                \impl \int_{C\rho\of{\alpha}}^{} \frac{f\of{w}}{w-\alpha} \dif w &= 2\pi i f\of{\alpha}\qedhere
            \end{align*}
        \end{proof}
    \end{satz}

    \begin{bemerkung}
        In Satz~\ref{satz:cauchy-int} gilt sogar $\forall z\in D_{\rho}\of{\alpha}$ ist
        \begin{align*}
            f\of{z} &= \frac{1}{2\pi i}\int_{C\rho\of{\alpha}}^{} \frac{f\of{w}}{w-z} \dif w
        \end{align*}
    \end{bemerkung}

    \begin{lemma}
        Sei $C\rho\of{\alpha}: z\of{t} = \rho e^{it} + \alpha \subseteq D_R\of{\alpha}$ ($R > \rho > 0$). Dann gilt für alle $z\in D_{\rho}\of{\alpha}$
        \begin{align*}
            \int_{C\rho\of{\alpha}}^{} \frac{1}{w-z} \dif w = 2\pi i
        \end{align*}

        \begin{proof}
            \begin{align*}
                \int_{C \rho\of{\alpha}}^{} \frac{1}{w-z} \dif w &= \int_{C\rho\of{\alpha}}^{} \frac{1}{w-\alpha - \pair{z-\alpha}} \dif w\\
                &= \int_{C\rho\of{\alpha}}^{} \frac{1}{\pair{w-\alpha}\pair{1-\frac{z-\alpha}{w-\alpha}}} \dif w
                \intertext{Es ist $\abs{z-\alpha} < \rho$ und $\abs{w-\alpha} < \rho$. Also ist}
                \delta &= \abs{\frac{z-\alpha}{w-\alpha}} = \frac{\abs{z-\alpha}}{\rho} < 1\quad\forall w\in C\rho\of{\alpha}
                \intertext{Wir schreiben um über die geometrische Reihe}
                \frac{1}{1-\frac{z-\alpha}{w-\alpha}} &= \sum_{n=0}^{\infty} \pair{\frac{z-\alpha}{w-\alpha}}^n
                \intertext{Wir haben gleichmäßige und absolute Konvergenz}
                \impl \int_{C\rho\of{\alpha}}^{} \frac{1}{\pair{w-\alpha}\pair{1-\frac{z-\alpha}{w-\alpha}}} \dif w &= \int_{C\rho\of{\alpha}}^{} \frac{1}{w-\alpha} \sum_{n=0}^{\infty} \pair{\frac{z-\alpha}{w-\alpha}}^n \dif w\\
                &= \sum_{n=0}^{\infty} \pair{z-\alpha}^n \int_{C\rho\of{\alpha}}^{} \frac{1}{\pair{w-\alpha}^{n+1}} \dif w\\
                \int_{C\rho\of{\alpha}}^{} \frac{1}{\pair{w-\alpha}^{n+1}} \dif w &= \int_{0}^{2\pi} \frac{i\rho e^{it}}{\pair{\rho e^{it}}^{n+1}} \dif t\\
                &= \frac{1}{\rho^n} \int_{0}^{2\pi} e^{-int}\dif t = \begin{cases}
                                                                         2\pi i &n = 0\\
                                                                         0 &n\geq 1
                \end{cases}\\
                \impl \sum_{n=0}^{\infty} \pair{z-\alpha}^n \int_{C\rho\of{\alpha}}^{} \frac{1}{\pair{w-\alpha}^{n+1}} \dif w &= 2\pi i\\
                \impl \int_{C \rho\of{\alpha}}^{} \frac{1}{w-z} \dif w &= 2\pi i\qedhere
            \end{align*}
        \end{proof}
    \end{lemma}

    \begin{korollar}[Cauchy-Integralformel II] % Korollar 5
        Sei $U\subseteq\C$ offen und $f: U\to\C$ analytisch sowie $\alpha\in U$, $R > 0$ mit $D_{R}\of{\alpha} \subseteq U$. Dann folgt $\forall 0 < \rho < R$ und $z \in D_{\rho}\of{\alpha}$
        \begin{align*}
            f\of{z} &= \frac{1}{2\pi i} \int_{C\rho\of{\alpha}}^{} \frac{f\of{w}}{w-z} \dif w
        \end{align*}

        \begin{proof}
            Nach Korollar~\ref{kor:stammfunktion} gilt
            \begin{align*}
                0 &= \int_{C\rho\of{\alpha}}^{} \frac{f\of{w} - f\of{z}}{w-\alpha} \dif w = \int_{C\rho\of{\alpha}}^{} \frac{f\of{w}}{w-z} \dif w - f\of{z}\underbrace{\int_{C\rho\of{\alpha}}^{} \frac{1}{w-z} \dif w}_{= 2\pi i}\qedhere
            \end{align*}
        \end{proof}
    \end{korollar}

    \subsection{Lokale Potenzreihenentwicklung}

    \begin{satz}
        \label{satz:potenzreihenentwicklung}
        Sei $U\subseteq\C$ offen, $f: U \to \C$ analytisch, $\alpha\in U$, $R > 0$, sodass $D_R\of{\alpha}\subseteq U$. Dann hat $f$ eine in $D_R\of{\alpha}$ konvergente Potenzreihe. Das heißt es existiert eine Folge $(a_n)_n \subseteq\C$, sodass
        \begin{align*}
            f\of{z} &= \sum_{n=0}^{\infty} a_n\pair{z-\alpha}^n\quad\forall z\in D_R\of{\alpha}
        \end{align*}
        Dabei ist $D_R\of{\alpha}$ \anf{$=$} $\C$ erlaubt.

        \begin{proof}
            Für $z\in D_{\rho}\of{\alpha}$  haben wir
            \begin{align*}
                f\of{z} &= \frac{1}{2\pi i}\int_{C\rho\of{\alpha}}^{} \frac{f\of{w}}{w-z} \dif w = \frac{1}{2\pi i} \int_{C\rho\of{\alpha}}^{} \frac{f\of{w}}{w-\alpha-\pair{w-\alpha}} \dif w\\
                &= \frac{1}{2\pi i} \int_{C \rho\of{\alpha}}^{} \frac{f\of{w}}{\pair{w-\alpha}\pair{1- \frac{z-\alpha}{w-\alpha}}} \dif w
                \intertext{Wir können wieder die Umschreibung zur geometrischen Reihe verwenden. Also gilt}
                &= \frac{1}{2\pi i} \sum_{n=0}^{\infty} \pair{z-\alpha}^n \underbrace{\int_{C\rho\of{\alpha}}^{} \frac{f\of{w}}{\pair{w-\alpha}^{n+1}} \dif w}_{\eqqcolon a_n\of{\rho}}\\
                &= \frac{1}{2\pi i} \sum_{n=0}^{\infty} a_n\of{\rho}\pair{z-\alpha}^n\qedhere
            \end{align*}
        \end{proof}
    \end{satz}

    \begin{bemerkung}
        In Satz~\ref{satz:potenzreihenentwicklung} ist $(a_n)_n$ unabhängig von $ R > 0$, solange $D_R\of{\alpha}\subseteq U$.
    \end{bemerkung}

    \begin{beobachtung}
        Ist $z\in D_R\of{\alpha}$, so existiert ein $\rho$ mit $\abs{z-\alpha} < \rho < R$. Für dieses $\rho$ haben wir
        \begin{align*}
            f\of{z} &= \sum_{n=0}^{\infty} \frac{a_n\of{\rho}}{2\pi i}\pair{z-\alpha}^n
        \end{align*}
    \end{beobachtung}

    \begin{beobachtung}
        $a_n\of{\rho}$ ist unabhängig von $0 < \rho < R$.

        \begin{proof}
            Sei $0 < \rho_1 < \rho_2 < R$ und $\abs{z-\alpha} < \rho_1$. Dann folgt
            \begin{align*}
                \impl f\of{z} &= \sum_{n=0}^{\infty} a_n\of{\rho_1}\pair{z-\alpha}^n = \sum_{n=0}^{\infty} a_n\of{\rho_2}\pair{z-\alpha}^n\quad\forall z\in D_{\rho_1}\of{\alpha}
            \end{align*}
            Nach dem Eindeutigkeitssatz für Potenzreihen haben wir $a_n\of{\rho_1} = a_n\of{\rho_2}$.
        \end{proof}
    \end{beobachtung}

    \begin{korollar} % Korollar 7
        Sei $U\subseteq\C$ offen und $f: U \to\C$ analytisch. Dann ist $f$ unendlich oft differenzierbar.

        \begin{proof}
            Lokal ist $f\of{z}$ durch eine konvergente Potenzreihe gegeben. Also ist $f$ unendlich oft komplex differenzierbar.
        \end{proof}
    \end{korollar}

    \begin{korollar}
        \marginnote{[13. Mai]}
        Sei $U\subseteq\C$ offen und $f: U \to\C$ analytisch. Dann ist
        \begin{align*}
            f^{(n)}\of{\alpha} &= \frac{n!}{2\pi i} \int_{C\rho\of{\alpha}}^{} \frac{f\of{w}}{\pair{w-\alpha}^{n+1}} \dif w
        \end{align*}
    \end{korollar}

    \begin{satz}
        \label{satz:temp-9}
        Sei $f: U \to \C$ analytisch und $\alpha\in U$. Dann ist
        \begin{align*}
            g\of{z} &\coloneqq \begin{cases}
                                   \frac{f\of{z} - f\of{\alpha}}{z-\alpha} &z\neq \alpha\\
                                   f'\of{\alpha} &z = \alpha
            \end{cases}
        \end{align*}
        analytisch.

        \begin{proof}
            In $D_{R}\of{\alpha} \subseteq U$ gilt $f\of{z} \sum_{n=0}^{\infty} a_n \pair{z-\alpha}^n$ mit $a_0 = f\of{\alpha}$. Das heißt
            \begin{align*}
                g\of{z} &= \frac{f\of{z} - f\of{\alpha}}{z-\alpha} = \sum_{n=1}^{\infty} a_n \pair{z-\alpha}^{n-1}\qedhere
            \end{align*}
        \end{proof}
    \end{satz}

    \subsection{Liouville}

    \begin{notation}
        Eine analytische Funktion $f: \C \to \C$ heißt ganze Funktion.
    \end{notation}

    \begin{korollar}
        Habe $f: U \to \C$ in $U$ genau $N$ Nullstellen $a_1, \ldots, A_N \in U$. Dann definieren wir
        \begin{align*}
            g\of{z} &\coloneqq \frac{f\of{z}}{\pair{z-a_1}\pair{z-a_2}\cdots\pair{z-a_N}}\tag{$z\neq a_j$}
        \end{align*}
        Dann gilt $ \lim_{z\to a_k} g\of{z}$ existiert und $g$ ist analytisch.

        \begin{proof}
            Sei $f_ 0 \of{z} \coloneqq f\of{z}$ und
            \begin{align*}
                f_k\of{z} &\coloneqq \frac{f_{k-1}\of{z} - f_{k-1}\of{a_k}}{z-a_k} = \frac{f_{k-1}\of{z}}{z-a_k}\tag{$z\neq a_k$}
            \end{align*}
            Nach Induktion und Satz~\ref{satz:temp-9} gilt dann, dass $f_k$ analytisch ist für alle $k\in\set{0, \ldots, N}$. Beachte $g = f_N$.
        \end{proof}
    \end{korollar}

    \begin{satz} % Satz 11
        \label{satz:liouville}
        Eine beschränkte ganze Funktion ist konstant.

        \begin{proof}
            Für alle $R > 0$ ist $z\in D_R\of{0}$ und
            \begin{align*}
                f\of{z} &= \frac{1}{2\pi i} \int_{C_R\of{0}}^{} \frac{f\of{w}}{w-z} \dif w
                \intertext{Seien $z_1, z_2 \in D_R\of{\cdot}$}
                f\of{z_1} - f\of{z_2} &= \frac{1}{2\pi i} \int_{D_R\of{0}}^{} \pair{\frac{f\of{w}}{w-z_1} - \frac{f\of{w}}{w-z_2}} \dif w\\
                &= \frac{1}{2\pi i} \int_{D_R\of{0}}^{} f\of{w}\frac{z_1 - z_2}{\pair{w-z_1}\pair{w-z_2}} \dif w\\
                \intertext{Es ist $\abs{f\of{w}} < M < \infty$ für alle $w\in \C$. Nach M-L-Regel gilt}
                \abs{f\of{z_1} - f\of{z_2}} &\leq \frac{M}{2\pi}\frac{2\pi R}{\pair{R-\abs{z_1}}\pair{R-\abs{z_2}}} \to 0 \text{ für }R\toinf
            \end{align*}
            Das heißt $f$ ist konstant.
        \end{proof}
    \end{satz}

    \begin{satz} % Satz 12
        Sei $f$ eine ganze Funktion und für ein $k\in\N_0$ existieren $A, B \geq 0$ mit $\abs{f\of{z}}\leq A + B\abs{z}^k~\forall z\in\C$. Dann ist $f$ ein Polynom vom Grad $\leq k$.

        \begin{proof}
            Wir verwenden Induktion. $k = 0$ ist Satz~\ref{satz:liouville}.
            \begin{align*}
                g\of{z} &= \begin{cases}
                               \frac{f\of{z} - f\of{0}}{z} &z\neq 0\\
                               f'\of{0} &z = 0
                \end{cases}
                \intertext{Dann ist $g$ eine ganze Funktion}
                \abs{g\of{z}} &= \frac{\abs{f\of{z} - f\of{0}}}{\abs{z}} \tag{$\abs{z}\geq 1$}\\
                &\leq \frac{A + B\abs{z}^k + \abs{f\of{0}}}{\abs{z}} \leq \tilde{A} + B \abs{z}^{k-1}\\
                \abs{g\of{z}} &\leq C\tag{$\abs{z} \leq 1$}\\
                \impl \abs{g\of{z}} &\leq \max\pair{\tilde{A}, C} + B \abs{z}^{k-1}\quad\forall z\in\C
            \end{align*}
            Per Induktionsvoraussetzung ist $g$ ein Polynom vom Grad $\leq k-1$. Dann ist $f\of{z} = zg\of{z} + f\of{0}$ ein Polynom vom Grad $\leq k$.
        \end{proof}
    \end{satz}

    \begin{satz} % Satz 13
        \label{satz:fundamentalsatz}
        Jedes nicht-konstante Polynom mit komplexen Koeffizienten hat (mindestens) eine Nullstelle in $\C$.

        \begin{proof}
            \begin{align*}
                \abs{P\of{z}} &= \abs{a_0 + a_1 z + \ldots + a_{n-1}z^{n-1} + a_n z^n}\\
                &= \abs{z^n}\underbrace{\abs{a_n + a_{n-1}z^{-1} + \ldots + a_2 z^{2-n} + a_1 z^{1-n} + a_0 z^-n}}_{\geq \abs{a_n} - \abs{a_{n-1}}\abs{z}^{-1} - \ldots  - \abs{a_1}\abs{z}^{1-n} - \abs{a_0}\abs{z}^{-n}}\\
                &\geq \frac{\abs{a_n}}{2}\abs{z}^n
                \intertext{Das heißt $P\of{z} \toinf$ für $\abs{z}\toinf$. Wenn $P$ keine Nullstelle hat, dann ist}
                h\of{z} &= \frac{1}{P\of{z}}
            \end{align*}
            wohldefiniert und $h\of{z} \to 0$ für $\abs{z}\toinf$. Damit ist $h$ eine beschränkte analytische Funktion und nach Satz~\ref{satz:liouville} ist $h$ konstant. Das heißt $P$ war bereits konstant.
        \end{proof}
    \end{satz}

    \begin{bemerkung}
        Satz~\ref{satz:fundamentalsatz} erweitert sich wie folgt: Sei $g\of{z} \coloneqq \frac{P\of{z}}{z-a_1}$ für $P$ ein Polynom mit Nullstelle $a_1$. Dann ist
        \begin{align*}
            \abs{P\of{z}} &\leq A + B\abs{z}^n\\
            \impl \abs{g\of{z}} &\leq \tilde{A} + \tilde{B}\abs{z}^{n-1}
        \end{align*}
        Das heißt $g$ ist ein Polynom vom Grad $\leq n-1$ und nach Satz~\ref{satz:liouville} gilt $P\of{z} = \pair{z-a_1} g\of{z}$. Das heißt wir können ein Polynom bis auf einen konstanten Faktor vollständig in seine Nullstellen aufspalten.
    \end{bemerkung}

    \newpage


    \section{[*] Eindeutigkeit, Mittelwert \& Max, Modulus-Sätze}

    \subsection{Formulierung}
    \thispagestyle{sectionpage}

    \begin{satz}[Eindeutigkeit]
        \label{satz:eindeutigkeit}
        Sei $U\subseteq\C$ offen und zusammenhängend und $f: U\to\C$ analytisch. Sei $S\subseteq U$ mit Häufungspunkt in $U$ und $f\of{z} = 0~\forall z\in S$. Dann ist $f = 0$ in $U$.

        \begin{proof}
            Sei $\alpha\in U$ Häufungspunkt von $S$. Dann existiert ein $R > 0$, sodass $D_R\of{\alpha}\subseteq U$ und
            \begin{align*}
                f\of{z} &= \sum_{n=0}^{\infty} a_n \pair{z-\alpha}^n\quad\forall z\in\D_R\of{\alpha}
                \intertext{Das heißt es existiert eine Folge $z_k \in S$ mit $z_k \neq \alpha$, $z_k \to \alpha$ mit $f\of{z_k} = 0$ $\forall k$. Nach dem Eindeutigkeitssatz von Potenzreihen folgt damit}
                \impl f &= 0\text{ auf }D_R\of{\alpha}\\
                A &\coloneqq \set{z\in U: z\text{ ist Häufungspunkt von Nullstellen von }f}\\
                B &= U \setminus A
                \intertext{Behauptung 1: $A$ ist offen. Sei $z_0 \in A$, existieren $R > 0$, sodass}
                f\of{z} &= 0\quad\forall z\in D_r\of{z_0}\\
                \impl D_r\of{z_0} &\subseteq A
                \intertext{Das heißt $A$ ist offen. Behauptung 2: $B$ ist offen. Sei $w\in B$. Dann hat $w$ einen Siherheitsabstand von der Nullstelle von $f$}
                \impl \exists r > 0: D_r\of{w} &\subseteq B\\
                \intertext{Damit ist $B$ offen. Daraus, dass $U$ zusammenhänged ist, folgt, dass $A$ oder $B$ leer ist. Da $\alpha\in A$ folgt also $B = \emptyset$}
                \impl A &= U
            \end{align*}
            Da $f$ stetig ist, folgt $f\of{z} = 0~\forall z\in U$.
        \end{proof}
    \end{satz}

    \begin{korollar}
        Seinen $f, g: U\to\C$ analytisch, $U$ offen und zusammenhängend. Seien $f\of{z} = g\of{z}$ für alle $z\in S$, wobei $S$ hat Häufungspunkt in $U$. Dann gilt bereits $f = g$ auf $U$.

        \begin{proof}
            Betrachte $f-g$ und wende vorherigen Satz an.
        \end{proof}
    \end{korollar}

    \begin{beispiel}
        $\sin\of{z} = 0$ für $z = \pi k$, $k\in\Z$.
    \end{beispiel}

    \begin{satz}[Mittelwertsatz] % Satz 3
        \label{satz:mittelwertsatz}
        Sei $f: U\to\C$ analytisch und $U$ offen. Sei $\alpha\in U$, $R > 0$, $D_R\of{\alpha}\subseteq U$. Dann folgt
        \begin{align*}
            \forall 0 < r < R\colon~f\of{\alpha} &= \frac{1}{2\pi} \int_{0}^{2\pi} f\of{\alpha + r e^{it}} \dif t
        \end{align*}

        \begin{proof}
            Sei $C_r\of{\alpha}: w\of{t} = re^{it} + \alpha$
            \begin{align*}
                f\of{\alpha} &= \frac{1}{2\pi i} \int_{C_r\of{\alpha}}^{} \frac{f\of{w}}{w-\alpha} \dif w\\
                &= \frac{1}{2\pi} \int_{0}^{2\pi} \frac{f\of{re^{it} + \alpha}}{re^{it}}\cdot re^{it} \dif t\qedhere
            \end{align*}
        \end{proof}
    \end{satz}

    \begin{satz}[Maximum modulus] % Satz 2
        \marginnote{[19. Mai]}
        \label{satz:max-modulus}
        Sei $U\subseteq\C$ offen und zusammenhängend und $f: U \to \C$ eine nicht-konstante analytische Funktion. Dann hat $\abs{f}$ \underline{kein} lokales Maximum in $U$. Das heißt $\forall \alpha\in U$ und $\delta > 0$ mit $D_{\delta}\of{\alpha}\subseteq U$ existiert ein $z\in D_{\delta}\of{\alpha}$ mit $\abs{f\of{z}} > \abs{f\of{\alpha}}$.

        \begin{proof}
            Sei $\alpha\in U$. Nach Satz~\ref{satz:mittelwertsatz} gilt
            \begin{align*}
                \abs{f\of{\alpha}}&\leq \frac{1}{2\pi} \int_{0}^{2\pi} \pair{f\of{re^{it}+\alpha}} \dif t\quad\forall \alpha\in U, \overline{D_r\of{\alpha}\subseteq U}\\
                &\leq \max_{0\leq t \leq 2\pi} \abs{f\of{re^{it}+\alpha}}\tag{1}
                \intertext{Satz~\ref{satz:max-modulus} folgt, wenn wir folgendes zeigen: Ist $f$ nicht-konstant, so gibt es ein $w\in C r\of{\alpha}$ mit $\abs{f\of{w}} > \abs{f\of{\alpha}}$. Angenommen das ist der Fall, dann würde gelten}
                \abs{f\of{w}} &\leq \abs{f\of{\alpha}}\quad\forall w\in C r\of{\alpha} \text{ für $r > 0$ klein genug}
                \intertext{Darum muss $\abs{f\of{w}} = \abs{f\of{\alpha}}$ sein für alle $w\in C r\of{\alpha}$ für $ r> 0$ klein genug. Ansonsten ist}
                \abs{f\of{w_0}} &\leq \abs{f\of{\alpha}}
                \intertext{?, dass $w_0\in C r\of{\alpha}$. $w_0 = w\of{t_0} = re^{it} + \alpha$, $t_0 \in \interv{0, 2\pi}$. Aus der Stetigkeit von $\abs{f}$ folgt jetzt}
                \exists \delta > 0\colon &\abs{f\of{w\of{t}}} < \abs{f\of{w}}\quad t\in\interv{0, 2\pi}, \abs{t-t_0} \leq \delta\\
                \impl \frac{1}{2\pi} \int_{0}^{2\pi} \abs{f\of{w\of{t}}} \dif t &< \abs{f\of{\alpha}}
                \intertext{Das ist ein Widerspruch zu (1). Also haben wir}
                \abs{f\of{re^{it} + \alpha}} &= \abs{f\of{\alpha}}\quad\forall t\in\interv{0,2\pi}\text{ mit $r>0$ klein genug}
            \end{align*}
            Das heißt $\abs{f}$ ist konstant auf $\overline{D_{r_0}\of{\alpha}}$ für ein $r_0 > 0$. Damit ist nach Satz~\ref{satz:temp-8} bereits $f$ konstant auf $D_{r_0}\of{\alpha}$ und nach Satz~\ref{satz:eindeutigkeit} ist $f$ konstant auf $U$.
        \end{proof}
    \end{satz}

    \begin{korollar} % Korollar 3
        Sei $U\subseteq\C$ offen und zusammenhängend und $f: U \to\C$ analytisch. Nimmt $\abs{f}$ ein (lokales oder globales) Maximum in $U$ an, so ist $f$ konstant.
        \begin{proof}
            Umformulierung von Satz~\ref{satz:max-modulus}
        \end{proof}
    \end{korollar}

    \begin{korollar}
        Sei $U\subseteq\C$ offen und zusammenhänged und beschränkt, $f: \overline{U}\to\C$ stetig und $f$ analytisch auf $U$. Dann folgt
        \begin{align*}
            \sup_{z\in U} \abs{f\of{z}} &= \sup_{z\in\overline{U}} \abs{f\of{z}} = \sup_{z\in \partial U} \abs{f\of{z}}
        \end{align*}
        Das heißt das Maximum wird am Rand $\partial U$ angenommen.
    \end{korollar}

    \subsection{Anwendungen}

    \textbf{Situation}: Wir haben die Einheitsscheibe $D_1\of{0}$ und $f: D_1\of{0} \to \C$ analytisch mit $f\of{0} = 0$. Dann wollen wir zeigen, dass
    \begin{enumerate}[label=(\alph*)]
        \item $\abs{f\of{z}} \leq \abs{z}\quad\forall \abs{z} < 1$
        \item $\abs{f'\of{0}} \leq 1$
    \end{enumerate}

    \begin{proof}
        Definiere
        \begin{align*}
            g\of{z} &\coloneqq \begin{cases}
                                   \frac{f\of{z}}{z} &z \neq 0\\
                                   f'\of{0} &z = 0
            \end{cases}
            \intertext{Dann ist $g$ analytisch auf $D_1\of{0}$. Sei $0 < r < 1$ mit $\abs{z} = r$}
            \impl \abs{g\of{z}} &= \frac{\abs{f\of{z}}}{\abs{z}} = \frac{\abs{f\of{z}}}{r} \leq \frac{1}{r}\quad \forall 0 < r < 1
            \intertext{Nach Satz~\ref{satz:max-modulus} folgt}
            \abs{g\of{z}} &\leq \frac{1}{r}\quad\forall\abs{z} \leq r\\
            \abs{g\of{z}} &\leq \frac{1}{r}\quad\abs{z} \leq r\\
            \impl \abs{g\of{z}} &\leq 1\quad\forall \abs{z} < 1\\
            \impl \forall 0 < \abs{z} < 1\colon \frac{\abs{f\of{z}}}{\abs{z}} &= \abs{g\of{z}} \leq 1\\
            \impl \abs{f\of{z}} &\leq \abs{z}
            \intertext{und für 8b) gilt}
            \abs{f'\of{0}} &= \abs{g\of{0}} \leq 1\qedhere
        \end{align*}
    \end{proof}
    Ferner gilt: Gilt \anf{$=$} für ein $\abs{z} < 1$ in (a) oder \anf{=} in (b), so ist
    \begin{align*}
        f\of{z} &= e^{i\Theta} z
    \end{align*}
    für ein $\Theta\in\R$.

    \begin{bemerkung}
        Die obigen Aussagen kombiniert sind auch bekannt als das Lemma von Schwarz.
    \end{bemerkung}

    \textbf{Andere Situation}: Sei $f: D_1\of{0}\to\C$ analytisch, $\abs{f\of{z}} \geq 1$ für alle $\abs{z} < 1$ und $f\of{\alpha} = 0$ für ein $\abs{\alpha} < 1$. Definiere
    \begin{align*}
        B_{\alpha}\of{z} &\coloneqq \frac{z-\alpha}{1-\conj{\alpha}z}
    \end{align*}
    Dann folgt (mit Beweis, der hier fehlt), dass
    \begin{enumerate}[label=(\alph*)]
        \item $\abs{f\of{z}} \leq \abs{B_{\alpha}\of{z}}$ für alle $\abs{z} < 1$
        \item $\abs{f'\of{\alpha}} \leq \frac{1}{1-\abs{\alpha}^2}$
    \end{enumerate}

    \begin{notation}
        Sei $\mA_{\alpha}\coloneqq\set{f: D_1\of{0}\to \C\text{ analytisch und }\abs{f\of{z}} \leq 1~\forall \abs{z} < 1 \text{ und }f\of{\alpha}=0}$. Das heißt
        \begin{align*}
            \sup_{f\in\mA_{\alpha}} \abs{f\of{z}} &= \abs{B_{\alpha}\of{z}}
            \intertext{und}
            \sup_{f\in\mA_{\alpha}} \abs{f'\of{\alpha}} &= \frac{1}{1-\abs{\alpha}^2}
        \end{align*}
    \end{notation}

    \textbf{Leicht andere Frage}: Sei $\mA\coloneqq\set{f: D_{\eta}\of{0} \to \C\text{ analytisch mit }\abs{f\of{z}} \leq 1, \abs{\eta} < 1}$. Was ist
    \begin{align*}
        \sup_{f\in\mA}\abs{f'\of{\alpha}}
        \intertext{? Es ist}
        \sup_{f\in\mA}\abs{f'\of{\alpha}} &= \frac{1}{1-\abs{\alpha}^2}
    \end{align*}

    \begin{satz}[Minimum-Modulus]
        \label{satz:minimum-modulus}
        \marginnote{[20. Mai]}
        Sei $U\subseteq\C$ offen und zusammenhängend und $f: U\to\C$ analytisch und nicht-konstant. Dann kann kein Punkt $\alpha\in U$ ein lokales Minimum von $\abs{f}$ sein, außer $\abs{f\of{\alpha}} = 0$.

        \begin{proof}
            Angenommen $\abs{f\of{\alpha}}$ ist ein lokales Minimum von $\abs{f\of{z}}$, $z$ nahe $\alpha$ und $f\of{\alpha} \neq 0$. Wegen der Stetigkeit von $f$ folgt
            \begin{align*}
                \exists r > 0\colon D_r\of{\alpha} \subseteq U
                \intertext{und}
                f\of{z} &\neq 0\quad\forall z\in D_r\of{\alpha}\\
                h: D_{r}\of{\alpha} \to \C,~h\of{z} = \frac{1}{f\of{z}}
            \end{align*}
            ist analytisch auf $D_r\of{\alpha}$. $\abs{h}$ hat in $\alpha$ ein lokales Maximum. Nach Satz~\ref{satz:max-modulus} ist $h$ damit konstant in $D_r\of{\alpha}$. Damit ist $f$ konstant in $U$.
        \end{proof}
    \end{satz}

    \begin{satz} % Satz 8
        Sei $f: U \to\C$ analytisch und nicht-konstant. Dann ist $f\of{D}$ offen für jede offene Menge $D\subseteq U$.

        \begin{proof}
            Sei $f$ nicht-konstant und analytisch, $D \subseteq U$ offen, $\beta\in f\of{D}$. Dann existiert ein $\alpha\in D\colon f\of{\alpha} = \beta$. Es reicht zu zeigen, dass das Bild einer kleinen Kreisscheibe um $\alpha$ unter $f$ enthält eine kleine Kreisscheibe um $\beta$. O.B.d.A. sei $f\of{\alpha} = 0$, sonst betrachten wir $g\of{z} = g\of{z} - f\of{\alpha}$. Es gilt Kreis $C_r\of{\alpha} \subseteq D$ für $r > 0$ klein genug. Behauptung: Es existiert ein $r > 0$ klein genug, sodass
            \begin{align*}
                f\of{z} &\neq 0\quad\forall z\in C_r\of{\alpha}
            \end{align*}
            Falls nicht, dann existieren $r_n \coloneqq n^{-1}$, sodass für $n$ groß genug $z_n\in C_{r_n}\of{\alpha}, f\of{z_n} = 0$. Dann wäre $f$ aber nach Satz~\ref{satz:eindeutigkeit} konstant. Das ist ein Widerspruch.\\[.5\baselineskip]
            Sei $z_{\varepsilon} \coloneqq \inf_{z\in C_r\of{\alpha}} \abs{f\of{\alpha}} > 0$. Behauptung 2: $f\of{D_r\of{\alpha}} \supseteq D_{\varepsilon}\of{0} = D_{\varepsilon}\of{f\of{\alpha}}$. Bew: Sei $w\in D_{\varepsilon}\of{0}$. Zu zeigen ist, dass ein $z\in D_r\of{\alpha}$ existiert, sodass $f\of{z} = w \equivalent \underbrace{f\of{z} - w}_{\eqqcolon h\of{z}} = 0$. Wir haben also $h: D_r\of{\alpha}\to\C$. $z\in\partial D_r\of{\alpha} = C_r\of{\alpha}$
            \begin{align*}
                \abs{h\of{z}} &= \abs{f\of{z} - w}\geq \abs{f\of{z}} - \abs{w} \geq 2\varepsilon - \varepsilon = \varepsilon
                \intertext{und}
                \abs{h\of{\alpha}} &= \abs{f\of{\alpha} - w} = \abs{-w} < \varepsilon
            \end{align*}
            Nach Satz~\ref{satz:minimum-modulus} gibt es ein $z\in D_r\of{\alpha}$ mit $h\of{z} = 0 \impl f\of{z} = w$. Das heißt $f\of{D_r\of{\alpha}} \supseteq D_{\varepsilon}\of{0}$.
        \end{proof}
    \end{satz}

    \begin{satz} % Satz 9
        Sei $f: \C\to\C$ eine ganze Funktion mit
        \begin{align*}
            \abs{f\of{z}} &\leq \frac{1}{\abs{\Im z}}\tag{$z\in\C\setminus\R$}
        \end{align*}
        Dann ist $f = 0$ auf $\C$.

        \begin{proof}
        (mittels Skizze, nicht hier)
        \end{proof}
    \end{satz}

    \newpage


    \section{[*] Eine Umkehrung von Cauchy: Morera}

    \subsection{Satz von Morera}
    \thispagestyle{sectionpage}

    \begin{satz} % Satz 1
        \label{satz:morera}
        Sei $U\subseteq\C$ offen, $f: U\to\C$ stetig und für jedes Rechteck (achsenparallel) $R\subseteq U$ sei
        \begin{align*}
            \int_{\Gamma}^{} f\of{z} \dif z &= \int_{\partial R}^{} f\of{z} \dif z = 0\tag{$\Gamma = \partial R$}
        \end{align*}
        Dann ist $f$ analytisch auf $U$.

        \begin{proof}
            Sei $z_0\in U$, $D_r\of{z_0}\subseteq U$, $z\in D_r\of{z_0}$. Wir definieren
            \begin{align*}
                \gamma_{w,z}: z\of{t} &\coloneqq \begin{cases}
                                                     w + t\Re\of{z-w} &0\leq t \leq 1\\
                                                     \Re\of{?} + \pair{t-1}\Im\of{z-w} &1\leq t \leq 2
                \end{cases}\\
                F\of{z} &\coloneqq \int_{\gamma_{z_0, z}}^{} f\of{w} \dif w\\
                F\of{z+h} - F\of{z} &= \int_{\gamma_{z, z+h}}^{} f\of{w} \dif w\\
                \frac{F\of{z+h} - F\of{z}}{h} &= \frac{1}{h}\int_{\gamma_{z, z+h}}^{} f\of{w} \dif w\\
                \frac{F\of{z+h} - F\of{z}}{h} - f\of{z} &= \frac{1}{h} \int_{\gamma_{z, z+h}}^{} \pair{f\of{w} - f\of{z}} \dif w\to 0\text{ für } h\to 0\\
                \impl F' &= f\text{ auf } D_r\of{z_0}\\
                \impl F\text{ ist analytisch und } &f\text{ ist auch analytisch}\qedhere
            \end{align*}
        \end{proof}
    \end{satz}

    \begin{definition}
        Sei $U\subseteq\C$ offen, $f_n, f: U\to\C$ . Dann konvergiert $(f_n)_n$ lokal gleichmäßig gegen $f$, falls $f_n \to f$ gleichmäßig auf allen kompakten Teilmengen $K\subseteq U$. Das heißt
        \begin{align*}
            \forall K\subseteq U\text{ kompakt}\colon \lim_{n\to\inf} \sup_{z\in K}\abs{f_n\of{z} - f\of{z}} = 0
        \end{align*}
    \end{definition}

    \begin{satz}
        Sei $U\subseteq\C$ offen und $f_n: U\to\C$ analytisch konvergiere lokal gleichmäßig gegen $f: U\to\C$. Dann ist $f$ analytisch.

        \begin{proof}
            1) $f$ ist stetig. $z_0\in U$, $\overline{D_r\of{z_0}}\subseteq U$, dann $f_n \to f$ gleichmäßig auf $\overline{D_r\of{z_0}}$. Das heißt $f_n$ ist steig auf $\overline{D_r\of{z_0}}\impl $ f ist stetig auf $\overline{D_r\of{z_0}}$ $\impl$ $f$ ist stetig auf $U$.\\
            2) $R\subseteq D_r\of{z_0}$ ein Rechteck, $\Gamma = \partial R$
            \begin{align*}
                \int_{\Gamma}^{} f\of{z} \dif z &= \int_{\Gamma}^{} \lim_{n\toinf} f_n\of{z} \dif z = \lim_{n\toinf} \underbrace{\int_{\Gamma}^{} f_n\of{z} \dif z}_{= 0} = 0
            \end{align*}
            Das heißt $f$ ist analytisch nach Satz~\ref{satz:morera}.
        \end{proof}
    \end{satz}

    \subsection{Reflexionsprinzip von Schwarz}

    \begin{satz} % Satz 4
        \marginnote{[26. Mai]}
        \label{satz:liniensegmente}
        Sei $U\subseteq\C$ offen, $L\subseteq U$ ein Liniensegment, $f: U \to\C$ stetig und analytisch auf $U\setminus L$. Dann ist $f$ analytisch auf $U$.

        \begin{proof}
            Sei \OBDA $L \subseteq\R$, sonst sei $g\of{z} = az+b$ mit $L\subseteq g\of{\R}$ und betrachte $h = f\circ g$. Wenn $h$ auf $g^{-1}\of{U}$ analytisch ist, dann ist $f$ auf $U$ analytisch.\\
            Wir betrachten jetzt ein beliebiges Rechteck $R\subseteq U$ und wollen Satz~\ref{satz:morera} anwenden. Das heißt wir müssen zeigen, dass
            \begin{align*}
                \int_{\partial R}^{} f\of{z} \dif z &= 0
            \end{align*}
            \textsc{Fall 1}: $R\cap L = \emptyset$. In diesem Fall gilt
            \begin{align*}
                \int_{\partial R}^{} f\of{z} \dif z &= 0
            \end{align*}
            da $f$ analytisch in einer offenen Umgebung von $R$ ist.\\
            \textsc{Fall 2}: $\partial R \cap L \neq \emptyset$. Sei $\varepsilon > 0$ und $R_{\varepsilon}$ wie oben im Bild. Dann gilt
            \begin{align*}
                \int_{\partial R_{\varepsilon}}^{} f\of{z} \dif z &= 0
                \intertext{Da}
                \int_{a}^{b} f\of{x+i\varepsilon} \dif x &\to \int_{a}^{b} f\of{x} \dif x
                \intertext{folgt}
                0 = \lim_{\varepsilon \searrow 0} \int_{\partial R}^{} f\of{z} \dif z &= \int_{\partial R}^{} f\of{z} \dif z
            \end{align*}
            \textsc{Fall 3}: $L$ liegt im Inneren von $R$. Wir teilen $R$ in zwei Rechtecke $R_1, R_2$ auf, sodass $L$ auf dem Rand von $R_1, R_2$ liegt. Dann gilt nach \textsc{Fall 2}:
            \begin{align*}
                \int_{\partial R}^{} f\of{z} \dif z &= \int_{\partial R_1}^{} f\of{z} \dif z + \int_{\partial R_2}^{} f\of{z} \dif z = 0 + 0 = 0
            \end{align*}
            Nach Satz~\ref{satz:morera} ist $f$ damit analytisch auf $U$.\qedhere
        \end{proof}
    \end{satz}

    \begin{notation}
        Wir schreiben $\C_+\coloneqq\set{z\in\C: \Im z > 0}$, $\C_- \coloneqq \set{z\in \C: \Im z < 0}$.
    \end{notation}

    \begin{satz}[Reflexionsprinzip] % Satz 5
        \label{satz:reflexionsprinzip}
        Sei $D\subseteq\C_+$ oder $D\subseteq\C_-$ und $L\coloneqq \partial D \cap \R \neq \emptyset$. Sei $F: D\to\C$ analytisch, stetig auf $D \cup L$ sowie $f\of{z} \in\R~\forall z\in L$. Wir definieren $D_- \coloneqq \set{\conj{z}: z\in D}$. Dann ist die Funktion $g: D_+ \cup L \cup D_- \to \C$ mit
        \begin{align*}
            g\of{z} &\coloneqq \begin{cases}
                                   f\of{z} &z \in D \cup L\\
                                   \conj{f\of{\conj{z}}} &z\in D_-
            \end{cases}
        \end{align*}
        analytisch.

        \begin{proof}
            $g$ ist stetig auf $D_+ \cup L \cup D_-$, da $f$ analytisch auf $D, D_-$ ist und $f$ reellwertig auf $L$ ist. Es bleibt noch zu beweisen, dass $g$ auf $D$ und $D_-$ analytisch ist. Auf $D$ ist das klar, weil $g\of{z} = f\of{z}$. Sei also $z\in D_-\impl \conj{z}\in D_+$, $h\neq 0$ klein genug, dass $z + h \in D_-$. Wir betrachten den Differenzenquotient
            \begin{align*}
                \frac{g\of{z+h} - g\of{z}}{h} &= \frac{\conj{f\of{\conj{z+h}}} - \conj{f\of{\conj{z}}}}{h} = \frac{\conj{f\of{\conj{z} + \conj{h}} - f\of{\conj{z}}}}{\conj{h}} \to \conj{f'\of{\conj{z}}} \text{ für } h\to 0
            \end{align*}
            nach Satz~\ref{satz:liniensegmente} weil $f$ analytisch ist auf $D\cup L \cup D_-$.
        \end{proof}
    \end{satz}

    \begin{korollar} % Korollar 6
        Sei $U\subseteq\C$ offen und symmetrisch bezüglich der reellen Achse $\R$. Sei außerdem $f: U\to\C$ analytisch und $f$ reellwertig auf $U \cap \R$. Dann folgt
        \begin{align*}
            f\of{\conj{z}} &= \conj{f\of{z}}\quad\forall z\in U
        \end{align*}

        \begin{proof}
            Wende Satz~\ref{satz:reflexionsprinzip} auf $f: D_+\cup L \to\C$ an. Dann erhalten wir
            \begin{align*}
                g\of{z} &\coloneqq \begin{cases}
                                       f\of{z} &z\in D_+ \cup L\\
                                       \conj{f\of{\conj{z}}} & z\in D_-
                \end{cases}
            \end{align*}
            ist analytisch auf $U = D_+ \cup L \cup D_-$ und $g = f$ auf $D_+$. Nach Satz~\ref{satz:eindeutigkeit} gilt damit aber $g = f$ auf $U$. Das heißt $f\of{z} = \conj{f\of{\conj{z}}}$ für alle $z\in D_-$ und damit auch für alle $z\in D_+$.\qedhere
        \end{proof}
    \end{korollar}

    \begin{beispiel}
        \begin{enumerate}
            \item Betrachte $z\mapsto e^z = \exp\of{z}$. Es gilt $\exp\of{\conj{z}} = \conj{\exp\of{z}}$.
        \end{enumerate}
    \end{beispiel}

    \newpage


    \section{[*] Einfach zusammehängende Gebiete und der Cauchy-Integral-Satz}
    \subseccount
    \thispagestyle{sectionpage}

    Ist $f: U \to \C$ analytisch, $z_0 \in U$, $D_r\of{z_0}\subseteq U$ und $\gamma$ ein geschlossener Weg in $D_r\of{z_0}$. Dann folgt
    \begin{align*}
        \int_{\gamma}^{} f\of{z} \dif z &= 0
    \end{align*}

    Erinnerung: Stetige Linie $\gamma$ in $U$ ist eine stetige Funktion $\gamma: \interv{0,1}\to U$.\\
    $U$ ist wegzusammenhängend, wenn für alle $z, w\in U$ eine stetige FUnktion $\gamma: \interv{0,1} \to U$ existiert, sodass $\gamma\of{0} = z$, $\gamma\of{1} = w$.\\
    $U$ ist zusammenhängend, wenn für jede Partitation $U = A \cup B$ mit offenen Mengen $A, B$ (in $U$) und $A \cap B \neq \emptyset$ folgt $A = \emptyset$ oder $B = \emptyset$.\\
    $U$ ist lokal wegzusammenhängend, wenn für jedes $z_0\in U$ ein $r > 0$ existiert, sodass $U\cap D_r\of{z_0}$ wegzusammenhängend ist.

    \begin{lemma}
        Sei $U$ lokal wegzusammenhängend. Dann gilt $U$ ist genau dann zusammenhängend, wenn $U$ wegzusammenhängend ist.

        \begin{proof}
            \anf{$\Leftarrow$}: Gilt immer\\
            \anf{$\impl$}: siehe Skript
        \end{proof}
    \end{lemma}

    \begin{definition}
        Zwei Wege $\gamma_0, \gamma_1: \interv{0,1}\to U$ mit $\gamma_0\of{0} = \gamma_1\of{0}$, $\gamma_0\of{1} = \gamma_1\of{1}$ sind homotop, falls es eine stetige Funktion $\Gamma: \interv{0,1} \times \interv{0,1}\to U$ gibt mit $\gamma_0 = \Gamma\of{\cdot, 0}, \gamma_1 = \Gamma\of{\cdot, 1}$ und $\Gamma\of{0, s} = \gamma_0\of{0}, \Gamma\of{1, s} = \gamma_1\of{0}$. In diesem Fall heißt $\Gamma$ Homotopie und liefert Äquivalenzklassen.
    \end{definition}

    \begin{genv}
        Zwei geschlossene Kurven $\gamma_0, \gamma_1: \interv{0,1}\to U$ und $\gamma_0\of{0} = \gamma_0\of{1} = \gamma_1\of{0} = \gamma_1\of{1}$ sind homotop, wenn es eine Homotopie $\Gamma: \interv{0,1}\times\interv{0,1}\to U$ gibt mit obigem Ergebnis.
    \end{genv}

    \begin{definition}
        Sei $U\subseteq\C$ ein Gebiet. Dann heißt $U$ einfach zusammenhängend, wenn jede geschlossene Kurve $\gamma:\interv{0,1}\to U$ mit $z_0 \coloneqq \gamma\of{0} = ?$ homotop zu den trivialen ? $\hat{\gamma} = z_0$ ist.
    \end{definition}

    \begin{beispiel}
        Sei $U$ offen und konvex. Dann ist $U$ einfach zusammenhängend.
        \begin{proof}
            Zu $z_0, w\in U$ ist $\interv{w, z_0} = \set{w + s\pair{z_0 - w}: 0 \leq s \leq 1} \subseteq U$. $\gamma: \interv{0, 1}\to U$ ist ein geschlossener Weg, $z_0 \coloneqq \gamma\of{0}$. $\Gamma\of{t, s}\coloneqq \pair{1-s}\gamma\of{t} + s z_0$ ist eine Homotopie von $\gamma$ und $\hat{\gamma} = z_0$.
        \end{proof}
    \end{beispiel}

    \begin{beispiel}
        Sei $U$ offen und sternförmig (d.h. $\exists z_0 \in U: \interv{w, z_0}\subseteq U~\forall w\in U$). Dann ist $U$ einfach zusammenhängend.

        \begin{proof}
            \textsc{Fall 1}: $\gamma: \interv{0,1}\to U$ ist eine geschlossene Kurve mit $\gamma\of{0} = z_0 = \gamma\of{1}$. Dann funktioniert wieder $\Gamma\of{t, s}\coloneqq \pair{1-s}\gamma\of{t} + s z_0$.\\
            \textsc{Fall 2}: $\gamma\of{0} = z_1 \neq z_0$. Nehme Weg $\gamma_1: \interv{0, 1}\to U$, $\gamma_1\of{0} = z_1, \gamma_1\of{1} = z_0$. Betrachte Weg $\overline{\gamma}\coloneqq \gamma_1^{-1}\gamma_1 \gamma \gamma_1^{-1}\gamma_1 = \gamma_1^{-1} \gamma\gamma_1$, $\hat{\gamma} \coloneqq \gamma_1 \gamma_1 \gamma_1^{-1}$ geschlossener Weg von $z_0$ nach $z_0$. $\overline{\gamma}$ ist monotop zu $\hat{\gamma} = \gamma_1 \gamma \gamma_1^{-1}$. $\gamma_s\of{t}\coloneqq \gamma-1\of{st}$ für $0\leq s \leq 1$. ist ein Weg von $z_1$ nach $\gamma_1\of{s}$. $\Gamma\of{\cdot, s} = \gamma_s^{-1}\hat{\gamma}?\gamma_s$. Dann ist $\overline{\gamma}$ homotop zu $\hat{\gamma}$. (??)
        \end{proof}
    \end{beispiel}

    \begin{satz} % Satz 3
        \marginnote{[27. Mai]}
        \label{satz:homotop-int-gleich}
        Sei $U\subseteq\C$ ein Gebiet und $f: U \to\C$ analytisch sowie $\gamma_0, \gamma_1: \interv{0,1}\to U$ zwei homotope, glatte Kurven. Also insbesondere $\gamma_0\of{0} = \gamma_1\of{0}, \gamma_0\of{1} = \gamma_1\of{1}$. Dann folgt bereits, dass
        \begin{align*}
            \int_{\gamma_0}^{} f\of{z} \dif z &= \int_{\gamma_1}^{} f\of{z} \dif z
        \end{align*}
    \end{satz}

    \begin{korollar}
        Sei $U$ einfach zusammenhängend, $\gamma$ eine geschlossene Kurve in $U$, $f: U \to\C$ analytisch. Dann folgt
        \begin{align*}
            \int_{\gamma}^{} f\of{z} \dif z &= 0
        \end{align*}

        \begin{proof}
            $\gamma$ ist homotop zu $\beta\of{t}\coloneqq z_0 = \gamma\of{0}$. Also gilt nach Satz~\ref{satz:homotop-int-gleich}, dass
            \begin{align*}
                \int_{\gamma}^{} f\of{z} \dif z &= \int_{\beta}^{} f\of{z} \dif z = \int_{0}^{1} f\of{\beta\of{t}}\dot\beta\of{t} \dif t = 0\qedhere
            \end{align*}
        \end{proof}
    \end{korollar}

    \begin{korollar}
        Sei $U\subseteq\C$ einfach zusammenhängend, $f: U\to\C$ analytisch. Dann hat $f$ eine globale Stammfunktion, das heißt es existiert eine analytische Funktion $F: U \to\C$ mit $F' = f$ auf $U$.

        \begin{proof}
            Sei $z_0\in U$ und $z\in U$ mit regulärem, glatten Weg $\gamma_{z_0, z}$ von $z_0$ nach $z$. Wir definieren
            \begin{align*}
                F\of{z} &= \int_{\gamma_{z_0, z}}^{} f\of{w} \dif w
                \intertext{Das ist nach Satz~\ref{satz:homotop-int-gleich} wohldefiniert.}
                \impl F\of{z+h} - F\of{z} &= \int_{z}^{z+h} f\of{w} \dif w\\
                z\of{t} &= z+th\\
                \impl \dot z\of{t} &= h\\
                \int_{z}^{z+h} f\of{w} \dif w &= \int_{0}^{1} f\of{z+th}h \dif t\\
                \impl \frac{F\of{z+h} - F\of{z}}{h} &= \int_{0}^{1} f\of{z+th} \dif t \to \int_{0}^{1} f\of{z} \dif t = f\of{z}\\
                \impl F'\of{z} &= f\of{z}\qedhere
            \end{align*}
        \end{proof}
    \end{korollar}

    \begin{lemma} % Lemma 6
        Sei $U\subseteq\C$ ein Gebiet und $\gamma_0, \gamma_1: \interv{0,1}\to U$ zwei homotope (glatte) Kurven, $\Gamma$ Homotopie von $\gamma_1$ zu $\gamma_0$. Dann können wir $\Gamma$ so modifizieren, dass alle Zwischenkurven $\Gamma\of{\cdot, s}$ glatt sind.
        Für ein Grid auf $\interv{0,1}^2$ der Feinheit $\frac{1}{N}, N\in\N$.
        \begin{align*}
            \Delta_{j,k}^N &\coloneqq \set{\pair{t,s}: \frac{j-q}{N} \leq t \leq \frac{j}{N}, \frac{k-1}{N}\leq s \leq \frac{k}{N}}
            \intertext{Dann existiert $N\in\N$ groß genug so, dass offene Kreisscheiben $D_{j,k}$ existieren mit}
            \Gamma\of{\Delta_{j,k}^N} &\subseteq D_{j,k}\subseteq U\qedhere
        \end{align*}
    \end{lemma}

    \begin{proof}[Beweis von Satz~\ref{satz:homotop-int-gleich}]
        Sei $s_k \coloneqq \frac{k}{N}$, $\beta_0 = 0$, $\gamma_0 = \Gamma\of{\cdot, s_0}$, $\gamma_k \coloneqq \Gamma\of{\cdot, s_k}$, $\gamma_{s_N} = \gamma_1$. Es reicht zu zeigen, dass
        \begin{align*}
            \int_{\gamma_{s_k}}^{} f\of{z} \dif z &= \int_{\gamma_{s_{k+1}}}^{} f\of{z} \dif z
            \intertext{Wir hangeln uns also von einer Kurve zur nächsten. Alle Zwischenintegrale werden paarweise in entgegeben gesetzte Richtungen durchlaufen. In Summe ergeben die zusätzlichen Integrale 0.}
            \int_{\gamma_{k+1}}^{} f\of{z} \dif z - \int_{\gamma_{k}}^{} f\of{z} \dif z &= \sum_{j=1}^{N} \underbrace{\int_{\gamma_{j,k}}^{} f\of{z} \dif z}_{= 0} = 0\qedhere
        \end{align*}
    \end{proof}

    \newpage
    \marginnote{[02. Jun]} (fehlt)
    \newpage

    \begin{beispiel}
        \marginnote{[03. Jun]}
        \begin{align*}
            D &= \C\setminus\R\\
            z_0 &= 1\\
            \log 1 &= 0\\
            \log z &= \int_{1}^{z} \frac{\dif w}{w}
            \intertext{Sei $-\pi < \Arg\of{z} < \pi$}
            \int_{1}^{z} \frac{\dif w}{w} &= \int_{\gamma_1}^{} \frac{\dif w}{w} + \int_{\gamma_2}^{} \frac{\dif w}{w} \dif x
            \intertext{Dabei ist $\gamma_1: w\of{t} = 1 + \pair{\abs{z} - 1}t$, $\gamma_2: w\of{t} = \abs{z}e^{it}$. Damit gilt}
            &= \int_{1}^{\abs{z}} \frac{\dif t}{t} + \Arg\of{z} = \log\abs{z} + \int_{0}^{\Arg\of{z}} \frac{i\abs{z}e^{it}}{\abs{z}e^{it}} \dif t\\
            &= \log\abs{z} + i\Arg\of{z}
        \end{align*}
    \end{beispiel}

    \begin{beispiel}
        \begin{align*}
            D &= \C\setminus \R_+\\
            z_0 &= -1 = e^{i\pi}\\
            \log z_0 &= i\pi\\
            \impl \log y &= \int_{-1}^{z} \frac{\dif w}{w} + i\pi = \int_{\gamma_1}^{} \frac{\dif w}{w} + \int_{\gamma_2}^{} \frac{\dif w}{w} + \pi i\\
            &= \int_{\gamma_1}^{} \frac{\dif w}{w} + \int_{\gamma_2}^{} \frac{\dif w}{w} + i\pi
            \intertext{Dabei sei $\gamma_2: w\of{t} = \abs{z}e^{it}$ für $t$ von $\pi$ bis $\Theta\coloneqq \Arg\of{z}$ mit $0 < \Theta < 2\pi$}
            &= \int_{-1}^{-\abs{z}} \frac{\dif t}{t} = \log\of{\abs{z}} + \int_{\gamma_2}^{} \frac{\dif w}{w} + \pi i\\
            \impl \int_{\gamma_2}^{} \frac{\dif w}{w} &= \int_{\pi}^{\Theta} \frac{i\abs{z}e^{it}}{\abs{z}e^{it}}\dif t = i \int_{\pi}^{\Theta} \dif t = i\pair{\Theta - \pi}\\
            \impl \log z &= \log \abs{z} + i\pair{\Theta-\pi} + i\pi = \log \abs{z} + i\Arg\of{z}
        \end{align*}
        Das ist die gleiche Funktion wie im vorherigen Beispiel. Aber jetzt für $0 < \Arg\of{z} < \pi$.
    \end{beispiel}

    \begin{definition}[Komplexe Potenzen]
        Sei $n\in\N$, $z^{\frac{1}{n}} \coloneqq \exp\of{\frac{1}{n}\log z}$ wobei $\log z$ ein analytischer Zweig des Logarithmus ist.
        \begin{align*}
            z^{\frac{1}{2}} &= i = e^{i\frac{\pi}{2}}\\
            z_1 &= e^{i\frac{\pi}{4}}\\
            z_2 &= e^{i\frac{\pi}{4} + i\pi} = -i
            \intertext{(endlich viele genau $n$ verschiedene Zweige)}
            \exp\of{\frac{1}{n}\pair{\log z + 2\pi ik}} &= \exp\of{\frac{1}{n}\log z + 2\pi i \frac{k}{n}}
            \intertext{verschiedene Werte ? für $k = 0, \ldots, n-1$}
            z^{\frac{1}{3}} &= i = e^{i\frac{\pi}{2}}\\
            &= e^{i\frac{\pi}{2} + 2\pi k}\tag{$k\in\Z$}
            z_1 &= e^{i\frac{\pi}{6}}\\
            \impl z &= e^{\frac{1}{k}\pair{i\frac{\pi}{2} + 2\pi i k}} = e^{i\frac{\pi}{2} + 2\pi i \frac{k}{3}}\\
            k = 0: z_0 &= e^{i\frac{\pi}{6}}\\
            k = 1: z_1 &= e^{i\frac{\pi}{6} + i \frac{2\pi}{3}} = e^{i\frac{5\pi}{6}}\\
            k = 2: z_2 &= e^{i\frac{\pi}{6} + \frac{4\pi}{3}i} = e^{i\frac{7\pi}{6}}
            \intertext{Allgemein}
            z &= \abs{z}e^{i\Theta} = \abs{z} e^{i\Theta + 2\pi ik}\\
            z^{\frac{1}{n}} &= \abs{z}^{\frac{1}{n}} e^{i\frac{\Theta + 2\pi k}{n}}\tag{$k= 0, \ldots, n-1$}
            \intertext{$w\in\C\setminus\set{0}$, $z\in\C$}
            \impl w^z &\coloneqq \exp\of{z\log w}
        \end{align*}
    \end{definition}

    \newpage


    \section{[*] Halbierte Singularitäten}
    \thispagestyle{sectionpage}
    \subseccount

    \begin{notation}
        Eine gepunktete Kreisscheibe
        \begin{align*}
            \dot D_k\of{z_0} &\coloneqq \set{z\in\C: 0 < \abs{z - z_0} < R} = D_R\of{z_0}\setminus\set{z_0}\\
            \intertext{?}
            A_{R_1, R_2}\of{z_0} &= \set{z\in\C: R_1 < \abs{z-z_0} < R_2}\tag{Ring}
        \end{align*}
    \end{notation}

    \begin{definition}
        Sei $f: U \setminus\set{z_0} \to \C$, $z_0\in\C$, $U$ offen. $f$ hat eine isolierte Singularität in $z_0$, falls $f: \dot D_r\of{z_0}\to\C$ analytisch ist, aber \underline{nicht analytisch} auf $z_0$.
    \end{definition}

    \begin{beispiel}
        \theoremescape
        \begin{enumerate}
            \item $f\of{z} = \frac{1}{z}$
            \item \begin{align*}
                      f\of{z} &= \begin{cases}
                                     \sin z &z\neq 2\\
                                     0 &z = 2
                      \end{cases}
            \end{align*}
            \item $f\of{z} = \frac{1}{z-3}$
            \item $f\of{z} = \exp\of{\frac{1}{z}}$ für $z\neq 0$
        \end{enumerate}
    \end{beispiel}

    \begin{definition}
        $f$ habe in $z_0$ eine isolierte Singularität. Dann definieren wir
        \begin{enumerate}
            \item Die Singularität ist (auf-)hebbar, falls eine punktierte Kreisscheibe $\dot D_r\of{z_0}$ sowie eine analytische Funktion $g: D_r\of{z_0} \to \C$ existieren derart, dass $f\of{z} = g\of{z}$ für $z\in\dot D_r\of{z_0}$.
            \item Falls es analytische Funktionen $A, B: D_r\of{z_0}\to\C$ gibt mit $A\of{z_0}\neq 0$ und $B\of{z_0} = 0$ und $f\of{z} = \frac{A\of{z}}{B\of{z}}$ für $z\in\dot D_r\of{z_0}$, so hat $f$ einen Pol in $z_0$. Hat $B$ eine Nullstelle der Ordnung $k\in\N$ in $z_0$, so hat $f$ einen Pol der Ordnung $k$ in $z_0$.
            \item $f$ hat in $z_0$ ein wesentliche Singularität, falls es keine hebbare Singularität und keinen Pol in $z_0$ hat.
        \end{enumerate}
    \end{definition}

    \begin{satz}[Riemanns Prinzip für hebbare Singularitäten]
        \label{satz:krit-hebbar}
        $f$ habe in $z_0$ eine isolierte Singularität. Ist
        \begin{align*}
            \lim_{z\to z_0} \pair{z-z_0}f\of{z} &= 0
        \end{align*}
        so ist die Singularität hebbar.

        \begin{proof}
            $U$ sei eine Umgebung von $z_0$. Wir definieren
            \begin{align*}
                h\of{z} &\coloneqq \begin{cases}
                                       \pair{z-z_0}f\of{z} &z\in U\setminus\set{z_0}\\
                                       0 &z = z_0
                \end{cases}
                \intertext{Das heißt $h$ ist stetig auf $U$ und analytisch auf $U\setminus\set{z_0}$. Nach Satz~\ref{satz:morera} ist $h$ analytisch auf $U$. Es gilt außerdem}
                h\of{z_0} &= 0\\
                f\of{z} &= \frac{h\of{z}}{z-z_0} = \frac{h\of{z}-h\of{z_0}}{z-z_0} \to h'\of{z_0}\text{ für }z\to z_0
            \end{align*}
            Nach Satz~\ref{satz:morera} hat $f$ damit eine analytische Fortsetzung auf $U$.
        \end{proof}
    \end{satz}

    \begin{korollar}
        Ist $f$ beschränkt in der Nähe von $z_0$. Dann ist die Singularität in $z_0$ hebbar.

        \begin{proof}
            \begin{align*}
                \lim_{z\to z_0} \pair{z-z_0}f\of{z} = 0
            \end{align*}
            mit vorherigem Satz liefert direkt die Behauptung.
        \end{proof}
    \end{korollar}

    \begin{beispiel}
        Sei $f\of{z} = \frac{\sin z}{z}$ für $z\neq 0$. Dann ist
        \begin{align*}
            f\of{z} &= \frac{1}{z} \sum_{k=0}^{\infty} \pair{-1}^k\frac{z^{2k+1}}{(2k+1)!} = \sum_{k=0}^{\infty} \pair{-1}^k\frac{z^{2k}}{\pair{2k+1}!}
        \end{align*}
    \end{beispiel}

    \begin{satz} % Satz 5
        Sei $f$ analytisch in einer punktierten Umgebung von $z_0$ und es gebe ein $k\in\N$, sodass
        \begin{align*}
            \lim_{z\to z_0} \pair{z-z_0}^k f\of{z}&\neq 0
            \intertext{und}
            \lim_{z\to z_0}\pair{z-z_0}^{k+1} f\of{z} &= 0
        \end{align*}
        Dann hat $f$ in $z_0$ ein Pol der Ordnung $k$.

        \begin{proof}
            Sei $f: U\setminus\set{z_0}\to\C$ analytisch. Wir definieren
            \begin{align*}
                g\of{z} &\coloneqq \begin{cases}
                                       \pair{z-z_0}^{k+1}f\of{z} &z\in U\setminus\set{z_0}\\
                                       0 &z = z_0
                \end{cases}
                \intertext{Dann ist $g$ stetig auf $U$ und analytisch in $U\setminus\set{z_0}$. Nach Satz~\ref{satz:morera} ist $g$ analytisch auf $U$}
                a\of{z} &\coloneqq \frac{g\of{z}}{z-z_0} = \frac{g\of{z} - g\of{z_0}}{z-z_0} \to g'\of{z_0}\text{ für } z\to z_0\\
                A\of{z_0} &\to \lim_{z\to z_0} \pair{z-z_0}^{k}f\of{z} \neq 0\\
                \impl f\of{z} &= \frac{A\of{z}}{\pair{z-z_0}^k}\tag{$z\in U\setminus\set{z_0}$}\\
                A\of{z_0} &\neq 0
            \end{align*}
            Das heißt wir haben nach Definition einen Pol der Ordnung $k$ in $z_0$.
        \end{proof}
    \end{satz}

    \begin{satz}[Caseratti (?) Weierstraß] % Satz 6
        $f$ habe in $z_0$ eine wesentliche Singularität und $U\subseteq\C$ sei offen mit $z_0\in U$. Außerdem sei $f: U\setminus\set{z_0}\to\C$ analytisch. Dann ist $R = R_U = f\of{U\setminus\set{z_0}} = \set{f\of{z}: z\in U\setminus\set{z_0}}$ dicht in $\C$.

        \begin{proof}
            Angenommen die Aussage ist falsch. Dann existiert ein $w\in\C$ und ein $\delta > 0$, sodass
            \begin{align*}
                \abs{f\of{z} - w} &> \delta\quad\forall z\in U\setminus\set{z_0}\\
                \equivalent \frac{1}{\abs{f\of{z} - w}} &< \frac{1}{\delta}\quad\forall z\in U\setminus\set{z_0}
                \intertext{Das heißt die Funktion}
                g\of{z} &\coloneqq \frac{1}{f\of{z} - w}\tag{$z\in U\setminus\set{z_0}$}
                \intertext{Nach Satz~\ref{satz:krit-hebbar} hat $g$ eine hebbare Singularität in $z_0$. Das heißt $g$ hat eine analytische Fortsetzung auf $U$}
                \impl f\of{z} &= w + \frac{1}{g\of{z}}\tag{$z\in U\setminus\set{z_0}$}
            \end{align*}
            Das heißt $f$ hat eine hebbare Singularität oder einen Pol in $z_0$, das ist ein Widerspruch zur Annahme.
        \end{proof}
    \end{satz}

    \subsection{Laurententwicklung}

    \marginnote{[16. Jun]}Sei $f: D_R\of{z_0}$ analytisch. Dann existiert eine konvergente Potenzreihenentwicklung $f\of{z} = \sum_{k=0}^{\infty} a_k \pair{z-z_0}^k$ für $\abs{z-z_0} < R$.\\[.5\baselineskip]
    \textbf{Frage}: Was passiert, wenn $f: \dot D_R\of{z_0} \subseteq D_R\of{z_0}\setminus\set{z_0} \to \C$ analytisch ist? Oder wenn $f: A_{R_1, R_2}\of{z_0} \to \C$ analytisch ist?\\[.5\baselineskip]
    \textbf{Antwort}: Wir werden sehen, dass wir stattdessen die Reihe $\dsty f\of{z} = \sum_{n\in\Z}^{} a_n \pair{z-z_0}^n$ betrachten müssen, um ähnliche Konvergenzresultate zu erhalten.

    \begin{definition}
        Sei gegeben eine zweiseitige Folge $(\mu_n)_{n\in\Z}$. Dann ist
        \begin{align*}
            L\coloneqq &\sum_{n\in\Z}^{} \mu_n = \sum_{n=-\infty}^{\infty} \mu_n
            \intertext{konvergent, falls}
            L_- &\coloneqq \sum_{n=-\infty}^{-1} \mu_n \coloneqq \lim_{k\toinf} \sum_{n=-k}^{-1} \mu_n
            \intertext{und}
            L_+ &\coloneqq \sum_{n=0}^{\infty} \mu_k = \lim_{j\toinf} \sum_{n=0}^{j} \mu_n
            \intertext{beide existieren und}
            L &= L_- + L_+
        \end{align*}
    \end{definition}

    \begin{satz} % Satz 8
        Gegeben $(a_n)_{n\in\Z} \subseteq \C$. Dann ist
        \begin{align*}
            f\of{z}&\coloneqq \sum_{n\in\Z}^{} a_n \pair{z-z_0}^n
            \intertext{konvergent auf}
            A\of{z_0} &= A_{R_1, R_2}\of{z_0} = \set{R_1 < \abs{z-z_0} < R_2}
            \intertext{mit}
            R_2 &= \frac{1}{\limsup_{n\toinf} \abs{a_n}^{\frac{1}{n}}}\\
            R_1 &= \limsup_{n\toinf} \abs{a_{-n}}^{\frac{1}{n}} = \limsup_{n\to -\inf} \abs{a_n}^{-\frac{1}{n}}
        \end{align*}
        Beachte den Spezialfall, dass $R_1 = 0$. In diesem Fall ist $A_{0, R}\of{z_0} = \dot D_R\of{z_0}$.

        \begin{proof}
            \begin{align*}
                f_+ \of{z} &= \sum_{n=0}^{\infty} a_n \pair{z-z_0}^n
                \intertext{konvergent für $z\in D_R\of{z_0}$ mit}
                R_2 &= \frac{1}{\limsup_{n\toinf} \abs{a_n}^{\frac{1}{n}}}
                \intertext{nach Wurzelkriterium. Damit bleibt noch der Negativteil}
                f_-\of{z} &= \sum_{n=-\infty}^{-1} a_n\pair{z-z_0}^n = \sum_{k=1}^{\infty} a_{-k}\pair{z-z_0}^{-k}
                \intertext{Für Wurzelkrit. müsste gelten}
                \limsup_{k\toinf} \abs{a_{-k}\pair{z-z_0}^{-k}}^{\frac{1}{k}} &= \limsup_{k\toinf} \abs{a_{-k}}^{\frac{1}{k}} \frac{1}{\abs{z-z_0}} < 1\\
                \equivalent \limsup_{k\toinf} \abs{a_{-k}}^{\frac{1}{k}} &< \abs{z-z_0}
            \end{align*}
            Das heißt wir haben die Behauptung, da $f_+$ auf $\abs{z-z_0} < R_1$ und $f_-$ auf $\abs{z-z_0} > R_1$ konvergiert und analytisch ist. Damit ist auch $f\of{z} = f_+\of{z} + f_-\of{z}$ analytisch auf $R_1 < \abs{z-z_0} < R_2$.
        \end{proof}
    \end{satz}

    \begin{satz}[Laurententwicklung] % Satz 9
        Sei $f: A_{R_1, R_2}\of{z_0} \to \C$ analytisch. Dann wird $f$ durch eine konvergente Laurententwicklung dargestellt. Das heißt es exstiert eine Folge $(a_n)_{n\in\Z} \subseteq\C$ mit
        \begin{align*}
            f\of{z} &= \sum_{n=-\infty}^{\infty} a_n\pair{z-z_0}^n\tag{$z\in A_{R_1, R_2}\of{z_0}$}
        \end{align*}
        \begin{proof}
            Sei $R_1 < \rho_1 < \rho_2 < R_2$ und $C \rho_1 \of{z_0}, C\rho_2\of{z_0}$ Kreisscheiben um $z_0$ mit Radius $\rho_1$ bzw. $\rho_2$.\\
            \textsc{Schritt 1}: Für alle analytischen Funktion $g: A_{R_1, R_2}\of{z_0}\to\C$ gilt
            \begin{align*}
                0 &= \int_{C\rho_2\of{z_0}}^{} g\of{w} \dif w = \int_{C\rho_1\of{z_0}}^{} g\of{w} \dif w
                \intertext{Nach dem ??-Satz gilt damit}
                \int_{C\rho_1\of{z_0}}^{} g\of{w} \dif w &= \int_{\tilde{C}\rho_2\of{z_0}}^{} g\of{w} \dif w = \int_{C\rho_2\of{z_0}}^{} g\of{w} \dif w
            \end{align*}
            \textsc{Schritt 2}: Sei $z\in A_{R_1, R_2}\of{z_0}$. Dann existieren $R_1 < \rho_1 < \abs{z-z_0} < \rho_2 < R_2$. Wir definieren
            \begin{align*}
                g\of{w} &\coloneqq \begin{cases}
                                       \frac{f\of{w}- f\of{z}}{w-z} &w\in A_{R_1, R_2}\of{z_0}\setminus\set{z}\\
                                       f'\of{w} &w = z
                \end{cases}
            \end{align*}
            Dann ist $g$ stetig in $A_{R_1, R_2}\of{z_0}$ und analytisch in $A_{R_1, R_2}\of{z_0}\setminus\set{z}$. Dann gilt nach Satz~\ref{satz:morera}, dass $g$ analytisch auf $A_{R_1, R_2}\of{z_0}$ ist und nach \textsc{Schritt 1} ist damit
            \begin{align*}
                \int_{C\rho_1\of{z_0}}^{} g\of{w} \dif w &= \int_{C\rho_2\of{z_0}}^{} g\of{w} \dif w\\
                \impl \int_{C\rho_1\of{z_0}}^{} g\of{w} \dif w &= \int_{C\rho_1\of{z_0}}^{} \frac{f\of{- f\of{z}}}{w-z} \dif w\\
                &= \int_{C\rho_1\of{z_0}}^{} \frac{f\of{w}}{w-z}\dif w - f\of{z} \int_{C\rho_1\of{z_0}}^{} \frac{1}{w-z} \dif w\\
                &= \int_{C\rho_2\of{z_0}}^{} \frac{f\of{w}}{w-z} \dif w = f\of{z} \int_{C\rho_2\of{z_0}} \frac{\dif w}{w-z}\\
                \impl f\of{z} \pair{\underbrace{\int_{C\rho_2\of{z_0}}^{} \frac{\dif w}{w-z}}_{=2\pi i} - \underbrace{\int_{C\rho_1\of{z_0}}^{} \frac{\dif w}{w-z}}_{=0}} &= \int_{C\rho_2\of{z_0}}^{} \frac{f\of{w}}{w-z} \dif w - \int_{C\rho_1\of{z_0}}^{} \frac{f\of{w}}{w-z} \dif w\\
                \impl 2\pi i f\of{z} &= \int_{C\rho_2\of{z_0}}^{} \frac{f\of{w}}{w-z} \dif w - \int_{C\rho_1\of{z_0}}^{} \frac{f\of{w}}{w-z} \dif w
            \end{align*}
            Auf $C\rho_2\of{z_0}: \abs{w-z_0} = \rho_2 > \abs{z-z_0}$
            \begin{align*}
                \impl \frac{1}{w-z} &= \frac{1}{w-z_0 - \pair{z-z_0}} = \frac{1}{w-z_0} - \frac{1}{1-\frac{z-z_0}{w-z_0}}\\
                &= \frac{1}{w-z_0} \sum_{n=0}^{\infty} \pair{\frac{z-z_0}{w-z}}^n = \sum_{n=0}^{\infty} \frac{\pair{z-z_0}^n}{\pair{w-z_0}^{n+1}}\\
                \impl \int_{C\rho_2\of{z_0}}^{} \frac{f\of{w}}{w-z} \dif w &= \int_{C\rho_2\of{z_0}}^{} f\of{w} \sum_{n=0}^{\infty} \frac{\pair{z-z_0}^n}{\pair{w-z_0}^{n+1}} \dif w\\
                &= \sum_{n=0}^{\infty} \underbrace{\int_{C\rho_2\of{z_0}}^{} \frac{f\of{w}}{\pair{w-z_0}^{n+1}}}_{=2\pi i a_n} \dif w \pair{z-z_0}^n
            \end{align*}
            auf $C\rho_2\of{z_0}: \abs{z-z_0} > \rho_2 = \abs{w-z_0}$ überträgt sich das auch. Das heißt wir haben mit
            \begin{align*}
                f\of{z} &= \sum_{n=0}^{\infty} \frac{1}{2\pi i} \int_{C\rho_2\of{z_0}}^{} \frac{f\of{w}}{\pair{w-z_0}^{n+1}} \dif w \pair{z-z_0}^n + \sum_{k=0}^{\infty} \frac{1}{2\pi i}\int_{C\rho_1\of{z_0}}^{} \pair{w-z_0}^k f\of{w}\dif w \pair{z-z_0}^{-\pair{k+1}}\\
                \impl f\of{z} &= \sum_{n=0}^{\infty} a_n \pair{z-z_0}^n + \sum_{k=1}^{\infty} a_{-\pair{k+1}} \pair{z-z_0}^{-\pair{k+1}}\\
                a_n &= \frac{1}{2\pi i}\begin{cases}
                                           \int_{C\rho_2\of{z_0}}^{} \frac{f\of{w}}{\pair{w-z_0}^{n+1}} \dif w &n \geq 0\\
                                           \int_{C\rho_1\of{z_0}}^{} \frac{f\of{w}}{\pair{w-z_0}^{n+1}} \dif w &n < 0
                \end{cases}
            \end{align*}
            eine Potenzreihenentwicklung.
        \end{proof}
    \end{satz}

    \begin{bemerkung}
        Im vorherigen Beweis hängt $a_n$ nicht von $\rho_1$ und $\rho_2$ ab. Es gilt
        \begin{align*}
            a_n &= \frac{1}{2\pi i} \int_{C r\of{z_0}}^{} \frac{f\of{w}}{\pair{w-z_0}^{n-1}} \dif w\tag{$\forall n\in\Z$}
        \end{align*}
        für alle $R_1 < r < R_2$. Das zeigt sich wie folgt: Es gilt $z_0 \not\in A_{R_1, R_2}\of{z_0}$
        \begin{align*}
            \impl A_{R_1, R_2}\of{z_0} \ni w \mapsto \frac{f\of{w}}{\pair{w-z_0}^{n+1}}\text{ analytisch}
        \end{align*}
        Nach \textsc{Schritt 1} aus dem Beweis gilt der Zusammenhang also unaghngig von $R_1 < r < R_2$.
    \end{bemerkung}

    \begin{bemerkung}
        Die Laurent-Entwicklung ist eindeutig. Das heißt: Ist $f\of{z} = \sum_{n=0}^{\infty} b_n\pair{z-z_0}^n$ auf $A_{R_1, R_2}\of{z_0}$. Dann ist
        \begin{align*}
            b_n &= \frac{1}{2\pi i} \int_{C r\of{z_0}}^{} \frac{f\of{w}}{\pair{w-z_0}^{n+1}} \dif w\tag{$R_1 < r < R_2$}
        \end{align*}
        Das heißt die $b_n$ entsprechen den hergeleiteten $a_n$ und es gibt damit insbesondere nur eine Koeffizientenfolge.
    \end{bemerkung}

    \begin{korollar}
        Ist $f: U\setminus\set{z_0} \to \C$ analytisch. Dann folgt für alle $\delta > 0$ mit $D_{\delta}\of{z_0}\subseteq U$ ist $f\of{z} = \sum_{n=-\infty}^{\infty} a_n\pair{z-z_0}^n$ für alle $z\in \dot D_{\delta}\of{z_0}$. Wobei
        \begin{align*}
            a_n &= \frac{1}{2\pi i}\int_{C r\of{z_0}}^{} \frac{f\of{w}}{\pair{w-z}^{n+1}} \dif w\tag{$0 < r < \delta$}
        \end{align*}
        \begin{proof}
            Wir setzen $R_2 = \delta$ und wenden den vorherigen Satz an.
        \end{proof}
    \end{korollar}

    \begin{beispiel}
        \theoremescape
        \begin{enumerate}
            \item $\frac{\pair{z+1}^2}{z} = \frac{1}{z} + 2 + z$ für $z\neq 0$.
            \item
            \begin{align*}
                \frac{1}{z^2\pair{1-z}} &= \frac{1}{z^2} \sum_{n=0}^{\infty} z^n = \sum_{n=-2}^{\infty} z^n
            \end{align*}
            \item
            \begin{align*}
                \sin \frac{1}{z-1} &= \sum_{n=0}^{\infty} \pair{-1}^{n} \frac{\pair{\frac{1}{z-1}}^{2n+1}}{(2n+1)!}
            \end{align*}
        \end{enumerate}
    \end{beispiel}

    \begin{definition}[Definition Hauptteil, analytischer Teil] % Definition 11
        \marginnote{[17. Jun]}
        (fehlt)
    \end{definition}

    \begin{lemma} % Lemma 12
    (fehlt)
    \end{lemma}

    \begin{anwendung}[Partialbruchzerlegung]
        Sei $R\of{z} = \frac{P\of{z}}{Q\of{z}}$ mit $P, Q$ Polynome sowie $\grad P < \grad Q$. Dann ist $R$ analytisch auf $\C$ ohne die Nullstellen von $Q$. Außerdem gilt $\lim_{z\toinf} R\of{z} = 0$.\\
        Nun gebe es $n$ unterschiedliche Nullstellen von $Q$, jeweils mit Vielfachheit $k_j$. Dann gilt $\grad Q = \sum_{j=1}^{n} k_j$ sowie $Q\of{z} = \alpha \prod_{j=1}^{n} \pair{z-z_j}^{k_j}$. Behauptung: Es existieren Polynome $P_j$, $j=1, \ldots, n$ mit
        \begin{align*}
            R\of{z} &= \sum_{j=1}^{n} P_j\of{\frac{1}{z-z_j}}
        \end{align*}
        Partialbruchzerlegung von $R$.
        \begin{proof}
            $R$ hat in $z_j$ einen Pol der Ordnung $k_j$. In der Nähe von $z_j$ haben wir
            \begin{align*}
                R\of{z} &= \sum_{n=-k_j}^{\infty} a_n \pair{z-z_j}^n = \underbrace{\sum_{n=0}^{\infty} a_n\pair{z-z_j}^n}_{\text{analytisch in Nähe }z_j} + \underbrace{\sum_{n=-k_j}^{-1} a_n\pair{z-z_j}^n}_{=P_j\of{\frac{1}{z-z_j}}}
                \intertext{Das heißt}
                P_j\of{\frac{1}{z-z_j}}&\text{ existiert } \forall z\neq z_j\\
                R_1\of{z} &= R\of{z} - P_1\of{\frac{1}{z-z_1}}
                \intertext{ist analytisch für $z\neq z_1$ und $R_1\of{z}$ ist stetig in $z_1$. Nach Satz~\ref{satz:morera} ist $R_1$ analytisch in $z\neq z_j$ für $j=2, \ldots, n$. Dann ist}
                P_2\of{\frac{1}{z-z_2}}&\text{Hauptteil von $R$ in $z_2$}\\
                R_2\of{z} &= R_1\of{z} - P_2\of{\frac{1}{z-z_2}}
                \intertext{$R_2$ ist analytisch für $z\neq z_j$, $j=2, \ldots, n$ und stetig in $z_2$. $R_2$ hat eine analytische Fortsetzung auf $z\neq z_j, j=3, \ldots, n$. Diese nennen wir ebenfalls $R_2$. So folgt induktiv}
                R_j\of{z} &= R_{j+1}\of{z} - P_j\of{\frac{1}{z-z_j}\tag{$0\leq j \leq n-1$}}
                \intertext{Wir gehen auch hier wieder analog vor und verwenden die analytische Fortsetzung, die $R_{j+1}$ in $z_{j+1}$ hat. Damit gilt}
                R\of{z} &= R_n\of{z} + \underbrace{\sum_{j=1}^{n} P_j\of{\frac{1}{z-z_j}}}_{\to 0 \text{ für }z\toinf}\\
                \impl R_n\of{z} &\to 0\text{ für }\abs{z}\toinf
            \end{align*}
            Damit ist $R_n$ eine beschränkte ganze Funktion auf $\C$. Aber da $R_n$ auch analytisch auf $\C$ ist, folgt, dass $R_n$ konstant und damit gerade gleich $0$ sein muss. Damit folgt die Behauptung.
        \end{proof}
    \end{anwendung}

    \newpage


    \section{[*] Der Residuen-Satz}

    \subsection{Windungszahlen}
    \thispagestyle{sectionpage}

    $f$ hat Singularität in $z_0$
    \begin{align*}
        f\of{z} &= \sum_{n=-\infty}^{\infty} a_n\pair{z-z_0}^n\tag{$0< \abs{z-z_0}< \delta$}
        \intertext{Sei $0 < r < \delta^n$ und $C_r\of{z_0}$ der Kreis um $z_0$ mit Radius $r$. Dann ist $\dot D_r\of{z_0}$ nicht einfach zusammenhängend. Aber es lässt sich nachrechnen, dass}
        \int_{C_r\of{z_0}}^{} f\of{z} \dif z &= 2\pi i a_{-1}
    \end{align*}

    \begin{definition} % Definition 1
        Sei $f\of{z} = \sum_{n=-\infty}^{\infty} a_n\pair{z-z_0}^n$ für $0 < \abs{z-z_0} < \delta$. Wir nennen $a_{-1}$ das Residuum von $f$ in $z_0$ und schreiben
        \begin{align*}
            \Res\of{f, z_0} &\coloneqq 2\pi i a_{-1}
        \end{align*}
    \end{definition}

    \noindent\textbf{Berechnung von Residuen.}
    \begin{enumerate}
        \item $f$ hat Pol von Grad 1 bei $z_0$.
        \begin{align*}
            f\of{z} &= \frac{A\of{z}}{B\of{z}}
            \intertext{mit $A, B$ analytisch nahe $z_0$ und $A\of{z_0} \neq 0$}
            \impl a_{-1} &= \lim_{z\to z_0} \pair{z-z_0}\frac{A\of{z}}{B\of{z}} = \frac{A\of{z_0}}{B'\of{z_0}}
        \end{align*}
        \begin{proof}
            \begin{align*}
                f\of{z} &= \sum_{n=-1}^{\infty} a_n\pair{z-z_0}^n = \frac{a_{-1}}{z-z_0} + a_0 + a_1\pair{z-z_0}^n + \ldots\\
                \impl a_{-1} &= \lim_{z\to z_0} \pair{z-z_0} f\of{z} = \lim_{z\to z_0} \frac{\pair{z-z_0}A\of{z}}{B\of{z}} = \frac{A\of{z}}{B'\of{z}}\qedhere
            \end{align*}
        \end{proof}
        \item $f$ hat einen Pol der Ordnung $k$ bei $z_0$. Dann gilt
        \begin{align*}
            a_{-1} &= \frac{1}{(k-1)!}\frac{\dif^{k-1}}{\dif z^{k-1}}\interv{\pair{z-z_0}^k f\of{z}}\of{z_0}
        \end{align*}
        \begin{proof}
            \begin{align*}
                f\of{z} &= \sum_{n=-k}^{\infty} a_n \pair{z-z_0}^n\\
                \pair{z-z_0}^k f\of{z} &= \sum_{n=-k}^{\infty} a_n\pair{z-z_0}^{n+k} = \sum_{n=0}^{\infty} a_{n-k}\pair{z-z_0}^{n}\\
                \frac{\dif^{k-1}}{\dif z^{k-1}}\pair{\pair{z-z_0}^{k}f\of{z}} &= \pair{k-1}! a_{-1} + k! a_0 \pair{z-z_0} + \ldots\\
                \impl \frac{\dif^{k-1}}{\dif z^{k-1}}\pair{\pair{z-z_0}^k f\of{z}}\vert_{z=z_0} &= \pair{k-1}!a_{-1}\qedhere
            \end{align*}
        \end{proof}
        \item $f$ hat eine wesentliche Singularität in $z_0$. Dann lese $a_{-1}$ in der Laurententwicklung ab.
    \end{enumerate}

    \begin{beispiel}
        \theoremescape
        \begin{enumerate}
            \item $f\of{z} = \frac{1}{\sin z}$. Dann ist
            \begin{align*}
                \Res\of{f, 0} &= \lim_{z\to 0} \frac{z}{\sin z} = \frac{1}{\sin 0} = 1
            \end{align*}
            \item $f\of{z} = \frac{1}{z^n-1}$
            \begin{align*}
                \frac{1}{z^n-1} &= \frac{1}{\pair{z-1}\pair{z+1}\pair{z-i}\pair{z+i}}\\
                \impl \Res\of{f, i} &= \frac{1}{\pair{i-1}\pair{i+1}\pair{2i}} = \frac{1}{-4i} = \frac{i}{4}
            \end{align*}
        \end{enumerate}
    \end{beispiel}

    \marginnote{[23. Jun]}

    \noindent\textbf{Ziel}: Berechne $\dsty \int_{\gamma}^{} f\of{z} \dif z$, wobei $\gamma$ eine beliebige geschlossene Kurve ist und $f$ Singularitäten in $z_1, \ldots, z_n$ hat.

    \begin{definition}
        Sei $\gamma$ eine geschlossene Kurve und $a\not\in\gamma$. Dann heißt
        \begin{align*}
            n\of{\gamma, a} &\coloneqq \frac{1}{2\pi i} \int_{\gamma}^{} \frac{\dif z}{z-a}
        \end{align*}
        die Windungszahl von $\gamma$ um $a$.
    \end{definition}

    \begin{bemerkung}
        Wir hätten gerne, dass die Windungszahl aus der vorherigen Definition eine ganze Zahl ist. Anhand der Definition lässt sich das nicht direkt erkennen, aber die Betrachtung einiger Beispiele legt tatsächlich nahe, dass das der Fall ist:
        \begin{enumerate}
            \item $a$ liegt vollständig außerhalb von $\gamma$. Dann gilt direkt $n\of{\gamma, a} = 0$.\\
            \item Sei $\gamma = C_r\of{z_0}, a\in D_r\of{z_0}$ mit Singularität $z_0$. Dann haben wir bereits ausgerechnet, dass
            \begin{align*}
                \int_{\gamma}^{} \frac{1}{z-a} \dif z = 2\pi i
            \end{align*}
            \item Sei $\gamma = C_r\of{z_0}, a\not\in D_r\of{z_0}\cup C_r\of{z_0}$ mit Singularität $z_0$. Dann gilt direkt $n\of{\gamma, a} = 0$.\\
            \item Sei $\gamma$ ein Kreis, der sich $k$-mal um $a$ windet mit $\abs{a-z_0} < r$, $C^k_r\of{z_0} = re^{it} + z_0$, $0 \leq t \leq 2\pi k$
            \begin{align*}
                \impl n\of{C_r^k\of{z_0}, z_0} &= \frac{1}{2\pi i} \int_{0}^{2\pi k} \frac{\dot z\of{t}\dif t}{z\of{t} - z_0} = \frac{1}{2\pi i} \int_{0}^{2\pi k} \frac{ire^{it}}{re^{it}} \dif t = \frac{1}{2\pi i} i \int_{0}^{2\pi k}  \dif t = \frac{i2\pi k}{2\pi i} = k
            \end{align*}
            Damit ist $n\of{C_r^k\of{z_0}, a} = k$ für $\abs{a-z_0} < r$.
        \end{enumerate}
        Das gilt tatsächlich auch im Allgemeinen wie das folgende Lemma zeigt:
    \end{bemerkung}

    \begin{lemma}
        Sei $\gamma$ eine geschlossene Kurve und $a\not\in\gamma$. Dann gilt $n\of{\gamma, a} \in\Z$.

        \begin{proof}
            Sei $z\of{t}$ für $0\leq t \leq 1$ eine Parametrisierung von $\gamma$. Wir wollen zeigen
            \begin{align*}
                \int_{0}^{1} \frac{z\of{t}}{z\of{t}-a} \dif t &\in\Z\\
                \intertext{Definiere für $0\leq s \leq 1$}
                F\of{s} &\coloneqq \int_{0}^{s} \frac{z\of{t}}{z\of{t}-a} \dif t\\
                h\of{s} &\coloneqq \pair{z\of{s} - a}e^{-F\of{s}}\\
                \impl \dot F\of{s} &= \frac{\dot z\of{s}}{z\of{s} - a}\\
                \impl \dot h\of{s} &= \dot z\of{s}e^{-F\of{s}} - \pair{z\of{s} - a}e^{-F\of{s}}\dot F\of{s}\\
                &= e^{-F\of{s}}\interv{\dot z\of{s} - \pair{z\of{s} - a}\frac{\dot z\of{s}}{z\of{s} - a}} = 0\\
                \impl h&\text{ ist konstant}\\
                \impl z\of{0} - a &= h\of{0} = h\of{s} = \pair{z\of{s} - a}e^{-F\of{s}}\\
                \impl e^{F\of{s}} &= \frac{z\of{s} - a}{z\of{0} - a}\\
                \impl e^{F\of{1}} &= \frac{z\of{1} - a}{z\of{0} - a} = 1\\
                \impl 2\pi i n\of{\gamma, a} &= F\of{1} = 2\pi i k
            \end{align*}
            für ein $k\in\Z$. Damit gilt $n\of{\gamma, a} \in \Z$.
        \end{proof}
    \end{lemma}

    \begin{definition}
        Eine Kurve $\gamma$ heißt reguläre, geschlossene Kurve, falls $\gamma$ eine einfache geschlossene Kurve ist (keine Selbstüberschneidungen) und $n\of{\gamma, a} \in\set{0, 1}$ für alle $a\not\in\gamma$.\\
        In diesem Fall ist $\set{a\in \C: n\of{\gamma, a} = 1}$ das Innere von $\gamma$ und $\set{a\in\C: n\of{\gamma, a} = 0}$ das Äußere von $\gamma$.
    \end{definition}

    \begin{satz}[Cauchy-Residuensatz] % Satz 5
        \label{satz:cauchy-residuen}
        Sei $U\subseteq\C$ ein einfach zusammenhängendes Gebiet und $f$ analytisch auf $U$ bis auf endlich viele beliebige Singularitäten $z_1, \ldots, z_m$. Das heißt $f: U\setminus\set{z_1, \ldots, z_m} \to \C$ ist analytisch. Sei außerdem $\gamma$ eine geschlossene Kurve in $U$ mit $z_j \not\in\gamma$. Dann gilt
        \begin{align*}
            \int_{\gamma}^{} f\of{z} \dif z &= 2\pi i \sum_{j=1}^{m} n\of{\gamma, z_j}\Res\of{f, z_j}
        \end{align*}

        \begin{proof}
            Sei $k\in\set{1, \ldots, m}$
            \begin{align*}
                \impl f\of{z} &= \underbrace{A_k\of{z}}_{\text{analytisch nahe $z_k$}} + \underbrace{H_k\of{\frac{1}{z-z_k}}}_{\text{Hauptteil von $f$ nahe $z_k$}}\\
                \impl \sum_{n=-\infty}^{\infty} a_n^{(k)}\pair{z-z_k}^n &= \sum_{n=0}^{\infty} a_n^{(k)}\pair{z-z_k}^n + \underbrace{\sum_{n=-\infty}^{-1} a_n^{(k)} \pair{z-z_k}^n}_{=H_k\of{\frac{1}{z-z_k}}}
                \intertext{Betrachte, dass $H_k\of{\frac{1}{z-z_k}}$ konvergiert für alle $z\neq z_k$. Das heißt für alle $\C\setminus\set{z_k}$. Wir definieren}
                g\of{z} &= f\of{z} - \sum_{j=1}^{m} H_j\of{\frac{1}{z-z_j}}\tag{$z\in\C\setminus\set{z_1, \ldots, z_m}$}
                \intertext{analytisch auf seinem Definitionsbereich und}
                \lim_{z\to z_k} g\of{z} &= \lim_{z\to z_k} \interv{f\of{z} - H_k\of{\frac{1}{z-z_k}} - \sum_{j=1,j\neq k}^{m} H_j\of{\frac{1}{z-z_k}}}\\
                &= A_k\of{z_k} - \sum_{j=1,j\neq k}^{m} H_j\of{\frac{1}{z_k - z_j}}
                \intertext{Nach Satz~\ref{satz:morera} können wir $g$ analytisch auf $U$ fortsetzen}
                \impl \int_{\gamma}^{} f\of{z} \dif z &= \int_{\gamma}^{} g\of{z} \dif z + \sum_{j=1}^{m} \int_{\gamma}^{} H_j\of{\frac{1}{z-z_j}} \dif z\\
                \impl \int_{\gamma}^{} H_j\of{\frac{1}{z-z_j}} \dif z &= \int_{\gamma}^{} \sum_{n=1}^{\infty} a_{-n}^{(j)}\frac{1}{\pair{z-z_j}^n} \dif z = \sum_{n=1}^{\infty} a_{-n}^{(j)} \int_{\gamma}^{} \frac{\dif z}{\pair{z-z_j}^n}\\
                \intertext{Es gilt aber}
                \frac{\dif}{\dif z} \pair{z-z_j}^{-n+1} &= \pair{1-n}\pair{z-z_j}^{-n}\\
                \impl \int_{\gamma}^{} \frac{\dif z}{\pair{z-z_j}^n} &= 0\quad\forall n\geq 2
                \intertext{Damit können wir fast alle Summenterme weglassen und erhalten}
                \int_{\gamma}^{} H_j\of{\frac{1}{z-z_j}} \dif z &= a_{-1}^{(j)} \int_{\gamma}^{} \frac{\dif z}{z-z_j} = n\of{\gamma, z_j}\Res\of{f, z_j}2\pi i\\
                \impl \int_{\gamma}^{} f\of{z} \dif z &= 2\pi i \sum_{j=1}^{m} n\of{\gamma, z_j}\Res\of{f, z_j}\qedhere
            \end{align*}
        \end{proof}
    \end{satz}

    \begin{korollar}
        Sei $\gamma$ eine reguläre geschlossene Kurve und $f$ analytisch bis auf Singularitäten in $z_1, \ldots, z_m$ im Inneren von $\gamma$. Dann gilt
        \begin{align*}
            \int_{\gamma}^{} f\of{z} \dif z &= 2\pi i \sum_{j=1}^{m} \Res\of{f, z_j}
        \end{align*}
        \begin{proof}
            Folgt direkt aus Satz~\ref{satz:cauchy-residuen}.
        \end{proof}
    \end{korollar}

    \subsection{Anwendungen}

    \begin{definition} % Definition 7
        $f$ ist meromorph im Gebiet $D$, wenn $f$ nicht in $D$ ist, bis auf endlich viele beliebige Singularitäten.
    \end{definition}

    \begin{satz} % Satz 8
        Sei $\gamma$ eine reguläre geschlossene Kurve und $f$ meromorph im Inneren von $\gamma$ und auf $\gamma$. Außerdem enthalte $\gamma$ keine Nullstellen und keine Pole von $f$.\\
        Es sei $Z$ die Anzahl (gezählt mit Vielfachheiten) der Nullstellen von $f$ im Inneren von $\gamma$ und $P$ die Anzahl (gezählt mit Vielfachheiten) der Pole von $f$ im Inneren von $\gamma$. Dann gilt
        \begin{align*}
            \frac{1}{2\pi i}\int_{\gamma}^{} \frac{f'}{f} \dif z &= Z - P
        \end{align*}

        \begin{proof}
            $\frac{f'}{f}$ ist analytisch in $U$ bis auf Nullstellen und Pole von $f$. Nun habe $f$ eine Nullstelle in $z = a$. Dann gilt
            \begin{align*}
                f\of{z} &= \pair{z-a}^{k}g\of{z}\tag{$g\of{a}\neq 0$}\\
                \frac{f'\of{z}}{f\of{z}} &= \frac{k\pair{z-a}^{k-1}g\of{z} + \pair{z-a}^{k}g'\of{z}}{\pair{z-a}^k g\of{z}}\\
                &= \frac{k}{z-a} + \frac{g'\of{z}}{g\of{z}}\\
                \impl \Res\of{\frac{f'}{f}, a} &= k
                \intertext{Habe nun $f$ einen Pol der Ordnung $k$ in $a$. Dann gilt}
                f\of{z} &= \pair{z-a}^{-k}g\of{z}\\
                f'\of{z} &= -k\pair{z-a}^{-k-1}g\of{z} + \pair{z-a}^{-k}g'\of{z}\\
                \impl \frac{f'\of{z}}{f\of{z}} &= \frac{-k}{z-a} + \frac{g'\of{z}}{g\of{z}}\\
                \impl \Res\of{\frac{f'}{f}, a} &= -k
                \intertext{Damit folgt aus Satz~\ref{satz:cauchy-residuen}, dass}
                \frac{1}{2\pi i}\int_{\gamma}^{} \frac{f'}{f} \dif z &= \sum_{j=1}^{m} \Res\of{\frac{f'}{f}, z_j} = Z - P\qedhere
            \end{align*}
        \end{proof}
    \end{satz}

    \begin{korollar}[Argumentprinzip] % Korollar 9
        \marginnote{[24. Jun]}
        \label{korollar:argument}
        Sei $U\subseteq\C$ einfach zusammenhängend und offen und $f: U \to \C$ analytisch, sowie $\gamma$ eine reguläre Kurve in $U$, die keine Nullstellen von $f$ enthält. Dann gilt
        \begin{align*}
            \frac{1}{2\pi i} \int_{\gamma}^{} \frac{f'}{f} \dif z &= Z\of{f, \gamma} \coloneqq \text{Anzahl Nullstellen von $f$ im Inneren von $\gamma$ mit Vielfachheiten}
        \end{align*}

        \begin{proof}
            \textit{(fehlt)}
        \end{proof}
    \end{korollar}

    \begin{satz}[Rauché] % Satz 10
        \label{satz:rauche}
        Sei $U\subseteq\C$ ein einfach zusammenhängendes, offenes Gebiet, $f,g: U\to\C$ analytisch und $\gamma$ eine reguläre, geschlossene Kurve mit $\abs{f\of{z}} > \abs{g\of{z}}~\forall z\in\gamma$. Dann gilt $Z\of{f+g, \gamma} = Z\of{f, \gamma}$.

        \begin{proof}
            Sei $f\of{z} = A\of{z}B\of{z}$. Damit ist
            \begin{align*}
                f' &= A'B + AB'\\
                \frac{f}{f'} &= \frac{A'}{A} + \frac{B'}{B}\\
                \impl \int_{\gamma}^{} \frac{f'}{f} \dif z &= \int_{\gamma}^{} \frac{A'}{A} \dif z + \int_{\gamma}^{} \frac{B'}{B} \dif z\\
                f + g &= f\pair{1 + \frac{g}{f}}\\
                \impl Z\of{f+g, \gamma} &= \frac{1}{2\pi i} \int_{\gamma}^{} \frac{(f+g)'}{f+g} \dif z\\
                &= \underbrace{\frac{1}{2\pi i} \int_{\gamma}^{} \frac{f'}{f} \dif z}_{= Z\of{f, \gamma}} + \underbrace{\frac{1}{2\pi i} \int_{\gamma}^{} \frac{\pair{1+\frac{g}{f}}'}{\pair{1+\frac{g}{f}}} \dif x}_{=0}\\
                \intertext{Wobei der rechte Teil gleich null ist, weil $1+\frac{g\of{z}}{f\of{z}}\in D_1\of{1}$ und nach Korollar~\ref{korollar:argument}}
                &= Z\of{f, \gamma}\qedhere
            \end{align*}
        \end{proof}
    \end{satz}

    \begin{anwendung}
        Betrachte $P\of{z} = z^n + a_{n-1} z^{n-1} + \ldots + a_1 z + a_0$, $f\of{z} = z^n$, $g\of{z} = a_{n-1}z^{n-1} + \ldots + a_1 z + a_0$. Das heißt $f + g = P$. Damit gilt für großes $R$, dass $Z\of{f, C_R\of{0}} = n$. Wir betrachten
        \begin{align*}
            \abs{\frac{g\of{z}}{f\of{z}}} &= \abs{a_{n-1} z^{-1} + a_{n-2}z^{-2} + \ldots + a_0 z^{-n}}\\
            &\leq \abs{a_{n-1}} \frac{1}{\abs{z}} + \abs{a_{n-2}}\frac{1}{\abs{z}^2} + \ldots + \frac{\abs{a_n}}{\abs{z}^n}\\
            &< 1 \text{ für $R$ groß genug, da $\abs{z} = R$}\\
            \impl Z\of{P, C_R\of{0}} &= Z\of{f, C_R\of{0}} = n
        \end{align*}
        nach Satz~\ref{satz:rauche}. Somit haben wir einen alternativen Beweis zum Fundamentalsatz der Algebra gefunden.
    \end{anwendung}

    \begin{satz}[Verallgemeinerung der Cauchy-Darstelung] % Satz 11
        Sei $U$ einfach zusammenhängend und $f: U\to\C$ analytisch, sowie $\gamma$ eine reguläre, geschlossene Kurve in $U$. Dann gilt
        \begin{align*}
            f^{(k)}\of{z} &= \frac{k!}{2\pi i} \int_{\gamma}^{} \frac{f\of{w}}{\pair{w-z}^{k+1}} \dif w\tag{$k\in\N_0$}
        \end{align*}
        für alle $z$ im Inneren von $\gamma$.

        \begin{proof}
            Es gilt
            \begin{align*}
                f\of{w} &= f\of{z} + \frac{f'\of{z}}{2!}\pair{w-z} + \ldots + \frac{f^{(k)}\of{z}}{k!}\pair{w-z}^k + \ldots\\
                \impl \frac{f\of{w}}{\pair{w-z}^{k+1}} &= \frac{f\of{z}}{\pair{w-z}^{k+1}} + \ldots + \frac{f^{(k-1)}\of{z}}{\pair{w-z}^2 (k-1)!}\\
                &~~~+ \frac{f^{(k)}\of{z}}{\pair{w-z}k!} + \frac{f^{(k+1)}\of{z}}{(k+1)!} + \frac{f^{(k+2)}\of{z} (w-z)}{(k+2)!} + \ldots\\
                \Res\of{\frac{f\of{w}}{(w-z)^{k+1}}, z} &= \frac{f^{(k)}\of{z}}{k!}\\
                \impl \int_{\gamma}^{} \frac{f\of{w}}{\pair{w-z}^{k+1}} \dif w &= 2\pi i \Res\of{\frac{f\of{w}}{\pair{w-z}^{k+1}}, z} = 2\pi i \frac{f^{(k)}\of{z}}{k!}\qedhere
            \end{align*}
        \end{proof}
    \end{satz}

    \begin{satz} % Satz 12
        Sei $U$ einfach zusammenhängend und $f_n: U \to\C$ analytisch mit $f_n \to f$ lokal gleichmäßig in $U$. Dann ist $f: U \to\C$ analytisch und $f_n' \to f'$ lokal gleichmäßig in $U$.

        \begin{proof}
            Es sei $K\subseteq U$. Dann ist für $z\in K$: $D_{3r\of{z}}\of{z}\subseteq U$
            \begin{align*}
                \impl K &= \bigcup_{z\in K} \set{z} \subseteq \bigcup_{z\in K} D_{r\of{z}}\of{z}\\
                K \subseteq \bigcup_{j=1}^{M} D_{r\of{z_j}}\of{z_j} &\subseteq \bigcup_{j=1}^{M} D_{3r_j}\of{z_j} \subseteq U\tag{$r_j \coloneqq r\of{z_j}$}
                \intertext{Es reicht also zu zeigen, dass $f_n' \to f'$ auf jedem $\overline{D_{r_j}\of{z_j} \subseteq D_{3r_j}\of{z_j}}$}
                f_n'\of{z} &= \frac{1}{2\pi i} \int_{C_{2r_j}\of{z_j}}^{} \frac{f\of{w}}{\pair{w-z}^2} \dif w\\
                \impl f_n'\of{z} - f\of{z} &= \frac{1}{2\pi i} \int_{C_{2r_j}\of{z_j}}^{} \frac{f_n\of{w} - f\of{w}}{\pair{w-z}^2} \dif w \to 0 \text{ für } n\toinf
            \end{align*}
            da $f_n \to f$ auf $C_{2r_j}\of{z_j}$.
        \end{proof}
    \end{satz}

    \begin{satz} % Satz 13
        Sei $U$ eine zusammenhängende Menge und $(f_n)_n$ eine Funktionenfolge mit $f_n: U \to \C$ analytisch sowie $f_n \to f$ lokal gleichmäßig in $U$ mit $f_n\of{z} \neq 0~\forall z\in U, n\in\N$. Dann gilt entweder $f\of{z} \neq 0~\forall z\in U$ oder $f\of{z} = 0~\forall z\in U$.

        \begin{proof}
            Wir nehmen an, dass es ein $z\in U$ gibt, sodass $f\of{z} = 0$ und es existiert ein $\delta > 0$, sodass $C_{\delta}\of{z} \subseteq U$ und $f\of{w} \neq 0~\forall w\in C_{\delta}\of{z}$
            \begin{align*}
                \frac{1}{2\pi i} \int_{C_{\delta}\of{z}}^{} \frac{f'}{f} \dif w &\geq 1
                \intertext{Auch}
                \frac{1}{2\pi i} \int_{C_{\delta}\of{z}}^{} \frac{f_n'}{f_n} \dif w &= 0\quad\forall n\in\N\\
                \intertext{Aus der lokal gleichmäßigen Konvergenz folgt damit}
                0 = \frac{1}{2\pi i} \int_{C_{\delta}\of{z}}^{} \frac{f_n'}{f_n} \dif w &\to \frac{1}{2\pi i}\int_{C_{\delta}\of{z}}^{} \frac{f'}{f} \dif w \geq 1
            \end{align*}
            Demnach ist die Annahme falsch.\qedhere
        \end{proof}
    \end{satz}

    \begin{korollar} % Korollar 14
        Sei $U$ eine zusammenhängende Menge und $(f_n)_n$ eine Funktionenfolge mit $f_n: U \to \C$ analytisch sowie $f_n \to f$ lokal gleichmäßig in $U$ mit $f_n\of{z} \neq a~\forall z\in U, n\in\N$ für ein $a\in\C$. Dann gilt entweder $f\of{z} \neq a~\forall z\in U$ oder $f\of{z} = a~\forall z\in U$.
    \end{korollar}

    \subsection{Integrale via Residuensatz ausrechnen}

    \begin{beispiel}
        \marginnote{[30. Jun]}
        \begin{align*}
            \int_{-\infty}^{\infty} \frac{1}{x^4+1} \dif x &= 2\pi i \sum_{}^{} \Res\of{\frac{1}{z^4+1}, z_j}\\
            z^4 &= -1 = e^{i\pi + 2\pi i k}\\
            z &= e^{i\frac{\pi}{4} + \frac{\pi i k}{2}} = z_k\\
            z^{4} + 1 &= \prod_{k=0}^{3} \pair{z-z_k}\\
            \Res\of{\frac{1}{z^4+1}, z_j} &= \lim_{z\to z_j} \pair{z-z_j} \frac{1}{\prod_{i=0}^{3} \pair{z-z_i}}\\
            \Res &= \frac{A}{B'},\quad B = z^4+1,\quad B' = 4z^3\\
            \Res\of{\frac{1}{z^4+1}, z_j} &= \frac{1}{4z_j^3}\\
            \Res\of{\frac{1}{z^4+1}, e^{i\frac{\pi}{4}}} &= \frac{1}{4z_1^3} = \frac{-z_1}{4} = -\frac{1}{8}\pair{\sqrt{2} + i\sqrt{2}}\\
            \Res\of{\frac{1}{z^4+1}, e^{i\frac{3\pi}{4}}} &= \frac{1}{8}\pair{\sqrt{2} - i \sqrt{2}}\\
            \sum \Res\of{\ldots} &= -\frac{i}{4}\sqrt{2}\\
            \impl \int_{-\infty}^{\infty} \frac{1}{x^4+1} \dif x &= 2\pi i \pair{-\frac{i}{4}\sqrt{2}} = \frac{\pi}{\sqrt{2}}
        \end{align*}
    \end{beispiel}

    \begin{beobachtung}
        \marginnote{[1. Jul]}
        \label{beobachtung:gebr-int}
        Für $P, Q$ Polynome mit $\deg P \leq \deg Q -2$ gilt
        \begin{align*}
            2\pi i \sum_{z_j}^{} \Res\of{\frac{P\of{x}}{Q\of{x}}\log z, z_j} &= \int_{\gamma_R}^{} \frac{P\of{z}}{Q\of{z}}\log z \dif z\\
            &\to \int_{0}^{\infty} \frac{P\of{x}}{Q\of{x}}\log x \dif x - \int_{0}^{\infty} \frac{P\of{x}}{Q\of{x}}\pair{\log x + 2\pi i} \dif x
            \intertext{für $\varepsilon \to 0$ und $R\toinf$ (Schlüssellochprinzip)}
            &= -2\pi i \int_{0}^{\infty} \frac{P\of{x}}{Q\of{x}} \dif x
            \intertext{Das heißt wir haben}
            \int_{0}^{\infty} \frac{P\of{x}}{Q\of{x}} \dif x &= - \sum_{z_j}^{} \Res\of{\frac{P\of{z}}{Q\of{z}}\log z, z_j}
        \end{align*}
    \end{beobachtung}

    \begin{beispiel}
        Wir wollen $\int_{0}^{\infty} \frac{1}{x^3+1} \dif x$ mit der vorherigen Beobachtung berechnen.
        \begin{align*}
            z^3 &= -1 = e^{i\pi}\\
            z_1 &= e^{i\frac{\pi}{3}}\\
            z_2 &= e^{i\frac{\pi}{3} + \frac{2\pi i}{3}} = e^{i\pi}\\
            z_3 &= e^{i\frac{5\pi}{3}}\\
            \Res\of{\frac{\log z}{1+z^3}, z_1} &= \frac{\log z_j}{3z_j^2} = \frac{z_j \log z_j}{3z_j^3} = -\frac{z_j\log z_j}{3}\\
            \Res\of{\frac{\log z}{1+z^3}, z_1} &= -\frac{i\pi}{9}\pair{\frac{1}{2}+i\frac{\sqrt{3}}{2}}\\
            \Res\of{\frac{\log z}{1+z^3}, z_2} &= \frac{i\pi}{3}\\
            \Res\of{\frac{\log z}{1+z^3}, z_3} &= -\frac{5\pi i}{9}\pair{\frac{1}{2} - i\frac{\sqrt{3}}{2}}\\
            \impl \sum_{j=1}^{3} \Res\of{\frac{\log z}{1+z^3}, z_j} &= \ldots = -\frac{2\pi}{9}\sqrt{3}\\
            \impl \int_{0}^{\infty} \frac{1}{1+x^3} \dif x &= \frac{2}{9}\pi \sqrt{3}
        \end{align*}
    \end{beispiel}

    \begin{bemerkung}
        Der Fall $P, Q$ Polynome mit $\deg P \leq \deg Q -2$ mit einem Integral der folgenden Form ($a > 0$)
        \begin{align*}
            \int_{a}^{\infty} \frac{P\of{x}}{Q\of{x}} \dif x&
            \intertext{lässt sich analog zu Beobachtung~\ref{beobachtung:gebr-int} ähnlich umsetzen. Betrachte dazu}
            \int_{\gamma_R}^{} \frac{P\of{z}}{Q\of{z}}\log\of{z-a} \dif z&\\
            \impl \int_{a}^{\infty} \frac{P\of{x}}{Q\of{x}} \dif x &= - \sum_{z_j \in \C\setminus\linterv{a, \infty}}^{} \Res\of{\frac{P\of{x}}{Q\of{x}}\log\of{z-a}, z_j}
        \end{align*}
        Damit ergibt sich außerdem
        \begin{align*}
            \int_{0}^{a} \frac{P\of{x}}{Q\of{x}} \dif x &= \int_{0}^{\infty} \frac{P\of{x}}{Q\of{x}} \dif x - \int_{a}^{\infty} \frac{P\of{x}}{Q\of{x}} \dif x\\
            &= -\pair{\sum_{z_j}^{} \pair{\Res\of{\frac{P\of{z}\log z}{Q\of{z}}, z_j} - \Res\of{\frac{P\of{z}}{Q\of{z}}\log\of{z-a}, z_j}}}\\
        \end{align*}
    \end{bemerkung}

    \begin{beispiel}
        Sei $\alpha = \frac{1}{2}$. Wir betrachten die Funktion $\frac{1}{\sqrt{x}\pair{1+x}}$
        \begin{align*}
            z^{\frac{1}{2}} &= \exp\of{\frac{1}{2}\log z}\\
            (-1)^{\frac{1}{2}} &= \exp\of{\frac{1}{2}i\pi} = e^{i\frac{\pi}{2}} = i\\
            \Res\of{\frac{1}{\sqrt{z}\pair{1+z}}, -1} &= \frac{1}{\sqrt{-1}} = \frac{1}{i} = -i\\
            \impl \pair{1-e^{-i\pi}} \int_{0}^{\infty} \frac{1}{\sqrt{x}\pair{1+x}} \dif x &= 2\pi i \Res\of{\frac{1}{\sqrt{z}\pair{1+z}}, -1} = 2\pi\\
            \impl \int_{0}^{\infty} \frac{1}{\sqrt{x}\pair{1+x}} \dif x &= \frac{2\pi}{1-e^{i\pi}} = \frac{2\pi}{1-(-1)} = \pi
        \end{align*}
    \end{beispiel}

    \begin{beobachtung}
        Wir betrachten Integrale der Form
        \begin{align*}
            \int_{0}^{2\pi} R\of{\sin\Theta, \cos\Theta} \dif \Theta
        \end{align*}
        wobei $R$ eine rationale Funktion ist. Das heißt wir haben einen Bruch von Polynomen trigonometrischer Funktionen als Integrand
        \begin{align*}
            \int_{\abs{z} = 1}^{} f\of{z} \dif z &= \int_{0}^{2\pi} f\of{e^{it}}ie^{it} \dif t\tag{$z\of{t} = e^{it}$}\\
            \cos t &= \frac{1}{2}\pair{e^{it} + e^{-it}} = \frac{1}{2}\pair{e^{it} + \frac{1}{e^{it}}} = \frac{1}{2}\pair{z+\frac{1}{z}} \tag{$z = e^{it}$, $0\leq t <2\pi$}\\
            \sin t &= \frac{1}{2}\pair{e^{it} - e^{-it}} = \frac{1}{2}\pair{e^{it} - \frac{1}{e^{it}}} = \frac{1}{2}\pair{z-\frac{1}{z}}
            \intertext{Das heißt wir können das Integral umschreiben und erhalten}
            \int_{0}^{2\pi} R\of{\cos\of{t}, \sin\of{t}} \dif t &= \int_{\abs{z} = 1}^{} R\of{\frac{1}{2}\pair{z+\frac{1}{z}}, \frac{1}{2}\pair{z-\frac{1}{z}}} \frac{\dif z}{iz}
        \end{align*}
    \end{beobachtung}

    \begin{beispiel}
        \begin{align*}
            \int_{0}^{2\pi} \frac{\dif \Theta}{2 + \cos \Theta} &= \int_{\abs{z} = 1}^{} \frac{1}{2 + \frac{1}{2}\pair{z+\frac{1}{z}}} \frac{\dif z}{i z}\\
            &= \frac{1}{i} \int_{\abs{z} = 1}^{} \frac{\dif z}{2z + \frac{1}{2}\pair{z^2+1}}\\
            &= \frac{2}{i} \int_{\abs{z} = 1}^{} \frac{\dif z}{z^2 + 4z + 1} \\
            &= \frac{2\cdot 2\pi i}{i} \Res\of{\frac{1}{z^2+4z+1}, \sqrt{3} - 2} = \frac{2}{3}\pi\sqrt{3}
        \end{align*}
    \end{beispiel}

    \subsection{Summen}

    \textbf{Setting}: Wir wollen die Summe $ \sum_{n=-\infty}^{\infty} f\of{n}$ für eine Funktion $f$ berechnen.\\[.5\baselineskip]
    \textbf{Idee}: Wenn wir eine Funktion bauen können, die Residuen bei allen ganzen Zahlen hat, dann können wir die Summe nach dem Resdiuensatz über ein Integral darstellen. Wie kriegen wir Residuen bei allen ganzen Zahlen? Wir haben Nullstellen bei ganzen Zahlen mit $\sin\of{\pi z}$, damit scheint $(\sin\of{\pi z})^{-1}$ ein guter Kandidat. Außerdem gilt
    \begin{align*}
        \Res\of{\frac{A\of{z}}{B\of{z}}, z_j} &= \frac{A\of{z_j}}{B'\of{z_j}} \annot{=}{!} 1
        \intertext{Wir wollen $B\of{z} = \sin\of{\pi z}$ setzen. Also muss gelten $A\of{z} = \pi \cos\of{\pi z}$. Untersuche also}
        f\of{z} \frac{\pi \cos\of{\pi z}}{\sin\of{\pi z}} &= f\of{z} \pi\cot\of{\pi z}
        \intertext{Sei $\gamma_N$ eine Kontur, die alle ganzen Zahlen $k$ mit $\abs{k} \leq N$ umschlingt. Also $-N, -N + 1, \ldots, 0, 1, \ldots, N$}
        \int_{\gamma_N}^{} f\of{z}\pi\cot\of{\pi z} \dif z &= 2\pi i \interv{\sum_{n=-N, n\neq z_j}^{N} f\of{n} + \sum_{k}^{} \Res\of{f\of{z}\pi \cot\of{\pi z}, z_k}}\tag{$z_k$ Pole von $f$}
        \intertext{Wollen, dass $ \sum_{n=-\infty}^{\infty} f\of{n}$ konvergiert $\impl$ wollen $\abs{f\of{z}} \leq \frac{1}{\abs{z}^2}$. Insbesondere $\lim_{z\toinf} \abs{z f\of{z}} = 0$. Wollen auch, dass}
        \int_{\gamma_N}^{} f\of{z}\pi \cot\of{\pi z}\dif z &\to 0\text{ für } N\toinf
        \intertext{sowie eine geschickte Wahl von $\gamma_N$. Dann erhält man für $z_k$ Singularitäten von $f$ die folgende Gleichung:}
        \impl \sum_{m=-\infty, m\neq z_k}^{\infty} f\of{m} &= - \sum_{k}^{} \Res\of{f\of{z}\pi \cot\of{\pi z}, z_k}\tag{1}
    \end{align*}
    \marginnote{[07. Jul]}
    \noindent Bedingung an $f$: $\abs{f\of{z}} \leq \frac{A}{\abs{z}^2}$ für eine Konstante $A > 0$ und $\abs{z} \geq R_0 \gg 0$
    \begin{align*}
        \impl \lim_{z\toinf} \abs{zf\of{z}} &= 0
        \intertext{Wahl von $\gamma_N$: Quadrat in der Komplexen Zahlenebene mit Zentrum im Ursprung und Seitenlänge $2N + 1$. Behauptung: $\exists C > 0$, sodass}
        \sup_{z\in\gamma_N}\abs{\cot\of{\pi z}} &\leq C\quad\forall N\in\N\tag{2}
        \intertext{Angenommen diese Ungleichung gilt, dann ist}
        \int_{\gamma_N}^{} f\of{z}\cot\of{\pi z} \dif z &\leq \max_{z\in\gamma_N} \abs{f\of{z}\cot\of{\pi z}} \cdot \laenge\of{\gamma_N}\\
        &= \max_{z\in\gamma_N} \abs{f\of{z}\cot\of{\pi z}} \cdot 4(2N+1)\\
        &= \frac{\abs{zf\of{z}\cot\of{\pi z}}}{\abs{z}} \cdot 4(2N+1)\\
        &\leq \frac{C\cdot 4 (2N+1)}{\pair{N+\frac{1}{2}}}\max_{z\in\gamma_N} \abs{zf\of{z}} \to 0\text{ für }N\toinf
        \intertext{Wenn also (2) gilt, dann folgt damit auch die Formel (1) für die Summe. Das heißt wir müssen jetzt nur onch nachweisen, dass (2) gilt:}
        \cot\of{z} &= -\cot\of{-z}\\
        \intertext{Es reicht den Fall $\Re z = N+\frac{1}{2}, \Im z = N+\frac{1}{2}$ zu betrachten. Sei $z = x+iy$, $x = N + \frac{1}{2}$}
        \cot\of{\pi z} &= \frac{\cos\of{\pi z}}{\sin\of{\pi z}} = \frac{\frac{1}{2}\pair{e^{i\pi z} - e^{-i\pi z}}}{\frac{1}{2}\pair{e^{i\pi z} - e^{-i\pi z}}}\\
        &= i \frac{e^{2\pi z} + e^{-i\pi z}}{e^{i\pi z} - e^{-i\pi z}}\\
        &= i \frac{e^{2\pi i z} + 1}{e^{2\pi i z} - 1}\\
        &= i \frac{e^{2\pi i x - 2\pi y} + 1}{e^{2\pi i x - 2\pi y} - 1}
        \intertext{Wir setzen $x = N + \frac{1}{2}$ ein}
        &= i \frac{e^{2\pi i\pair{N + \frac{1}{2}} - 2\pi y} + 1}{e^{2\pi i\pair{N + \frac{1}{2}} - 2\pi y} - 1}\\
        &= i \frac{e^{i\pi - 2\pi y} + 1}{e^{i\pi - 2\pi y} - 1}\\
        &= i \frac{-e^{-2\pi y} + 1}{-e^{2\pi y} - 1} = -i \frac{1-e^{-2\pi y}}{1+e^{-2\pi y}}\\
        \abs{\cot\of{\pi z}} &= \frac{\abs{e^{2\pi y} - 1}}{1 + e^{-2\pi y}} \leq 1
        \intertext{$y = N + \frac{1}{2}$}
        \impl \cot\of{\pi z} &= i \frac{e^{2\pi i x - 2\pi \pair{N + \frac{1}{2}}}}{e^{2\pi i x - 2\pi \pair{N + \frac{1}{2}}} - 1}\\
        &= i \frac{e^{-\pi\pair{N+1}}e^{2\pi i x} + 1}{e^{-\pi (N+1)} e^{2\pi i x} - 1}\\
        \abs{\cot\of{\pi z}} &= \frac{\abs{w+1}}{\abs{w-1}}\tag{$w = e^{-\pi(N+1)} e^{2\pi i x}$}\\
        &\leq \frac{e^{-\pi (N+1)} + 1}{1 - e^{-\pi (N+1)}}
        \intertext{Betrachte}
        f\of{t} &\coloneqq \frac{e^{-t} + 1}{1- e^{-t}}\\
        \impl f'\of{t} &= \frac{\pair{1-e^{-t}}\pair{-e^{-t}} - \pair{e^{-t} + 1}e^{-t}}{\pair{1-e^{-t}}^2}\\
        &= (1-e^{-t})^2 \pair{-e^{-t} - e^{-t} - e^{-2t} - e^{-t}}\\
        &= -\pair{1-e^{-t}}^2\pair{e^{-t} +e^{-2t}} < 0
        \intertext{Gemeinsam mit der Symmetrie gilt also (2) mit $C = 2$.}
    \end{align*}

    \begin{beispiel}
        \begin{align*}
            \sum_{n=1}^{\infty} \frac{1}{n^2} &= \frac{1}{2} \sum_{n=-\infty, n\neq 0}^{\infty} \frac{1}{n^2}\\
            \intertext{Mit $f\of{z} = z^{-2}$ gilt nach unserer vorherigen Erkentnis:}
            &= - \frac{1}{2}\Res\of{\frac{\pi \cot\of{\pi z}}{z^2}, 0}\\
            \cot\of{z} &= \frac{1}{z} - \frac{z}{3} - \frac{z^2}{45} - \ldots\\
            \impl \frac{\pi \cot\of{\pi z}}{z^2} &= \frac{\frac{\pi}{\pi z} - \frac{\pi^2 z}{3} - \frac{\pi^3 z^2}{45}}{z^2} - \ldots\\
            &= \frac{1}{z^3} - \underbrace{\frac{\pi^2}{3z}} - \frac{\pi^3}{45} - \ldots\\
            \impl \sum_{n=1}^{\infty} \frac{1}{n^2} &= \frac{\pi^2}{6}
        \end{align*}
    \end{beispiel}

    \begin{beobachtung}
        Wir wollen analog auch Summen der Form $ \sum_{n\in\Z}^{} (-1)^n f\of{n}$ ausrechnen. Es gilt $\Res\of{\frac{\pi}{\sin\of{\pi z}}, n} = (-1)^n$. Damit folgt analog
        \begin{align*}
            \sum_{n\in\Z, z\not\in\sing\of{f}}^{} (-1)^n f\of{n} &= - \sum_{z_j \in \sing\of{f}}^{} \Res\of{f\of{z} \frac{\pi}{\sin\of{\pi z}}, z_j}
        \end{align*}
    \end{beobachtung}

    \begin{beispiel}
        \begin{align*}
            \sum_{n\in\Z, n\neq 0}^{} \frac{(-1)^n}{n^2} &= -\Res\of{\frac{\pi}{z^2\sin\of{\pi z}}, 0} = - \frac{\pi^2}{6}
        \end{align*}
    \end{beispiel}

    \begin{beobachtung}[Summen mit Binomialkoeffizient]
        \begin{align*}
            \frac{(1+z)^n}{z^{n+1}} &= \sum_{j=1}^{n} \binom{n}{j} z^{j-k-1}\\
            &= \binom{n}{k} z^{-1} + ??\\
            \intertext{Sei $S$ die Kurve, die einmal um 0 herumläuft}
            \binom{n}{k} &= \frac{1}{2\pi} \int_{C}^{} \frac{(1+z)^n}{z^{k-1}} \dif u\\
            \binom{2n}{n} &= \frac{1}{2\pi i} \int_{C}^{} \frac{(1+z)^n}{z^{n+1}} \dif z
            \intertext{$C: z\of{t} = e^{it}$}
            \impl \binom{2n}{n} &\leq 2^{2n} = 4^n
        \end{align*}
    \end{beobachtung}

    \newpage~\thispagestyle{empty}\newpage


    \section{[*] Gewöhnliche Differentialgleichungen}
    \thispagestyle{sectionpage}

    \subsection{Definition}
    \marginnote{[08. Jul]}

    \textbf{Motivation}: Sei $x\of{t}$ die Position, dann ist $\dot x$ die Geschwindigkeit und $\ddot x $ die Beschleunigung. Nach Newton gilt $m \cdot \ddot x = F$, wobei $F$ die Kraft und $m$ die Masse ist.

    \begin{definition}[Gewöhnliche DGL]
        Seien $n, N\in\N$ und $D\subseteq \R^{1+(n+1)N} = \R \times\R^{(n+1)N}$ sowie $F: D\to\R^N$. Dann heißt eine Gleichung der Form
        \begin{align*}
            F\of{t, x, \dot x, \ddot x, \ldots, x^{(n-1)}, x^{(n)}} &= 0\tag{1}
        \end{align*}
        mit $t\in\R$ und $x$ Funktionswert, $n$ mal differenzierbar $x\of{t} \in\R^N$ heißt gewöhnliche Differentialgleichung (DGL).\\
        Die Lösung von (1) ist eine $n$-mal differenzierbare Funktion $\lambda: I \to\R^N$, wobei $I$ ein Intervall ist und gilt
        \begin{align*}
            F\of{t, \lambda\of{t}, \dot\lambda\of{t}, \ldots, \lambda^{(n)}\of{t}} &= 0\quad\forall t\in I
        \end{align*}
        Dabei nennen wir $n$ die Ordnung der DGL.
    \end{definition}

    \begin{beispiel}
        Sei $D = \R^3$ und $F\of{t, x, \dot x} = t \dot x + {\dot x}^2 - x$. Dann ist $\lambda\of{t} = 1 + t$ eine Lösung von $F$. Denn es gilt
        \begin{align*}
            \lambda\of{t} &= 1+t\\
            \impl \dot\lambda\of{t} &= 1\\
            \impl F\of{t, \lambda\of{t}, \dot\lambda\of{t}} &= t + 1^2 - (1+t) = 0
        \end{align*}
    \end{beispiel}

    \begin{beispiel}
        Sei $D = \R^3$ und $F\of{t, x, \dot x} = \dot x - x + t^2$. Dann können wir auch schreiben
        \begin{align*}
            f\of{t, x} &\coloneqq x-t^2\\
            \impl F\of{t, x, \dot x} = 0 \equivalent \dot x &= x - t^2 = f\of{t, x}
            \intertext{Die Lösung ist dann}
            \lambda\of{t} &= 2 + 2t + t^2 + e^t
        \end{align*}
        auf $\pair{-\infty, \infty}$.
    \end{beispiel}

    \begin{beispiel}
        Sei $D = \R^3$ und $\dot x = x^{2}t$. Dann gibt es Lösungen $\lambda_{\alpha}\of{t} = \frac{2\alpha}{2 - \alpha t}$ für $\alpha\in\R$.
    \end{beispiel}

    \begin{beispiel}[DGL ohne Lösung]
        \theoremescape
        \begin{enumerate}
            \item Die Gleichung ${\dot x}^2 + 1 = 0$ hat keine Lösung. Dieses Beispiel ist allerdings recht künstlich.
            \item Definiere
            \begin{align*}
                \charfunc_{\Q}\of{t} &= \begin{cases}
                                            1 &t\in\Q\\
                                            0 &t\not\in\Q
                \end{cases}
            \end{align*}
            Dann hat $\dot x = \charfunc_Q$ ebenfalls keine Lösung.
        \end{enumerate}
    \end{beispiel}

    \subsection{Lineare DGL erster Ordnung}

    \begin{beispiel}[Einfachster Fall]
        Wenn wir die Gleichung $\dot x = f\of{t}$ haben, dann ist jede Lösung der Form
        \begin{align*}
            x\of{t} &= \int_{}^{} f\of{t} \dif t + C
        \end{align*}
    \end{beispiel}

    \begin{definition}
        Eine allgemeine lineare DGL der Ordnung 1 und mit $N=1$ hat die Form
        \begin{align*}
            \dot x + a\of{t}x &= b\of{t}
        \end{align*}
        Wenn $b$ nicht die Nullfunktion ist, dann nennen wir die Gleichung außerdem inhomogen und $b$ die Inhomogenität der Gleichung. Eine Gleichung der Form
        \begin{align*}
            \dot x + a\of{t}x &= 0
        \end{align*}
        also Spezialfall $b\of{t} = 0$ nennen wir dann homogen.
    \end{definition}

    Wie löst man den homogenen Fall?

    \begin{bemerkung}
        Sei
        \begin{align*}
            \dot x\of{t} &= -a\of{t} x\of{t}
        \end{align*}
        Dann ist zum Beispiel $x\of{t} = 0$ eine Lösung. Wir wollen auch noch die anderen Lösungen finden. Also angenommen $x\of{t} \neq 0$. Dann gilt
        \begin{align*}
            \frac{\dot x\of{t}}{x\of{t}} &= - a\of{t}\\
            \equivalent \frac{\dif }{\dif t}\ln\abs{x\of{t}} &= -a\of{t}\\
            \impl \ln\abs{x\of{t}} &= -\int_{}^{} a\of{t} \dif t + c_1\\
            \impl \abs{x\of{t}} &= \exp\of{ - \int_{}^{} a\of{t} \dif t + c_1}\\
            &= c_2 \exp\of{- \int_{}^{} a\of{t} \dif t}\tag{$c_2\coloneqq e^{c_1}$}\\
            \impl c_2 &= \abs{x\of{t}\exp\of{ \int_{}^{} a\of{t} \dif t}}
            \intertext{Da wir angenommen haben, dass $x$ keine Nullstellen hat, kann die rechte Seite der Gleichung nicht das Vorzeichen wechseln. Wir können also den Betrag weglassen und die Lösungen sind}
            \impl x\of{t} &= c \exp\of{ - \int_{}^{} a\of{t} \dif t}\tag{$c\in\R$}
            \intertext{bzw.}
            x\of{t} &= c \exp\of{-A\of{t}}\tag{$A' = a$}
        \end{align*}
    \end{bemerkung}

    \begin{beispiel}
        Suchen Lösung für $\dot x + 2t+x = 0$. Dann ist $a\of{t} = 2t$
        \begin{align*}
            \impl \int_{}^{} a\of{t} \dif t &= t^2\\
            \impl x\of{t} &= c e^{-t^2}
        \end{align*}
    \end{beispiel}

    \begin{bemerkung}[Anfangswertproblem]
        Wir suchen die Lösung für eine Gleichung der Form $\dot x + a\of{t}x = 0$ mit der Bedingung, dass $x\of{t_0} = x_0$. Dann lässt sich (z.B. wie oben) zeigen, dass
        \begin{align*}
            \abs{\frac{x\of{t}}{x\of{t_0}}} &= \exp\of{-\int_{t_0}^{t} a\of{s} \dif s} > 0
            \intertext{Behauptung: $x\of{t}$ hat dasselbe Vorzeichen wie $x\of{t_0}$. Dann gilt}
            x\of{t} &= x\of{t_0} \exp\of{ - \int_{t_0}^{t} a\of{s} \dif s}
        \end{align*}
    \end{bemerkung}

    \begin{beispiel}
        Sei $\dot x + \sin\of{t}x = 0$, sowie $x\of{0} = \frac{3}{2}$. Dann gilt nach der Lösungsformel mit $a\of{t} = \sin\of{t}$, dass
        \begin{align*}
            x\of{t} &= \frac{3}{2}\exp\of{- \int_{0}^{t} \sin\of{s} \dif s} = \frac{3}{2}e^{\cos\of{t} - 1}
        \end{align*}
    \end{beispiel}

    Wir können also homogene Gleichungen lösen. Wie übertragen wir das auf inhomogene Gleichungen?

    \begin{bemerkung}
        Betrachte Gleichung der Form
        \begin{align*}
            \dot x + a\of{t} x &= b\of{t}
            \intertext{und ergänze Integrierenden Faktor}
            \mu\of{t} \dot x + \mu\of{t}a\of{t} x &= \mu\of{t} b\of{t}\\
            \impl \frac{\dif}{\dif t}\pair{\mu\of{t}x\of{t}} &= \mu\of{t} \dot x\of{t} + \dot\mu\of{t} x\of{t}\\
            &= \mu\of{t} \dot x\of{t} + \mu\of{t} a\of{t} x\of{t}
            \intertext{Brauchen}
            \dot\mu &= \mu a\\
            \mu\of{t} &= \exp\of{ \int_{t_0}^{t} a\of{s} \dif s}\\
            \impl \frac{\dif}{\dif t}\pair{\mu\of{t}x\of{t}} &= \mu\of{t}b\of{t}\\
            \impl \mu\of{t} x\of{t} - \underbrace{\mu\of{t_0}}_{=1} x\of{t_0} &= \int_{t_0}^{t} \mu\of{s} b\of{s} \dif s\\
            \impl x\of{t} &= \frac{?}{\mu\of{t}}x\of{t_0} + \mu\of{t}^{-1} \int_{t_0}^{t} \mu\of{s}b\of{s} \dif s\\
            \frac{1}{\mu\of{t}} &= \exp\of{- \int_{t_0}^{t} a\of{s} \dif s}
        \end{align*}
    \end{bemerkung}

    \subsection{Populationsmodelle}

    \marginnote{[14. Jul]}
    \noindent \textbf{Motivation}: Es gibt unterschiedliche Möglichkeit, Populationen über Zeit $p\of{t}$ zu modellieren. Zum Beispiel $p\of{t_0} = p_0$ und $\dot p = a p$. In diesem Fall wäre die Lösung $p\of{t} = p_0 \exp\of{at}$. Eine zweite Möglichkeit ist mit einem zusätzlichen Stressfaktor und der DGL
    \begin{align*}
        \dot p &= ap - bp^2
    \end{align*}

    \begin{lemma}
        Ist $\dot p = f\of{p}$ mit $f$ stetig und existiert $\lim_{t\toinf} p\of{t} = p_{\infty}$ so ist $f\of{p_{\infty}} = 0$.
        \begin{proof}
            \begin{align*}
                "p\of{t_2} - p\of{t_1} &= \int_{t_1}^{t_2} \dot p\of{s} \dif s = \int_{t_1}^{t_2} f\of{p\of{s}} \dif s\\
                \underbrace{p\of{t+1} - p\of{t}}_{\to 0} &= \int_{t}^{t+1} f\of{p\of{s}} \dif s - f\of{p_{\infty}} + f\of{p_{\infty}}\\
                &= \underbrace{\int_{t}^{t+1} \pair{f\of{p\of{s}} - f\of{p_{\infty}}} \dif s}_{\to 0\text{ für }t\toinf} + f\of{p_{\infty}}\\
                \impl 0 &= f\of{p_{\infty}}\qedhere
            \end{align*}
        \end{proof}
    \end{lemma}

    \begin{genv}
        Betrachte $\dot p = ap - bp^2$. Dann ist
        \begin{align*}
            \frac{\dot p}{ap - bp^2} &= 1\\
            \impl \int_{t_0}^{t} \frac{\dot p\of{s}}{ap\of{s} - bp^2\of{s}} \dif s &= \int_{t_0}^{t}  \dif s = t - t_0\\
            &= \int_{p\of{t_0}}^{p\of{t}} \frac{\dif r}{ar - br^2}
            \intertext{Nutze Partialbruchzerlegung}
            \frac{1}{ar - br^2} &= \frac{1}{r\pair{a - br}}\\
            &= \frac{A}{r} + \frac{B}{a-br}\\
            &= \frac{A\pair{a-br}+Br}{r\pair{a-br}}\\
            \impl 1 &= A\pair{a-br} + Br = Aa - Abr + Br = \pair{B - Ab}r +Aa\\
            \impl A = \frac{1}{a}\quad & \quad B = \frac{b}{a}\\
            \int_{p_0}^{p} \frac{\dif r}{ar - br^2} &= \frac{1}{a} \int_{p_0}^{p} \pair{\frac{1}{r} + \frac{b}{a-br}} \dif r\\
            &= \frac{1}{a}\pair{\ln \abs{p} - \ln \abs{a-br} - \ln \abs{p_0} + \ln\abs{a-bp}}\\
            &= \frac{1}{a}\pair{\ln \abs{\frac{p}{p_0}} + \ln \abs{\frac{a-bp_0}{a-bp}}} = \frac{1}{a}\ln \abs{\frac{p}{p_0} \cdot \frac{a-bp_0}{a-bp}}
            \intertext{Das muss aber immernoch gleich $t - t_0$ sein. Das heißt es folgt}
            \impl \frac{p\of{t}}{p_0}\cdot \frac{a-bp_0}{a-bp\of{t}} &= e^{a\pair{t-t_0}}\\
            \impl p_0 \pair{a-bp\of{t}} &= p\of{t}\pair{a-bp_0}\\
            \impl p\of{t} &= \frac{a p_0 e^{a \pair{t-t_0}}}{a - bp_0 + b p_0 e^{a\pair{t-t_0}}}\\
            \impl p\of{t} &= \frac{a p_0}{b p_0 + \pair{a-bp_0}e^{-a\pair{t-t_0}}} \to_{t\toinf} \frac{a p_0}{b p_0} = \frac{a}{b}
        \end{align*}
    \end{genv}

    \subsection{Warum existieren eindeutige Lösungen?}

    \begin{beispiel}[DGL mit nicht-eindeutiger Lösung]
        Betrachte $\dot y = \sqrt{y}$. Dann ist eine mögliche Lösung
        \begin{align*}
            2 y^{\frac{1}{2}} &= t+c\\
            \impl y &= \pair{\frac{t+c}{2}}^2
            \intertext{Betrachte andererseits die Lösung}
            y_s\of{t} &= \begin{cases}
                             0 &0\leq t \leq s\\
                             \frac{\pair{t-s}^2}{4} &t > s
            \end{cases}
        \end{align*}
    \end{beispiel}

    \begin{genv}
        Sei $\dot x = f\of{t, x}$ für $f: U \to \R^n$ und $x\of{t_0} = x_0\in\R^N$. mit $U\subseteq \R\times\R^N$ offen.\\
        \textsc{Schritt 1}: Umwandeln in Integral-Gleichung.\\
        \begin{align*}
            x\of{t} - x\of{t_0} &= \int_{t_0}^{t} \dot x \of{s} \dif s &= \int_{t_0}^{t} f\of{s, x\of{s}} \dif s\\
            \impl x\of{t} &= x_0 + \int_{t_0}^{t} f\of{s, x\of{s}} \dif s
        \end{align*}
    \end{genv}

    \subsection{Picard-Lindelöff}

    \marginnote{[15. Jul]}

    \textbf{Situation}: $D\subseteq \R\times\R^N$ offen und $\pair{t_0, x_0}\in D$ mit $f: D\to\R^N$. Wollen Lösung von
    \begin{align*}
        \dot x = f\of{t, x}\tag{1}
    \end{align*}
    und $x\of{t_0} = x_0$. Erst einmal falsch. Frage: Existiert ein Intervall $\interv{t_0 - \alpha, t + \alpha}$ und eine Funktion $\lambda: \interv{t_0 - \alpha, t_0 + \alpha} \to\R^N$ mit
    \begin{align*}
        \frac{\dif}{\dif t}\lambda\of{t} &= f\of{t, \lambda\of{t}}\tag{1'}\\
        \lambda\of{t_0} &= x_0\\
        \pair{t, \lambda\of{t}} &\in D
    \end{align*}
    für $t\in\interv{t_0 - \alpha, t_0 + \alpha}$.

    \begin{lemma} % Lemma 1
        Jede Lösung von (1) ist Lösung von
        \begin{align*}
            \lambda\of{t} &= x_0 + \int_{t_0}^{t} f\of{s, \lambda\of{s}} \dif s
        \end{align*}
        und umgekehrt jede stetige Funktion $\lambda$, die Lösung von (2) ist, ist Lösung von (1').

        \begin{proof}
            Haben $(t_0, x_0)\in D$. Dann existiert ein Zylinder $Z_{a,b} = Z_{a,b}\of{t_0, x_0} = \interv{t_0 - a, t_0 + a} times \overline{B_b^N\of{x_0}}$.
        \end{proof}
    \end{lemma}

    \begin{satz}[Picard-Lindelöff, lokal]
        \label{satz:temp-2}
        Es gebe eine Konstante $L$ so, dass
        \begin{align*}
            \abs{f\of{s, x_1} - f\of{s, x_2}} &\leq L \abs{x_1 - x_2}\quad\forall x_1, x_2\in\overline{B_b\of{x_0}}, s\in\interv{t_0 - a, t_0 + a}
        \end{align*}
        Dann existiert genau eine Lösung von (2) (also auch von (1')) auf dem Intervall $\interv{t_0 - \alpha, t_0 + \alpha}$ über
        \begin{align*}
            \alpha &\coloneqq \min\of{a, \frac{L}{M}}\\
            m &\coloneqq \max_{(s, x)\in Z_{a,b}} \abs{f\of{s, x}}
        \end{align*}
        \begin{proof}
        (Später)
        \end{proof}
    \end{satz}

    \begin{bemerkung}
        Sei $K\subseteq \R\times\R^N$ kompakt und $f: K \to\R^N$ stetig mit
        \begin{align*}
            \abs{f\of{s, x_1} - f\of{s, x_2}} &\leq L_2\abs{x_1 - x_2}\quad\forall \pair{s, x_1}, \pair{s, x_2} \in K
        \end{align*}
        Dann existiert ein $L\geq 0$, sodass
        \begin{align*}
            \abs{f\of{s, x_1} - f\of{s, x_2}} &\leq L \abs{x_1 - x_2}\quad\forall (s, x_1), (s, x_2)\in K
        \end{align*}

        \begin{proof}
        (Später)
        \end{proof}
    \end{bemerkung}

    \begin{bemerkung}
        Ist $f$ stetig differenzierbar in $x$ für ?? $s$ und $\abs{\nabla f\of{s, x}}$ ist beschränkt in $x$ für festes $s$. Dann erfüllt $f$ die die gleichmäßige Lipschitz-Bedingung.

        \begin{proof}
            \begin{align*}
                \abs{f\of{s, x_1} - f\of{s, x_2}} &\leq \max_{(s, x)\in D} \abs{\nabla_x f\of{s, x}}\abs{x_1 - x_2}\qedhere
            \end{align*}
        \end{proof}
    \end{bemerkung}

    \begin{proof}[Beweis von Satz~\ref{satz:temp-2}]
        Picard-Iterierten: Setzen $\lambda_0\of{0} \coloneqq x_0$. Für $n\in\N_0$ setzen wir
        \begin{align*}
            \lambda_{n+1}\of{t} &= x_0 + \int_{t_0}^{t} f\of{s, \lambda_n\of{s}} \dif s
        \end{align*}
        \textsc{Schritt 1}: Für $\alpha = \min\of{a, \frac{b}{M}}$ ist $\lambda_n$ wohldefiniert auf $Z_{\alpha, b}$ für alle $n\in\N_0$. Für $n=0$ gilt das direkt. Per Induktion lässt sich außerdem nachweisen, dass
        \begin{align*}
            \abs{\lambda_n\of{t} - x_0} &\leq M\abs{t - t_0} \leq b\tag{4}
        \end{align*}
        Der Induktionsanfang ist klar und angenommen (4) gilt für $\lambda_n$. Dann ist
        \begin{align*}
            \abs{\lambda_{n+1}\of{t} - x_0} &= \abs{ \int_{t_0}^{t} f\of{s, \lambda_n\of{s}} \dif s}\\
            &\leq \int_{t_0}^{t} \abs{f\of{s, \lambda_n\of{s}}} \dif s \leq M \abs{t - t_0}
        \end{align*}
        Damit und mit $M \abs{t-t_0} \leq b$ folgt der Schritt.\\
        \textsc{Schritt 2}:
        \begin{align*}
            \lambda_{n+1}\of{t} - \lambda_n\of{t} &\leq \int_{t_0}^{t} \abs{f\of{s, \lambda_n\of{s}} - f\of{s, \lambda_{n-1}\of{s}}} \dif s\\
            &\leq L  \int_{t_0}^{t} \abs{\lambda_n\of{s} - \lambda_{n-1}\of{s}} \dif s\\
            \lambda_n\of{t} - x_0 &= \sum_{j=1}^{n} \pair{\lambda_j\of{t} - \lambda_{j+1}\of{t}}\\
            \abs{\lambda_2\of{t} - \lambda_1\of{t}} &\leq L \int_{t_0}^{t} \abs{\lambda_1\of{s} - \lambda_0\of{s}} \dif s\\
            &\leq L M \int_{t_0}^{t} \pair{s - t_0} \dif s = \frac{LM}{2}\pair{t-t_0}^2\\
            \abs{\lambda_3\of{t} - \lambda_2\of{t}} &\leq L \int_{t_0}^{t} \abs{\lambda_2\of{s} - \lambda_1\of{s}} \dif s\\
            &\leq \frac{ML^2}{2} \int_{t_0}^{t} \pair{s-t_0}^2 \dif s = \frac{ML^2}{2\cdot 3}\abs{t-t_0}^3
            \intertext{Per Induktion}
            \abs{\lambda_{n+1}\of{t} - \lambda_n\of{t}} &\leq \frac{ML^n}{(n+1)!}\abs{t-t_0}^{n+1}\tag{$\ast$}\\
            \abs{\lambda_{n+1}\of{t} - \lambda_n\of{t}} &\leq L \int_{t_0}^{t} \abs{\lambda_n\of{s} - \lambda_{n-1}\of{s}} \dif s\\
            &\leq \frac{ML^n}{(n+1)!}\abs{t-t_0}^{n+1}
        \end{align*}
        \textsc{Schritt 3}:
        \begin{align*}
            \lambda_n\of{t} &= x_0 + \sum_{j=1}^{n} \pair{\lambda_j\of{t} - \lambda_{j-1}\of{t}}
            \intertext{Beobachten die Reihe $ \sum_{j=1}^{n} \pair{\lambda_j\of{t} - \lambda_{j-1}\of{t}}$ konvergiert absolut.}
            \sum_{j=1}^{\infty} \abs{\lambda_j\of{t} - \lambda_{j-1}\of{t}}&\leq \sum_{j=1}^{\infty} \frac{ML^{j-1}}{(j-1)!}\abs{t-t_0}^{j-1}\\
            \lambda_{n+1}\of{t} - x_0 &= \sum_{j=0}^{n} \pair{\lambda_{j-1}\of{t} - \lambda_j\of{t}}\\
            \sum_{j=0}^{\infty} \abs{\lambda_{j+1}\of{t} - \lambda_j\of{t}} &\leq \sum_{j=0}^{\infty} \frac{M}{L} \frac{L^j}{j!}\pair{t-t_0}^j\\
            &= \frac{M}{L} \exp\of{L\abs{t-t_0}}\\
            &\leq \frac{M}{L}\pair{\exp\of{L\alpha} - 1}\quad\forall \abs{t-t_0}\leq \alpha
            \intertext{$t\mapsto \lambda_n\of{t}$ ist stetig. Konvergieren gleichmäßig mit für $\abs{t-t_0}\leq \alpha$}
            \impl \lambda_{\infty}\of{t} &\coloneqq \lim_{n\toinf} \lambda_n\of{t}
        \end{align*}
        ist eine stetige Funktion.\\
        \textsc{Schritt 4}: Wir haben schon
        \begin{align*}
            \underbrace{\lambda_{n+1}\of{t}}_{\to \lambda_\infty\of{t}} &= x_0 + \int_{t_0}^{t} f\of{s, \underbrace{\lambda_n\of{s}}_{\to \lambda_{\infty}\of{s}}} \dif s
            \intertext{Da die rechte Konvergenz gleichmäßig ist, folgt damit}
            \impl \lambda_{\infty}\of{t} &= x_0 + \int_{t_0}^{t} f\of{s, \lambda_{\infty}\of{s}} \dif s\qedhere
        \end{align*}
    \end{proof}

    \begin{bemerkung}[Fehlerabschätzungen]
        Wir haben
        \begin{align*}
            \abs{\lambda_{n+1}\of{t} - \lambda_n\of{t}} &\leq M \frac{L^n}{(n+1)!}\abs{t-t_0}^{n+1}\\
            \impl \abs{\lambda_m\of{t} - \lambda_n\of{t}} &\leq \sum_{j=n}^{m-1} \abs{\lambda_{j-1}\of{t} - \lambda_j\of{t}}\\
            \abs{\lambda_{j+1}\of{t} - \lambda_j\of{t}} &\leq \int_{t_0}^{t} \abs{f\of{s, \lambda_j\of{t}} - f\of{s, \lambda_{j-1}\of{s}}} \dif s\\
            &\leq L \int_{t_0}^{t} \abs{\lambda_j\of{s} - \lambda_{j-1}\of{s}} \dif s
        \end{align*}
    \end{bemerkung}


\end{document}
